\documentclass[a4paper,12pt]{article}
\usepackage{styledoc19}


\begin{document} % конец преамбулы, начало документа
	
	\year{2020}
	\docNumber{RU.17701729.04.13-01 51 01-1}
	\docFormat{Программа и методика испытаний}
	\student{БПИ 174}{Д. Ю. Редникина}
	\supervisor{Профессор департамента \vfill программной инженерии  факультета компьютерных наук, к.т.н}
	{Е. М. Гринкруг}
	\project{СИСТЕМА УПРАВЛЕНИЯ ЗАДАНИЯМИ ПО АВТОМАТИЧЕСКОМУ СБОРУ ДАННЫХ ИЗ СЕТИ ИНТЕРНЕТ}
	
	\firstPage
	\newpage
	\secondPage
	\newpage
	\thirdPage
	\newpage
	
	\section{Объекты испытаний}
	\subsection{Наименование программы}
	Cистема управления заданиями по автоматическому сбору данных из сети Интернет
 (System for managing tasks of collecting data from the Internet
)
	\subsection{Область применения программы}
	\subsubsection{Функциональное назначение}
	Система будет применяться как средство управления проектами по созданию, редактированию и запуску веб краулеров для сбора данных в сети интернет. Продукт позволит следить за запусками в режиме реального времени, а также создавать периодические запуски по расписанию.

	\subsubsection{Эксплуатационное назначение}
	Web-приложение является компонентом CRM-системы для благотворительного фонда <<AIAIN>>, позволяющей облегчить бизнес процессы работы фонда с благополучателями и донорами. Web-приложение призвано обеспечить необходимую сотрудникам фонда функциональность для работы с системой. Им будут пользоваться как менеджеры фонда, так и администраторы, члены комиссий, операторы фонда, контент-менеджеры фонда. Каждый из пользователей будет иметь доступ к необходимой ему функциональности по обработке заявок, управлению фондом, администрированию и т.д.
	\subsubsection{Область применения}
	Технологии web-scraping используются как в науке, так и в бизнесе - многие люди чувствуют потребность в извлечении данных из HTML разметки интернет страниц. Существующие аналоги реализуют базовый функционал(сбор данных), но не предоставляют такие дополнительные возможности как периодический запуск или совместное редактирование. Многие аналоги (прим. scrapyd~\cite{scrapyd}) имеют ограниченный функционал.
Главные возможности, которыми продукт обеспечит предполагаемых пользователей:
\begin{itemize}
    \item Совместное управление запусками клаулеров
    \item Периодический запуск задач
    \item Сбор логов, ошибок
    \item Группировка краулеров,а также их запусков в проект 
    \item Бесплатная функциональность
\end{itemize}

					\newpage
	\section{Цель испытаний}
	Проверка на соответсвие требованиям, указанным в документе <<Cистема управления заданиями по автоматическому сбору данных из сети Интернет. Техническое задание>>.
					\newpage 
	\section{Требования к программе}
	\subsection{Требования к функциональным характеристикам}
	Программа выполняется в рамках темы курсовой работы в соответствии с учебным планом подготовки бакалавров по направлению 09.03.04 «Программная инженерия» Национального исследовательского университета «Высшая школа экономики», факультет компьютерных наук.
	
	Программа должна удовлетворять следующим требованиям:
	
	

\begin{enumerate}
		
		\item \textbf{Авторизация\\}
		Чтобы использовать сервис, клиентская программа должна иметь возможность авторизоваться в системе с помощью REST API
		\begin{enumerate}
			\item Для регистрации пользователю нужно указать следующие данные 
			\begin{enumerate}
			    \item Почта - уникальна для каждого зарегистрированного пользователя; 
			    \item Имя - длина больше 1 символ;
			    \item Логин - длина больше 2 символов;
			    \item Пароль - длина больше 2 символов;
			\end{enumerate}
			\item Для авторизации пользователя в системе должны быть указаны следующие данные
			\begin{enumerate}
			    \item Почта;
			    \item Пароль;
			\end{enumerate}
		\end{enumerate}
		
		\item \textbf{Проекты\\}
		Должны быть реализованы запросы REST API для предоставления клиенту следующей функциональности
		\begin{enumerate}
	    	\item Создание проекта со следующей информацией
	    	\begin{enumerate}
	    	    \item Имя проекта;
	    	    \item Описание проекта - опциональное поле;
	    	\end{enumerate}
			\item Обновление метаданных о проекте (редактирование) могут быть обновлены только участником с минимальным уровнем дотупа \texttt{ReadAndWrite}. Следующие данные могут быть обновлены:
			\begin{enumerate}
			    \item Имя проекта;
			    \item Описание проекта;
			    \item Настройки проекта для запуска краулеров;
			    \item Аргументы для запуска краулеров проекта;
			\end{enumerate}
			\item Обновление \texttt{egg} файла проекта (редактирование) -- минимальный уровень доступа участника, обновляющий данные о проекте \texttt{ReadAndWrite}.  
			\item Удаление данных о проекте. Удалить проект может только владелец \texttt{Owner}.
			\item Просмотр списка проектов (с пагинацией), к которым у пользователя есть как минимум \texttt{ReadOnly} доступ.
		\end{enumerate}
		
		\item \textbf{Участники проектов\\}
		Должны быть реализованы запросы REST API для предоставления клиенту следующей функциональности
		\begin{enumerate}
	    	\item Просмотр информации об участниках проекта;
	    	\begin{enumerate}
	    	    \item Имя, почта, логин участника;
	    	    \item Статус участника в проекте (\texttt{ReadOnly}, \texttt{ReadAndWrite} или \texttt{Owner});
	    	\end{enumerate}
			\item Обновление статуса участника проекта. Это действие совершать может только владелец проекта; 
			\item Удаление участника из проекта. Данное действие может совершать только владелец проекта;
			\item Добавление нового участника с указанными правами на редактирование. Данное действие может совершать только владелец проекта;
		\end{enumerate}
		
		\item \textbf{Краулеры\\}
		Должны быть реализованы запросы REST API для предоставления клиенту следующей функциональности
		\begin{enumerate}
			\item Просмотр списка краулеров проекта;
			\item Редактирование информации о краулере для последующих запусков. Следующая информация может быть изменена
			\begin{enumerate}
			    \item Настройки краулера для запуска;
			    \item Аргументы для запуска;
			\end{enumerate}
		\end{enumerate}
		
		\item \textbf{Запуски краулеров\\}
		Должны быть реализованы запросы REST API для предоставления клиенту следующей функциональности
		\begin{enumerate}
		    \item Просмотр списка запусков в определенном статусе (\texttt{Pending}, \texttt{Running} или \texttt{Finished}) с пагинацией, совершенных в проектах, к которым у пользователя есть как минимум \texttt{ReadOnly} доступ;
		    \item Редактирование запуска - остановка запуска, перевод его в состояние \texttt{Finished}. Операция может быть применена только к запускам в состоянии \texttt{Running} или \texttt{Pending};
		    \item Удаление запуска - удаление всех данных о запуске из базы данных. Операция может быть применена только к запускам в состоянии \texttt{Finished};
		    \item Создание запуска со следующей информацией
		    \begin{enumerate}
		        \item Краулер, с которым происходит запуск;
		        \item Настройки запуска --  это могут быть как и предопределенные настроки на \texttt{scrapyd} \footnote{\url{http://doc.scrapy.org/en/latest/topics/settings.html}}, так и собственные настройки;
		        \item Аргументы запуска -- аргументы для запуска краулера, которые передаются через командную строку;
		        \item Описание запуска;
		    \end{enumerate}
		\end{enumerate}
		
		\item \textbf{Периодические запуски\\}
		Должны быть реализованы запросы REST API для предоставления клиенту следующей функциональности
		\begin{enumerate}
		    \item Просмотр списка периодических запусков с пагинацией;
		    \item Редактирование следующей информации о периодическом запуске
		    \begin{enumerate}
		        \item Настройки будущих запусков --  это могут быть как и предопределенные настроки на \texttt{scrapyd}, так и собственные настройки;
		        \item Аргументы будущих запусков -- аргументы для запуска краулера, которые передаются через командную строку;
		        \item Краулер, с помощью которого будет совершен запуск;
		        \item cron-expression расписания запуска;
		    \end{enumerate}
		    \item Удаление периодического запуска;
		    \item Отмена последующих запусков - перевод периодической задачи в состояние \texttt{Disabled};
		    \item Возобновление запусков - перевод периодической задачи в состояние \texttt{Enabled};
		    \item Создание периодического запуска со следующими данными
		    \begin{enumerate}
		        \item Название;
		        \item Описание -- опциональное;
		        \item Краулер;
		        \item Приоритетность, влияющая на очередь запусков (\texttt{Low}, \texttt{Normal} или \texttt{High});
		        \item Статус (\texttt{Enabled} или \texttt{Disabled});
		        \item Настройки будущих запусков --  это могут быть как и предопределенные настроки на \texttt{scrapyd}, так и собственные настройки;
		        \item Аргументы будущих запусков -- аргументы для запуска краулера, которые передаются через командную строку;
		    \end{enumerate}
		\end{enumerate}
	
	\end{enumerate}

		
	
	
	
	\subsection{Требования к интерфейсу}
	Доступ к программе должен быть предоставлен посредством интерфейса консольной программы.
	\subsection{Требования к надежности}
	\begin{itemize}
	    \item Программа должна осуществлять проверку корректности посылаемого текста в кодировке utf-8 и возвращать ошибку в случае наличия невалидных символов.
        \item Программа должна работать непрерывно (не должно быть ситуации, в которой сервер возвращает status code 500).
        \item В базу данных не должны заноситься некорректные данные.
	\end{itemize}
						\newpage
	\section{Требования к программной документации}
	\subsection{Состав программной документации}
	Состав программной документации должен включать в себя следующие компоненты:
\begin{enumerate}
	\item Техническое задание <<Cистема управления заданиями по автоматическому сбору данных из сети Интернет
>> (ГОСТ 19.201-78) \label{tz}
	\item Программа и методика испытаний <<Cистема управления заданиями по автоматическому сбору данных из сети Интернет
>> (ГОСТ 19.301-78) \label{pmi}
	\item Пояснительная записка <<Cистема управления заданиями по автоматическому сбору данных из сети Интернет
>> (ГОСТ 19.404-79) \label{pz}
	\item Руководство оператора <<Cистема управления заданиями по автоматическому сбору данных из сети Интернет
>> (ГОСТ 19.505-79) \label{ro}
	\item Текст программы <<Cистема управления заданиями по автоматическому сбору данных из сети Интернет
>> (ГОСТ 19.401-78) \label{tp}
\end{enumerate}

\indent
Вся документация должна быть составлена согласно ЕСПД (ГОСТ 19.101-77, 19.104-78, 19.105-78, 19.106-78 и ГОСТ к соответствующим документам (см. выше)) \cite{gost}. Все документы сдаются в электронном виде в составе курсовой работы LMS НИУ ВШЭ.

Пояснительная записка <<Cистема управления заданиями по автоматическому сбору данных из сети Интернет
>> должна быть проверена на плагиат ($< 40\% $ заимствований). Документ, подтвержадющий проверку Пояснительной записки сдается в печатном виде вместе с подписанным отзывом от научного руководителя.

						\newpage
	\section{Средства и порядок испытаний}
	\subsection{Технические средства, используемые во время испытаний}
	\begin{enumerate}
        \item Компьютер оснащенный процессором Intel Core i5 с тактовой частотой 2,3 ГГц;
        \item 16 Гб ОЗУ;
        \item Жесткий диск с объемом свободной памяти более чем 50 ГБ;
        \item Клавиатура и мышь;
        \item Доступ в интернет.
\end{enumerate}
	\subsection{Программные средства, используемые во время испытаний}
	\begin{enumerate}
        \item macOS 10.15.2;
        \item scrapyd~\cite{scrapyd};
        \item Scala 2.12.6;
        \item Play-framework 2.6.13;
        \item PostgreSQL 11~\cite{postgresql};
    \end{enumerate}
	\subsection{Порядок проведения испытаний}
	В процессе разработки программы были написаны unit-тесты, а также автотесты, тестирующие API сервера.
	
	Автотесты исполняются на отдельной тестовой базе данных, которая перед началом каждого теста заполняется тестовыми данными. 
	
	Для запуска автотестов пользователю надо ввести команду \texttt{sbt test} в директории проекта \texttt{cw\_backend\_scala}, перед этим в отдельной директории запустить сервер scrapyd~\cite{scrapyd} командой \texttt{scapyd}.
						\newpage
	\section{Методы испытаний}
	\subsection{Проверка требований к функциональным характеристикам}
	Для тестирования API запросов были написаны автотесты с помощью библиотеки scralatestplus~\cite{scalatestplus}. Данная библиотека позволяет подменять зависимости основной программы и использовать \texttt{FakeRequest} как упрощенный способ выполнять запросы к контроллерам приложения. 
	
	Также для тестирования приложения используется библиотека play-silhouette-testkit~\cite{playsilhouettetestkit}. Эта библиотека позволяет авторизовываться в приложении без использования cookies.
	
	Для тестирования API запросов были написаны следующие тест-кейсы:
	
	\subsubsection{Регистрация}
	\begin{table}[htb]
		\centering
		\begin{tabular}{|p{0.02\linewidth}|p{0.15\linewidth}|p{0.2\linewidth}|p{0.3\linewidth}|p{0.2\linewidth}|} 
			\hline
			\textbf{№} & \textbf{Название} & \textbf{Предусловие} & \textbf{Описание} & \textbf{Ожидаемый результат} \\ \hline
			1 & Регистрация пользователя & Пользователь неавторизован & Пользователь отправляет запрос POST /api/signup с персональными данными в формате app.forms.SignUpForm & Status code 200. Cookies в хэдере \\ \hline
			2 & Регистрация пользователя с уже зарегистрированным email или логином & Пользователь неавторизован. В БД системы уже существует пользователь с вводимым логином и/или email. & Пользователь отправляет запрос POST /api/signup с персональными данными, которые совпадают с login и/или email другого пользователя в базе данных системы. & Status code 409. Сообщение об ошибке <<User Already Exists Message>> \\ \hline
		\end{tabular}
	\caption{Тест кейсы: регистрация}
	\end{table}
	
	\clearpage
	\subsubsection{Авторизация}
	\begin{table}[htb]
		\centering
		\begin{tabular}{|p{0.02\linewidth}|p{0.15\linewidth}|p{0.2\linewidth}|p{0.3\linewidth}|p{0.2\linewidth}|} 
			\hline
			\textbf{№} & \textbf{Название} & \textbf{Предусловие} & \textbf{Описание} & \textbf{Ожидаемый результат} \\ \hline
			1 & Авторизация пользователя & Пользователь неавторизован в системе. Пользователь был ранее зарегистрирован в системе & Пользователь отправляет запрос на POST /api/signin с персональными данными в формате app.forms.SignInForm & Status code 200. Cookies в хэдере \\ \hline
			2 & Авторизация пользователя с неправильным паролем  & Пользователь неавторизован в системе. Пользователь был ранее зарегистрирован в системе. & Пользователь отправляет запрос на POST /api/signin с персональными данными (email, пароль) так, что пароль не совпадает с уже введенным при регистрации. & Status code 403. \\
			\hline
			3 & Авторизация пользователя с несуществующими в системе credentials & Пользователь неавторизован в системе. Пользователь был ранее зарегистрирован в системе. & Пользователь отправляет запрос на POST /api/signin с персональными данными (email, пароль), которые не совпадают ни с одной парой credentials, хранящихся в БД. & Status code 409. \\ \hline
		\end{tabular}
	\caption{Тест кейсы: авторизация}
	\end{table}

    \clearpage
	\subsubsection{Проекты (\ref{terms:project})}
	
	\begin{table}[htb]
		\centering
		\begin{tabular}{|p{0.02\linewidth}|p{0.15\linewidth}|p{0.2\linewidth}|p{0.3\linewidth}|p{0.2\linewidth}|} 
			\hline
			\textbf{№} & \textbf{Название} & \textbf{Предусловие} & \textbf{Описание} & \textbf{Ожидаемый результат} \\ \hline
			1 & Запрос на список проектов без пагинации & Пользователь авторизован в системе. У пользователя имеются 20 проектов. & Пользователь отправляет запрос GET /api/projects/10 для получения первых 10-и из списка проектов, к которому у него есть доступ. & Status code 200. Json со списком проектов, к которым у пользователя есть как минимум readonly access, в хронологическом порядке их создания. \\ \hline
			2 & Запрос на список проектов с пагинацией & Пользователь авторизован в системе. У пользователя имеются 20 проектов. & Пользователь отправляет запрос GET с query argument limit=10 и id для получения 10-и проектов из списка, которые в хронологическом порядке (по дате создания) располагаются раньше чем указанный проект id. & Status code 200. Json со списком проектов, к которым у пользователя есть как минимум readonly access, id которых >= idProject \\ 	\hline
\end{tabular}
	\caption{Тест кейсы: просмотр списка проектов}
	\end{table}
	
\begin{table}[htb]
		\centering
		\begin{tabular}{|p{0.02\linewidth}|p{0.15\linewidth}|p{0.2\linewidth}|p{0.3\linewidth}|p{0.2\linewidth}|} 
			\hline
			\textbf{№} & \textbf{Название} & \textbf{Предусловие} & \textbf{Описание} & \textbf{Ожидаемый результат} \\ \hline
			3 & Создание проекта & Пользователь авторизован. & Пользователь отправляет запрос POST /api/projects с телом запроса в виде app.forms.ProjectCreForm &  Status code 200 и json с id созданного проекта. \\ \hline
			
\end{tabular}
	\caption{Тест кейсы: создание проекта}
	\end{table}
	
\begin{table}[htb]
		\centering
		\begin{tabular}{|p{0.02\linewidth}|p{0.15\linewidth}|p{0.2\linewidth}|p{0.3\linewidth}|p{0.2\linewidth}|} 
			\hline
			\textbf{№} & \textbf{Название} & \textbf{Предусловие} & \textbf{Описание} & \textbf{Ожидаемый результат} \\ \hline
			4 & Редактирова- ние проекта & Пользователь авторизован. У пользователя есть хотя бы 1 созданный проект. Пользователь имеет owner доступ к проекту. & Пользователь отправляет запрос PUT /api/projects/{id} с указанием id проекта для редактирования, с телом запроса в виде app.forms.ProjectForm. & Status code 200 и json с id созданного проекта. В таблице projects должна обновиться запись с id созданного проекта, где name, description, spiderSettings совпадают с переданными пользователем. \\ \hline
			5 & Редактирова- ние проекта: доступ пользователя readonly & Пользователь авторизован. У пользователя доступ к проекту id - readonly & Пользователь с отправляет PUT /api/projects/{id} с корректным телом запроса (см. п. 4) & Status code 403 \\ \hline
			6 & Редактирова- ние проекта: доступ пользователя writeandread & Пользователь авторизован. У пользователя доступ к проекту id - writeandread & Пользователь с отправляет PUT /api/projects/{id} с корректным телом запроса (см. п. 4) & Status code 200. Информация о проекте успешно обновлена \\ \hline 
\end{tabular}
	\caption{Тест кейсы: проекты - редактирование метаданных}
	\end{table}
\begin{table}[htb]
		\centering
		\begin{tabular}{|p{0.02\linewidth}|p{0.15\linewidth}|p{0.2\linewidth}|p{0.3\linewidth}|p{0.2\linewidth}|} 
			\hline
			\textbf{№} & \textbf{Название} & \textbf{Предусловие} & \textbf{Описание} & \textbf{Ожидаемый результат} \\ \hline
			7 & Удаление проекта владельцем & Пользователь авторизован. У пользователя есть доступ к проекту id с правами owner. (таблица membership) & Пользователь отправляет  DELETE   /api/projects/:projectId
запрос. & Status code 200, в таблице membership нет никаких записей о проекте с id, в таблице project тоже. \\ \hline
            8 & Удаление проекта: неверный id & Пользователь авторизован. Id проекта нет в таблице membership.  & Пользователь отправляет  DELETE   /api/projects/:projectId
запрос. & Status code 403 \\ \hline
            9 & Удаление проекта: права на проект НЕ owner & Пользователь авторизован. У пользователя права на проект НЕ owner. & Пользователь отправляет  DELETE   /api/projects/:projectId
запрос. & Status code 403 \\ \hline
\end{tabular}
	\caption{Тест кейсы: проект DELETE}
	\end{table}
\begin{table}[htb]
		\centering
		\begin{tabular}{|p{0.02\linewidth}|p{0.15\linewidth}|p{0.2\linewidth}|p{0.3\linewidth}|p{0.2\linewidth}|} 
			\hline
			\textbf{№} & \textbf{Название} & \textbf{Предусловие} & \textbf{Описание} & \textbf{Ожидаемый результат} \\ \hline
            10 & Deploy архива проекта & Пользователь авторизован. У пользователя есть права на редактирование проекта (либо owner, либо writeandread) & Пользователь делает запрос по адресу PUT      /api/projects/:projectId/ deploy с телом eggFile: <файл с расширением egg> & Status code 200. В таблице проектов у проекта с id появился bytes в колонке eggFile \\ \hline
            11 & Deploy архива проекта: файл неверного формата & Пользователь авторизован. У пользователя есть права на редактирование проекта (либо owner, либо writeandread) & Пользователь делает запрос по адресу PUT      /api/projects/:projectId/ deploy с телом eggFile: файл не того формата (не egg). & Status code 422 \\ \hline
            12 & Deploy архива проекта: права доступа не позволяют менять проект & Пользователь авторизован. У пользователя есть права на редактирование проекта readonly & Пользователь делает запрос по адресу PUT      /api/projects/:projectId/ deploy с телом eggFile: <файл с расширением egg> & Status code 403 \\ \hline
			
		\end{tabular}
	\caption{Тест кейсы: проект PUT - deploy}
	\end{table}
	\clearpage
	
	\subsubsection{Краулеры (\ref{terms:spider})}
	
	\begin{table}[htb]
		\centering
		\begin{tabular}{|p{0.02\linewidth}|p{0.15\linewidth}|p{0.2\linewidth}|p{0.3\linewidth}|p{0.2\linewidth}|} 
			\hline
			\textbf{№} & \textbf{Название} & \textbf{Предусловие} & \textbf{Описание} & \textbf{Ожидаемый результат} \\ \hline
            1 &  Отображения списка краулеров  & Пользователь авторизован и имеет доступ к проекту id минимум readonly. & Пользователь отправляет запрос GET      /api/projects/:projectId /crawlers с id проекта.
 & Status code 200, список пауков в проекте из БД \\ \hline
            2 & Отображение списка краулеров: нет доступа к проекту & Пользователь авторизован и не имеет доступ к проекту id & Пользователь отправляет запрос GET      /api/projects/:projectId /crawlers с id проекта. & Status code 403 \\ \hline
			
		\end{tabular}
	\caption{Тест кейсы: краулеры GET}
	\end{table}
	
	
	\begin{table}[htb]
		\centering
		\begin{tabular}{|p{0.02\linewidth}|p{0.15\linewidth}|p{0.2\linewidth}|p{0.3\linewidth}|p{0.2\linewidth}|} 
			\hline
			\textbf{№} & \textbf{Название} & \textbf{Предусловие} & \textbf{Описание} & \textbf{Ожидаемый результат} \\ \hline
			1 & Изменение настроек паука & Пользователь авторизован. У пользователя есть write или owner доступ к проекту. & Пользователь отправляет запрос PUT /api/project/:projectId /crawlers. & Status code 200 \\ \hline
			2 & Изменение настроек паука: нет прав доступа к редактированию & Пользователь авторизован. У пользователя есть readonly к проекту. & Пользователь отправляет запрос PUT /api/project/:projectId /crawlers. & Status code 403 \\ \hline
			3 & Изменение настроек паука: не совпадает id проекта с id паука & Пользователь авторизован. & Пользователь отправляет запрос PUT /api/project/:projectId /crawlers. & Status code 403 \\ \hline
	
	\end{tabular}
	\caption{Тест кейсы: краулеры PUT}
	\end{table}
	
	
	\subsubsection{Запуски (\ref{terms:job})}
	
	\begin{table}[H]
		\centering
		\begin{tabular}{|p{0.02\linewidth}|p{0.15\linewidth}|p{0.2\linewidth}|p{0.3\linewidth}|p{0.2\linewidth}|} 
			\hline
			\textbf{№} & \textbf{Название} & \textbf{Предусловие} & \textbf{Описание} & \textbf{Ожидаемый результат} \\ \hline
            1 &  Список джобов  & Пользователь авторизован. У пользователя есть список из 20 запущенных job-ов в статусе finished, запущенных с пауками из разных проектов. & Пользователь отправляет запрос GET      /api/projects/jobs/:limit, где указывает limit=10 и тип джобы finished. После этого отправляет тот же запрос со статусом running
 & Status code 200. Возвращается массив джобов со статусами finished и длиной 10. Отсортирован по id.  \\ \hline
            2 & Список джобов: пагинация & Пользователь авторизован. У пользователя есть список из 20 запущенных job-ов в статусе finished, запущенных с пауками из разных проектов. & Пользователь отправляет запрос GET      /api/projects/jobs/:limit, где указывает limit=10, excludiveFrom=3 и тип джобы finished
 & Status code 200. Возвращается массив джобов со статусами finished и длиной 2. Отсортирован по id. \\ \hline
			
		\end{tabular}
	\caption{Тест кейсы: список запусков GET}
	\end{table} 
	
	\begin{table}[hbt]
		\centering
		\begin{tabular}{|p{0.02\linewidth}|p{0.15\linewidth}|p{0.2\linewidth}|p{0.3\linewidth}|p{0.2\linewidth}|} 
			\hline
			\textbf{№} & \textbf{Название} & \textbf{Предусловие} & \textbf{Описание} & \textbf{Ожидаемый результат} \\ \hline
	3 & Создание запуска & Пользователь авторизован. Имеет ReadAndWrite или Owner доступ к проекту projectId.	& Клиент отправляет запрос POST     /api/projects/:projectId/ jobs & Status 200. Результат - id созданной задачи. При запросе GET /api/jobs status = Running задача c id возвращается. \\ \hline
	
	4 & Создание запуска: ReadOnly доступ & Пользователь авторизован. Имеет ReadOnly доступ к проекту projectId. & Клиент отправляет запрос POST     /api/projects/:projectId/ jobs & Status 403. Сообщение об ошибке <<NoPermission>>. \\ \hline
	
	5 & Создание запуска: id краулера не соответсвует id проекта & Пользователь авторизован. Имеет ReadAndWrite доступ к проекту projectId. Id паука не соответсвует id проекта. & Клиент отправляет запрос POST     /api/projects/:projectId/ jobs & Status 403. Сообщение об ошибке <<Crawler Doesnt Correspond To Project>> \\ \hline
	
	\end{tabular}
	\caption{Тест кейсы: создание запуска POST}
	\end{table} 
	
	\begin{table}[hbt]
		\centering
		\begin{tabular}{|p{0.02\linewidth}|p{0.15\linewidth}|p{0.2\linewidth}|p{0.3\linewidth}|p{0.2\linewidth}|} 
			\hline
			\textbf{№} & \textbf{Название} & \textbf{Предусловие} & \textbf{Описание} & \textbf{Ожидаемый результат} \\ \hline
	6 & Отмена джобы & Пользователь запустил джобу. Работа находится в статусе pending или running. & Клиент отправляет запрос PUT      /api/projects/:projectId/ jobs/:jobScrapydId/:jobId, где jobScrapydId, jobId - id джобы в статусе running. & Status 200. Возвращается объект JobExecution с обновленными данными: status = finished. \\ \hline
	
	7 & Отмена джобы: id проекта не совпадает с id джобы & Пользователь авторизован в приложении. В проекте с id нет запущенных джоб. & Клиент отправляет запрос PUT      /api/projects/:projectId/ jobs/:jobId. & Status 403. Сообщение об ошибке <<No Project Found>>. \\ \hline
	
	8 &  Отмена джобы: у пользователя readonly права & Пользователь авторизован в приложении. Имеет readonly access к проекту projectId. & Клиент отправляет запрос PUT      /api/projects/:projectId/ jobs/:jobId. & Status 403. Сообщение об ошибке <<No Permission>>. \\ \hline
	
	\end{tabular}
	\caption{Тест кейсы: отмена запуска PUT}
	\end{table} 
	
	\begin{table}[hbt]
		\centering
		\begin{tabular}{|p{0.02\linewidth}|p{0.15\linewidth}|p{0.3\linewidth}|p{0.2\linewidth}|p{0.2\linewidth}|} 
			\hline
			\textbf{№} & \textbf{Название} & \textbf{Предусловие} & \textbf{Описание} & \textbf{Ожидаемый результат} \\ \hline
	9 & Удаление запуска & Пользователь авторизован. Имеет ReadAndWrite доступ к проекту projectId. Джоба находится в статусе finished. & Клиент отправляет запрос DELETE   /api/projects/ :projectId/ jobs/:jobId & Status 200. Результат - id удаленной джобы.  \\ \hline 
	
	10 & Удаление запуска: джоба в статусе running/ pending & Пользователь авторизован. Имеет ReadAndWrite доступ к проекту projectId. Джоба находится в статусе running/pending. & Клиент отправляет запрос DELETE   /api/projects/ :projectId/ jobs/:jobId & Status 422. Сообщение об ошибке <<Couldn't delete job execution>>. \\ \hline
	
	11 & Удаление запуска: ReadOnly доступ к проекту & Пользователь авторизован. Имеет ReadOnly доступ к проекту projectId. Джоба находится в статусе finished. & Клиент отправляет запрос DELETE   /api/projects/ :projectId/ jobs/:jobId  & Status 403. Сообщение об ошибке <<No access>>. \\ \hline
	
	12 & Удаление запуска: id работы не соответсвует id проекта & Пользователь авторизован. Имеет ReadAndWrite доступ к проекту projectId. Id джобы не соответсвует id проекта. & Клиент отправляет запрос DELETE   /api/projects/ :projectId/ jobs/:jobId & Status 403. Сообщение об ошибке <<Job Execution Doesnt Correspond To Project>> \\ \hline
	
	\end{tabular}
	\caption{Тест кейсы: удаление запуска DELETE}
	\end{table} 
	
	\clearpage
	\subsubsection{Периодические задачи (\ref{terms:pjob})}
	
	\begin{table}[hbt]
		\centering
		\begin{tabular}{|p{0.02\linewidth}|p{0.15\linewidth}|p{0.25\linewidth}|p{0.25\linewidth}|p{0.2\linewidth}|} 
			\hline
			\textbf{№} & \textbf{Название} & \textbf{Предусловие} & \textbf{Описание} & \textbf{Ожидаемый результат} \\ \hline
			1 & Список периодических задач & Пользователь авторизован и имеет Owner доступ к проекту с номером projectId.  &  Пользователь отправляет запрос GET api/projects/:projectId/  periodicJobs/:limit, где limit = 5 & Статус-код 200. Json с массивом из 5 периодических задач проекта в порядке убывания id. \\ \hline
			2 & Список периодических задач c пагинацией & Пользователь авторизован и имеет Owner доступ к проекту с номером pId. Всего периодических задач в проекте 10 штук.  & Пользователь отправляет запрос GET api/projects/:projectId/  periodicJobs/:limit/ :exclusiveFrom, где limit = 5, exclusiveFrom = 2. & Статус-код 200. Json с массивом из 2 периодических задач проекта в порядке убывания id (2, 1). \\ \hline
			3 & Список периодических задач: доступа нет & Пользователь авторизован и не имеет доступа к проекту с номером pId. & Пользователь отправляет запрос GET api/projects/:projectId/ periodicJobs/ :limit с любым параметром limit. & Статус-код 403. Сообщение “NoPermission” \\ \hline
			\end{tabular}
	\caption{Тест кейсы: список периодических задач GET}
	\end{table} 
	
	\begin{table}[hbt]
		\centering
		\begin{tabular}{|p{0.02\linewidth}|p{0.15\linewidth}|p{0.2\linewidth}|p{0.3\linewidth}|p{0.2\linewidth}|} 
			\hline
			\textbf{№} & \textbf{Название} & \textbf{Предусловие} & \textbf{Описание} & \textbf{Ожидаемый результат} \\ \hline
			1 & Создание периодического запуска & Пользователь авторизован и имеет доступ ReadAndWrite к проекту projectId. & Пользователь отправляет запрос POST     /api/projects/:projectId/ periodicJobs с телом запроса PeriodicJobCreateForm. & Статус-код 200. Возвращается JSON с id созданной периодической задачи. \\ \hline
			
			2 & Создание периодического запуска: некорректный cron-expression & Пользователь авторизован и имеет доступ ReadAndWrite к проекту projectId.  & Пользователь отправляет запрос POST     /api/projects/:projectId/ periodicJobs с телом запроса PeriodicJobCreateForm, но cron-expression указывает в теле запроса некорректный (прим. <<text>>). & Статус-код 420. Возвращается сообщение с ошибкой <<invalid cron expression>>. \\ \hline
			
			3 & Создание периодического запуска: права ReadOnly & Пользователь авторизован и имеет доступ ReadOnly к проекту projectId.  &  Пользователь отправляет запрос POST     /api/projects/:projectId/ periodicJobs с телом запроса PeriodicJobCreateForm. & Статус-код 403. Сообщение об ошибке <<NoPermission>> \\ \hline
			
			4 & Создание периодического запуска: паук не соответсвует проекту & Пользователь авторизован и имеет доступ ReadAndWrite к проекту projectId. Id паука в теле запроса не соответствует projectId. & Пользователь отправляет запрос POST     /api/projects/:projectId /periodicJobs с телом запроса PeriodicJobCreateForm. & Статус-код 403. Сообщение об ошибке <<No Corresponding Crawler>> \\ \hline
	\end{tabular}
	\caption{Тест кейсы: создание периодических запусков POST}
	\end{table} 	
	
	
	\begin{table}[hbt]
		\centering
		\begin{tabular}{|p{0.02\linewidth}|p{0.2\linewidth}|p{0.25\linewidth}|p{0.28\linewidth}|p{0.15\linewidth}|} 
			\hline
			\textbf{№} & \textbf{Название} & \textbf{Предусловие} & \textbf{Описание} & \textbf{Ожидаемый результат} \\ \hline
			
			1 & Редактирование периодического запуска & Пользователь авторизован и имеет ReadAndWrite доступ к проекту projectId, в котором есть периодическая задача periodicJobId, а также паук crawlerId. & Пользователь отправляет запрос PUT      /api/projects/:projectId/ periodicJobs/ :periodicJobId с телом запроса PeriodicJobChangeForm. & Статус-код 200. \\ \hline
			
			2 & Редактирование периодического запуска: некорректный cron-expression & Пользователь авторизован и имеет ReadAndWrite доступ к проекту projectId, в котором есть периодическая задача periodicJobId, а также паук crawlerId. & Пользователь отправляет запрос PUT      /api/projects/:projectId/ periodicJobs/ :periodicJobId с телом запроса PeriodicJobChangeForm, но cron-expression указывает в теле запроса некорректный (прим. “text”). & Статус-код 422. Сообщение об ошибке <<invalid cron expression>> \\ \hline
			
			3 & Редактирование периодического запуска: в проекте нет паука & Пользователь авторизован и имеет ReadAndWrite доступ к проекту projectId, в котором есть периодическая задача periodicJobId. В проекте нет паука с crawlerId. & Пользователь отправляет запрос PUT      /api/projects/:projectId/ periodicJobs/ :periodicJobId с телом запроса PeriodicJobChangeForm с некорректным crawlerId в теле. & Статус-код 403 и сообщение об ошибке. \\ \hline
			
			4 & Редактирование периодического запуска: пользователь имеет ReadOnly доступ к проекту & Пользователь авторизован и имеет ReadOnly доступ к проекту projectId. & Пользователь отправляет запрос PUT      /api/projects/:projectId/ periodicJobs/ :periodicJobId с телом запроса PeriodicJobChangeForm. & Статус-код 403 и сообщение об ошибке. \\ \hline
			
			5 & Редактирование периодического запуска: в проекте нет редактируемой периодической задачи & Пользователь авторизован и имеет ReadAndWrite доступ к проекту projectId. В проекте нет редактируемой задачи periodicJobId. & Пользователь отправляет запрос PUT      /api/projects/:projectId/ periodicJobs/ :periodicJobId с телом запроса PeriodicJobChangeForm. & Статус-код 403 и сообщение об ошибке. \\ \hline
	\end{tabular}
	\caption{Тест кейсы: редактирование информации о периодическом запуске PUT}
	\end{table} 
	
	\begin{table}[hbt]
		\centering
		\begin{tabular}{|p{0.02\linewidth}|p{0.17\linewidth}|p{0.25\linewidth}|p{0.28\linewidth}|p{0.15\linewidth}|} 
			\hline
			\textbf{№} & \textbf{Название} & \textbf{Предусловие} & \textbf{Описание} & \textbf{Ожидаемый результат} \\ \hline
			
			1 & Отмена запусков периодической задачи & Пользователь авторизован и имеет ReadAndWrite доступ к проекту projectId, у которого есть периодическая задача periodicJobId в статусе Enabled. & Пользователь отправляет запрос PUT      /api/projects/:projectId/ periodicJobs/ :periodicJobId/disable. & Статус-код 200. \\ \hline
	
	        2 & Отмена запусков периодической задачи: задача уже отменена & Пользователь авторизован и имеет ReadAndWrite доступ к проекту projectId, у которого есть периодическая задача periodicJobId в статусе Disabled. & Пользователь отправляет запрос PUT      /api/projects/:projectId/ periodicJobs/ :periodicJobId/disable. &  Статус-код 422.  \\ \hline
	        
	\end{tabular}
	\caption{Тест кейсы: отмена запуска периодических задач PUT}
	\end{table} 
	
	\begin{table}[hbt]
		\centering
		\begin{tabular}{|p{0.02\linewidth}|p{0.17\linewidth}|p{0.25\linewidth}|p{0.28\linewidth}|p{0.15\linewidth}|} 
			\hline
			\textbf{№} & \textbf{Название} & \textbf{Предусловие} & \textbf{Описание} & \textbf{Ожидаемый результат} \\ \hline
			
			1 & Возобновление запусков периодической задачи & Пользователь авторизован и имеет ReadAndWrite доступ к проекту projectId, у которого есть периодическая задача periodicJobId в статусе Enabled. & Пользователь отправляет запрос PUT      /api/projects/:projectId/ periodicJobs/ :periodicJobId/enable. & Статус-код 200. \\ \hline
	
	        2 & Возобновление запусков периодической задачи: задача уже в статусе enabled & Пользователь авторизован и имеет ReadAndWrite доступ к проекту projectId, у которого есть периодическая задача periodicJobId в статусе Disabled. & Пользователь отправляет запрос PUT      /api/projects/ :projectId/ periodicJobs/ :periodicJobId/enable. &  Статус-код 422.  \\ \hline
	        
	\end{tabular}
	\caption{Тест кейсы: возобновление запусков периодических задач по расписанию PUT}
	\end{table} 
	
	
	\begin{table}[hbt]
		\centering
		\begin{tabular}{|p{0.02\linewidth}|p{0.17\linewidth}|p{0.25\linewidth}|p{0.28\linewidth}|p{0.15\linewidth}|} 
			\hline
			\textbf{№} & \textbf{Название} & \textbf{Предусловие} & \textbf{Описание} & \textbf{Ожидаемый результат} \\ \hline
	
	        1 & Удаление периодической задачи & Пользователь авторизован и имеет ReadAndWrite доступ к проекту projectId. У проекта есть периодическая задача periodicJobId в статусе enabled. & Пользователь отправляет запрос DELETE   /api/projects/:projectId/ periodicJobs/ :periodicJobId. & Статус код 200. \\ \hline
	        
	        2 & Удаление периодической задачи: задача в статусе disabled  & Пользователь авторизован и имеет ReadAndWrite доступ к проекту projectId. У проекта есть периодическая задача periodicJobId в статусе disabled. & Пользователь отправляет запрос DELETE   /api/projects/:projectId/ periodicJobs/ :periodicJobId. & Статус код 200. Возвращается id удаленной задачи. \\ \hline
	        
	        3 & Удаление периодической задачи: доступ Readonly & Пользователь авторизован и имеет ReadOnly доступ к проекту projectId. & Пользователь отправляет запрос DELETE   /api/projects/:projectId/ periodicJobs/ :periodicJobId. & Статус код 403. Сообщение об ошибке <<NoPermission>> \\ \hline
	\end{tabular}
	\caption{Тест кейсы: удаление периодических зада DELETE}
	\end{table} 
	
	
	
	\clearpage
	\subsubsection{Результаты тестирования}
	
    	\begin{lstlisting}[frame=single, basicstyle=\footnotesize\ttfamily, label={lst:tests}, caption={Результаты тестирования в консоли},captionpos=b]

[info] EmailValidatorTest:
[info] - EmailString.wrongEmail
[info] - EmailString.validEmail
[info] AuthorizationSpec:
[info] Authorize person
[info] - should signUp: OK
[info] - should signUp: userAlreadyExists
[info] - should signUp: invalid email format
[info] - should signIn: wrong credentials
[info] JobTestCase:
[info] CrawlerSpec:
[info] - GET crawlers: OK
[info] - GET crawlers: no access
[info] - PUT crawlers
[info] - PUT crawlers: ReadOnly access
[info] - PUT crawlers: spider not found
[info] ProjectSpec:
[info] ProjectsController
[info] - should GET list of projects for user
[info] - should GET list of projects: with pagination
[info] - should CREATE project
[info] - should PUT project's metadata
[info] - should PUT project's metadata: Readonly access - no permission
[info] - should PUT project's metadata: ReadAndWrite access
[info] - should DELETE project: Owner access
[info] - should DELETE project: NOT Owner access
[info] - should DELETE project: doesn't exist
[info] - should PUT deploy
[info] - should PUT deploy: wrong format file
[info] - should PUT deploy: no access
[info] MembershipSpec:
[info] MembershipController
[info] - should GET members: ReadAndWrite access
[info] - should GET members: no access to project
[info] - should DELETE member: Owner access
[info] - should DELETE member: ReadAndWrite access
[info] - should PUT member: Owner access
[info] - should PUT member: ReadAndWrite access
[info] ApplicationSpec:
[info] ApplicationController Logout GET
[info] - should should be unauthorized error
[info] - should redicrect if user was found
[info] PeriodicJobSpec:
[info] PeriodicJobController
[info] - should GET periodic jobs: basic
[info] - should GET periodic jobs: pagination
[info] - should GET periodic jobs: no access to project
[info] - should POST periodic job: basic
[info] - should POST periodic job: invalid cron-expression
[info] - should POST periodic job: crawler doesn't correspond to project
[info] - should POST periodic job: ReadOnly access
[info] - should PUT periodic job: basic
[info] - should PUT periodic job: crawler does not correspond to project
[info] - should PUT periodic job: wrong cron-expression
[info] - should DELETE periodic job: enabled status
[info] - should DELETE periodic job: disabled status
[info] - should DELETE periodic job: readonly access
[info] - should DELETE periodic job: no existing job with id
[info] - should PUT cancel periodic job: basic
[info] - should PUT cancel periodic job: already disabled
[info] - should PUT cancel periodic job: readonly access
[info] - should PUT cancel periodic job: no existing job with id
[info] - should PUT enable periodic job: basic
[info] - should PUT enable periodic job: already enabled
[info] - should PUT enable periodic job: readonly access
[info] - should PUT enable periodic job: no existing job found with id
[info] SettingsMergerTest:
[info] - SettingsMerger.basic
[info] - SettingsMerger.wrongInput
[info] - SettingsMerger.testPriority
[info] JobExecutionSpec:
[info] JobsController
[info] - should GET jobs
[info] - should GET jobs: pagination
[info] - should POST schedule: ordinary
[info] - should POST schedule: ReadOnly access
[info] - should POST schedule: ProjectId doesn't match CrawlerId
[info] - should PUT cancel
[info] - should PUT cancel: jobId doesn't match to projectId
[info] - should PUT cancel: ReadOnly access
[info] - should DELETE job
[info] - should DELETE job: still running
[info] - should DELETE job: ReadOnly access
[info] - should DELETE job: jobId doesn't match projectId
[info] ScalaTest
[info] Run completed in 1 minute, 33 seconds.
[info] Total number of tests run: 68
[info] Suites: completed 10, aborted 0
[info] Tests: succeeded 68, failed 0, canceled 0, ignored 0, pending 0
[info] All tests passed.
[info] Passed: Total 68, Failed 0, Errors 0, Passed 68
[success] Total time: 107 s, completed 11-Apr-2020 16:44:40


	\end{lstlisting}
	
	\subsection{Проверка требований к программной документации}
	Вся документация, представленная в требованиях, готова.
	
	\addition{Используемые понятия и определения}
	\begin{description}
		\item[\textbf{Web scraping}] -- это сбор данных с различных интернет-ресурсов. Общий принцип его работы можно объяснить следующим образом: некий автоматизированный код выполняет GET-запросы на целевой сайт и получая ответ, парсит HTML-документ, ищет данные и преобразует их в заданный формат. \label{terms:webscraping}
		\item[\textbf{Проект}] -- сущность для объединения и предоставления доступа к запускам/краулерам/периодическим задачам. \label{terms:project}
		
		\item[\textbf{Веб краулер}] --  программа, являющаяся составной частью поисковой системы и предназначенная для перебора страниц Интернета с целью занесения информации о них в базу данных поисковика. Неотъемлемая часть проекта. Именно с помощью пауков пользователь может “краулить” сайты для сбора необходимой информации. \label{terms:spider}
		\item[\textbf{Запуск}] -- единоразовый запуск краулера с настройками и аргументами, указанными для этого запуска. \label{terms:job}
		\item[\textbf{Периодический запуск}] -- запуск с множеством настроек, повторяющийся в определенные периоды времени (запуски по cron-expression).
		\label{terms:pjob}
	\end{description}
	                            \newpage
	%\section{Источники, использованные при разработке}
	\renewcommand{\refname}{Список источников}
	\addcontentsline{toc}{section}{\refname}
	\begin{thebibliography}{7}
		\bibitem{scrapyd} Github scrapyd/scrapyd [Электронный ресурс] URL: \url{https://github.com/scrapy/scrapyd} (Дата обращения: 16.04.2020, режим доступа: свободный)
		\bibitem{gost}Единая система программной документации – М.: ИПК, Издательство стандартов, 2000, 125 стр.
		\bibitem{scalatestplus} ScalaTest+Play [Электронный ресурс] URL:\url{http://www.scalatest.org/plus}(Дата обращения: 16.04.2020, режим доступа: свободный)
		\bibitem{playsilhouettetestkit} Testing - silhouette [Электронный ресурс] URL:\url{https://www.silhouette.rocks/docs/testing} (Дата обращения: 16.04.2020, режим доступа: свободный)
		\bibitem{postgresql} Postgresql [Электронный ресурс] URL:\url{https://www.postgresql.org} (Дата обращения: 16.04.2020, режим доступа: свободный)
	\end{thebibliography}
						\newpage
	\listRegistration
	\addcontentsline{toc}{section}{Лист регистрации изменений}
\end{document} % конец документа