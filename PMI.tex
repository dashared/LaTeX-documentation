\documentclass[a4paper,12pt,reqno]{article}

\usepackage{styledoc19}


\begin{document} % конец преамбулы, начало документа
	
	
	\year{2021}
    \docNumber{RU.17701729.09.09-62 51 01-1}
	\docFormat{Программа и методика испытаний}
	\student{БПИ 174}{Д. Ю. Редникина}
	
	\project{CRM-СИСТЕМА ДЛЯ БЛАГОТВОРИТЕЛЬНОГО ФОНДА <<AIAIN>>. WEB-ПРИЛОЖЕНИЕ ДЛЯ СОТРУДНИКОВ ФОНДА}
	

	\supervisor{Доцент департамента \vfill образовательной программы  \vfill <<Программная инженерия>>}
	{Х. М. Салех}
	
	\firstPage
						\newpage
	\secondPage
						\newpage
	\thirdPage
						\newpage
						
						
	\section{Объекты испытаний}
	\subsection{Наименование программы}
	Cистема управления заданиями по автоматическому сбору данных из сети Интернет
 (System for managing tasks of collecting data from the Internet
)
	\subsection{Область применения программы}
	\subsubsection{Функциональное назначение}
	Система будет применяться как средство управления проектами по созданию, редактированию и запуску веб краулеров для сбора данных в сети интернет. Продукт позволит следить за запусками в режиме реального времени, а также создавать периодические запуски по расписанию.

	\subsubsection{Эксплуатационное назначение}
	Web-приложение является компонентом CRM-системы для благотворительного фонда <<AIAIN>>, позволяющей облегчить бизнес процессы работы фонда с благополучателями и донорами. Web-приложение призвано обеспечить необходимую сотрудникам фонда функциональность для работы с системой. Им будут пользоваться как менеджеры фонда, так и администраторы, члены комиссий, операторы фонда, контент-менеджеры фонда. Каждый из пользователей будет иметь доступ к необходимой ему функциональности по обработке заявок, управлению фондом, администрированию и т.д.
	\subsubsection{Область применения}
	Технологии web-scraping используются как в науке, так и в бизнесе - многие люди чувствуют потребность в извлечении данных из HTML разметки интернет страниц. Существующие аналоги реализуют базовый функционал(сбор данных), но не предоставляют такие дополнительные возможности как периодический запуск или совместное редактирование. Многие аналоги (прим. scrapyd~\cite{scrapyd}) имеют ограниченный функционал.
Главные возможности, которыми продукт обеспечит предполагаемых пользователей:
\begin{itemize}
    \item Совместное управление запусками клаулеров
    \item Периодический запуск задач
    \item Сбор логов, ошибок
    \item Группировка краулеров,а также их запусков в проект 
    \item Бесплатная функциональность
\end{itemize}

					\newpage
					
	\section{Цель испытаний}
	Проверка на соответствие требованиям, указанным в документе <<Cистема управления заданиями по автоматическому сбору данных из сети Интернет
. Техническое задание>>.
	
	\section{Требования к программе}
	\subsection{Требования к функциональным характеристикам}
	Программа выполняется в рамках темы выпускной курсовой работы в соответствии с учебным планом подготовки бакалавров по направлению 09.03.04 «Программная инженерия» Национального исследовательского университета «Высшая школа экономики», факультет компьютерных наук.
	
	Программа должна удовлетворять следующим требованиям:
	
	

\begin{enumerate}
		
		\item \textbf{Авторизация\\}
		Чтобы использовать сервис, клиентская программа должна иметь возможность авторизоваться в системе с помощью REST API
		\begin{enumerate}
			\item Для регистрации пользователю нужно указать следующие данные 
			\begin{enumerate}
			    \item Почта - уникальна для каждого зарегистрированного пользователя; 
			    \item Имя - длина больше 1 символ;
			    \item Логин - длина больше 2 символов;
			    \item Пароль - длина больше 2 символов;
			\end{enumerate}
			\item Для авторизации пользователя в системе должны быть указаны следующие данные
			\begin{enumerate}
			    \item Почта;
			    \item Пароль;
			\end{enumerate}
		\end{enumerate}
		
		\item \textbf{Проекты\\}
		Должны быть реализованы запросы REST API для предоставления клиенту следующей функциональности
		\begin{enumerate}
	    	\item Создание проекта со следующей информацией
	    	\begin{enumerate}
	    	    \item Имя проекта;
	    	    \item Описание проекта - опциональное поле;
	    	\end{enumerate}
			\item Обновление метаданных о проекте (редактирование) могут быть обновлены только участником с минимальным уровнем дотупа \texttt{ReadAndWrite}. Следующие данные могут быть обновлены:
			\begin{enumerate}
			    \item Имя проекта;
			    \item Описание проекта;
			    \item Настройки проекта для запуска краулеров;
			    \item Аргументы для запуска краулеров проекта;
			\end{enumerate}
			\item Обновление \texttt{egg} файла проекта (редактирование) -- минимальный уровень доступа участника, обновляющий данные о проекте \texttt{ReadAndWrite}.  
			\item Удаление данных о проекте. Удалить проект может только владелец \texttt{Owner}.
			\item Просмотр списка проектов (с пагинацией), к которым у пользователя есть как минимум \texttt{ReadOnly} доступ.
		\end{enumerate}
		
		\item \textbf{Участники проектов\\}
		Должны быть реализованы запросы REST API для предоставления клиенту следующей функциональности
		\begin{enumerate}
	    	\item Просмотр информации об участниках проекта;
	    	\begin{enumerate}
	    	    \item Имя, почта, логин участника;
	    	    \item Статус участника в проекте (\texttt{ReadOnly}, \texttt{ReadAndWrite} или \texttt{Owner});
	    	\end{enumerate}
			\item Обновление статуса участника проекта. Это действие совершать может только владелец проекта; 
			\item Удаление участника из проекта. Данное действие может совершать только владелец проекта;
			\item Добавление нового участника с указанными правами на редактирование. Данное действие может совершать только владелец проекта;
		\end{enumerate}
		
		\item \textbf{Краулеры\\}
		Должны быть реализованы запросы REST API для предоставления клиенту следующей функциональности
		\begin{enumerate}
			\item Просмотр списка краулеров проекта;
			\item Редактирование информации о краулере для последующих запусков. Следующая информация может быть изменена
			\begin{enumerate}
			    \item Настройки краулера для запуска;
			    \item Аргументы для запуска;
			\end{enumerate}
		\end{enumerate}
		
		\item \textbf{Запуски краулеров\\}
		Должны быть реализованы запросы REST API для предоставления клиенту следующей функциональности
		\begin{enumerate}
		    \item Просмотр списка запусков в определенном статусе (\texttt{Pending}, \texttt{Running} или \texttt{Finished}) с пагинацией, совершенных в проектах, к которым у пользователя есть как минимум \texttt{ReadOnly} доступ;
		    \item Редактирование запуска - остановка запуска, перевод его в состояние \texttt{Finished}. Операция может быть применена только к запускам в состоянии \texttt{Running} или \texttt{Pending};
		    \item Удаление запуска - удаление всех данных о запуске из базы данных. Операция может быть применена только к запускам в состоянии \texttt{Finished};
		    \item Создание запуска со следующей информацией
		    \begin{enumerate}
		        \item Краулер, с которым происходит запуск;
		        \item Настройки запуска --  это могут быть как и предопределенные настроки на \texttt{scrapyd} \footnote{\url{http://doc.scrapy.org/en/latest/topics/settings.html}}, так и собственные настройки;
		        \item Аргументы запуска -- аргументы для запуска краулера, которые передаются через командную строку;
		        \item Описание запуска;
		    \end{enumerate}
		\end{enumerate}
		
		\item \textbf{Периодические запуски\\}
		Должны быть реализованы запросы REST API для предоставления клиенту следующей функциональности
		\begin{enumerate}
		    \item Просмотр списка периодических запусков с пагинацией;
		    \item Редактирование следующей информации о периодическом запуске
		    \begin{enumerate}
		        \item Настройки будущих запусков --  это могут быть как и предопределенные настроки на \texttt{scrapyd}, так и собственные настройки;
		        \item Аргументы будущих запусков -- аргументы для запуска краулера, которые передаются через командную строку;
		        \item Краулер, с помощью которого будет совершен запуск;
		        \item cron-expression расписания запуска;
		    \end{enumerate}
		    \item Удаление периодического запуска;
		    \item Отмена последующих запусков - перевод периодической задачи в состояние \texttt{Disabled};
		    \item Возобновление запусков - перевод периодической задачи в состояние \texttt{Enabled};
		    \item Создание периодического запуска со следующими данными
		    \begin{enumerate}
		        \item Название;
		        \item Описание -- опциональное;
		        \item Краулер;
		        \item Приоритетность, влияющая на очередь запусков (\texttt{Low}, \texttt{Normal} или \texttt{High});
		        \item Статус (\texttt{Enabled} или \texttt{Disabled});
		        \item Настройки будущих запусков --  это могут быть как и предопределенные настроки на \texttt{scrapyd}, так и собственные настройки;
		        \item Аргументы будущих запусков -- аргументы для запуска краулера, которые передаются через командную строку;
		    \end{enumerate}
		\end{enumerate}
	
	\end{enumerate}

		
	
	
	
	\subsection{Требования к интерфейсу}
	Цветовая гамма интерфейса должена быть выполнена в голубых тонах. Разработанный интерфейс должен соответствовать макетам (см. Приложение \ref{figma}).
	
	\subsection{Требования к надежности}
	
	\newlist{nfr}{enumerate}{10}
 
    \setlist[nfr, 1]{label=\textbf{NFR-\arabic*.}}
    \setlist[nfr, 2]{label*=\textbf{\arabic*.}}
    \setlist[nfr, 3]{label=\arabic*.}
	
	\begin{nfr}
	    \item Система должна сохранять работоспособность и обеспечивать восстановление своих функций при возникновении следующих внештатных ситуаций:
	    \begin{nfr}
	    \item При ошибках работы сервера, к которому по API отправляет запросы Web-приложение, программа должна отображать сообщения в формате уведомлений в UI и продолжать корректную работу;
        \item При ошибках в сбоях аппаратных средств (кроме носителей данных) восстановление работоспособности возлагается на ОС;
        \item При ошибках, связанных с веб-браузером, восстановление работоспособности возлагается на ОС;
	\end{nfr}
	\item Компоненты защиты программы от несанкционированного доступа к данным и функционалу должны обеспечивать:
	\begin{nfr}
	\item Идентификацию пользователя;
	\item Проверку полномочий пользователя при работе с системой;
	\item Разграничение прав доступа пользователей на уровне задач и доступа к данным;
    \end{nfr}
	\end{nfr}
    
    \section{Требования к программной документации}
	\subsection{Состав программной документации}
	Состав программной документации должен включать в себя следующие компоненты:
\begin{enumerate}
	\item Техническое задание <<Cистема управления заданиями по автоматическому сбору данных из сети Интернет
>> (ГОСТ 19.201-78) \label{tz}
	\item Программа и методика испытаний <<Cистема управления заданиями по автоматическому сбору данных из сети Интернет
>> (ГОСТ 19.301-78) \label{pmi}
	\item Пояснительная записка <<Cистема управления заданиями по автоматическому сбору данных из сети Интернет
>> (ГОСТ 19.404-79) \label{pz}
	\item Руководство оператора <<Cистема управления заданиями по автоматическому сбору данных из сети Интернет
>> (ГОСТ 19.505-79) \label{ro}
	\item Текст программы <<Cистема управления заданиями по автоматическому сбору данных из сети Интернет
>> (ГОСТ 19.401-78) \label{tp}
\end{enumerate}

\indent
Вся документация должна быть составлена согласно ЕСПД (ГОСТ 19.101-77, 19.104-78, 19.105-78, 19.106-78 и ГОСТ к соответствующим документам (см. выше)) \cite{gost}. Все документы сдаются в электронном виде в составе курсовой работы LMS НИУ ВШЭ.

Пояснительная записка <<Cистема управления заданиями по автоматическому сбору данных из сети Интернет
>> должна быть проверена на плагиат ($< 40\% $ заимствований). Документ, подтвержадющий проверку Пояснительной записки сдается в печатном виде вместе с подписанным отзывом от научного руководителя.

	
	\newpage
	\section{Средства и порядок испытаний}
	\subsection{Технические средства, используемые во время испытаний}
	\begin{enumerate}
        \item Компьютер оснащенный процессором Intel Core i5 с тактовой частотой 2,3 ГГц;
        \item 16 Гб ОЗУ;
        \item Жесткий диск с объемом свободной памяти более чем 50 ГБ;
        \item Клавиатура и мышь;
        \item Доступ в интернет.
\end{enumerate}
	\subsection{Программные средства, используемые во время испытаний}
	\begin{enumerate}
        \item macOS 10.15.2;
        \item scrapyd~\cite{scrapyd};
        \item Scala 2.12.6;
        \item Play-framework 2.6.13;
        \item PostgreSQL 11~\cite{postgresql};
    \end{enumerate}
	Тестирование производилось с развернутым Web-приложением и API~\cite{api} на стенде, по адресу \url{https://charity.infostrategic.com/}.
	
	
	\subsection{Порядок проведения испытаний}
	В процессе разработки программы были написаны unit-тесты, а также автотесты, тестирующие интеграционное взаимодействие Web-клиента и сервера~\cite{api}.
	
	Unit-тесты были написаны с помощью фреймворка Jest~\cite{jest}. При разработке приложения был использован сервис Gitlab, а также встроенный в Gitlab механизм CI/CD~\cite{cicd}. Разработка была построена следующим образом: на каждом коммите в пулл реквесте запускался пайплайн, который состоял из следующих этапов: Eslint, Prettier, Test. Первые два шага - линтеры, которые осуществляли проверку файлов проекта на код-стайл. На последнем этапе пайплана запускались unit-тесты проекта.
	
	Для автоматизации тестирования UI использовался фреймворк Silenium IDE~\cite{silenium}. С помощью этого фреймворка были написаны интеграционные авто-тесты, покрывающие основные бизнес-требования, отраженные на диаграмме прецедентов в Приложении. \ref{keyUsecase}.
	
	\newpage
	\section{Методы испытаний}
	
	Этап Test пайаплайна CI/CD автоматически запускает тесты проекта с помощью команды: \texttt{yarn test}. На изображении \ref{pic: test} можно увидеть успешный результат прохождения тестов в проекте.
	
	\begin{figure}[H]
		\centering
		\includegraphics[width = \linewidth]{img/test.png}
		\caption{Тестирование. CI job.}
		\label{pic: test}
	\end{figure}
	
	
	\subsection{Проверка требований к функциональным характеристикам}
	
	При тестировании Веб-приложения был использован фреймворк Silenium IDE~\cite{silenium}. С помощью плагина, установленного в браузер Google Chrome~\cite{chrome}, были записаны тесты, проверяющие удовлетворение проекта составленным требованиям.
	
	Для тестирования ключевых бизнес-процессов, выделенных на диаграмме прецедентов цветом (см. Приложение \ref{keyUsecase}), были составлены следующие тест-кейсы. По данным тест-кейсам были созданы тесты Silenium:
	
	\renewcommand{\labelenumi}{\textbf{TC-\arabic{enumi}}.}

\renewcommand{\labelenumii}{\textbf{TC-\arabic{enumi}.\arabic{enumii}}.}


\begin{enumerate}
    \item Авторизация;
    
    \begin{enumerate}
        \item Авторизация зарегистрированного пользователя в системе;
        
        \textbf{Предусловие}: Пользователь зарегистрирован в системе и имеет роль <<Оператор>>, <<Менеджер>>, <<Член комиссии>>, <<Администратор>> или <<Контент-менеджер>>, но не авторизован в системе; 
    
        \textbf{Постусловие}: Пользователь успешно авторизован в системе и имеет доступ к функционалу, соответствующего его роли;
        
        \textbf{Описание}: Пользователь открывает страницу веб-приложения с URL: \url{https://charity.infostrategic.com/login} и вводит логин/пароль, с которым был ранее зарегистрирован;
        
        \textbf{Ожидаемый результат}: Пользователь попадает на страницу с URL: \url{https://charity.infostrategic.com/applications} (если пользователь имеет роль <<Менеджер>>,  <<Член комиссии>>), URL: \url{https://charity.infostrategic.com/chats} (если пользователь имеет роль <<Оператор>>), URL: \url{https://charity.infostrategic.com/fund/description} (если пользователь имеет роль <<Контент-менеджер>>), URL: \url{https://charity.infostrategic.com/users} (если пользователь имеет роль <<Администратор>>); В хэдерах приложения должен быть токен \texttt{Authorization}.
        
        \textbf{Требования, покрываемые тест кейсом}: FR-1;
        
        \textbf{Покрытые юзкейсы}: Sign In;
        
        \item Авторизация незарегистрированного пользователя в системе;
        
        \textbf{Предусловие}: Пользователь не зарегистрирован в системе; 
    
        \textbf{Постусловие}: Пользователь не авторизован в системе;
        
        \textbf{Описание}: Пользователь открывает страницу веб-приложения с URL: \url{https://charity.infostrategic.com/login} и вводит любой логин/пароль;
        
        \textbf{Ожидаемый результат}: Система отображает уведомление с текстом <<Вы не зарегистированы в системе>>;
        
        \textbf{Покрытые требования}: FR-1;
        
        \textbf{Покрытые юзкейсы}: Sign In;
    \end{enumerate}
    
    \item Настройки;
    \begin{enumerate}
        \item Изменение настроек профиля пользователя;
        
        \textbf{Предусловие}: Пользователь авторизован в системе;
        
        \textbf{Постусловие}: Пользователь изменил информацию о своем профиле;
        
        \textbf{Описание}: Авторизованный пользователь переходит на URL: \url{https://charity.infostrategic.com/settings} и меняет: фотографию, ФИО, телефон, адрес, город, страну. После изменения информации пользователь нажимает кнопку <<Сохранить изменения>>.
        
        \textbf{Ожидаемый результат}: Система отображает оповещение <<Все изменения успешно сохранены>>, пользователь видит новые данные в том же разделе.
        
        \textbf{Покрытые требования}: FR-2 (FR-2.1, FR-2.2);
        
        \textbf{Покрытые юзкейсы}: Update and read profile;
        
        \item Изменение языка системы;
        
        \textbf{Предусловие}: Пользователь авторизован в системе;
        
        \textbf{Постусловие}: Пользователь изменил язык системы;
        
        \textbf{Описание}: Авторизованный пользователь переходит на URL: \url{https://charity.infostrategic.com/settings} и меняет язык системы.
        
        \textbf{Ожидаемый результат}: Система отображает оповещение <<Язык системы успешно изменен>>, система меняет надписи на соответствующие выбранному языку.
        
        \textbf{Покрытые требования}: FR-2 (FR-2.3);
        
        \textbf{Покрытые юзкейсы}: Update and read profile;
    \end{enumerate}
    
    \item Управление пользователями;
    \begin{enumerate}
        \item Регистрация пользователя в системе;
        
        \textbf{Предусловие}: Пользователь авторизован в системе под аккаунтом с ролью <<Администратор>>;
        
        \textbf{Постусловие}: Новый пользователь зарегистрирован в системе;
        
        \textbf{Описание}: Авторизованный пользователь переходит на URL: \url{https://charity.infostrategic.com/users/create} и заполняет ФИО будущего пользователя, а также указывает его email и роль в системе.
        
        \textbf{Ожидаемый результат}: Система отображает оповещение <<Пользователь успешно зарегистрирован>> и перебрасывает на URL:\\ \url{https://charity.infostrategic.com/users}, где можно увидеть данные о только что зарегистрированном пользователе.
        
        \textbf{Покрытые требования}: FR-4 (FR-4.3);
        
        \textbf{Покрытые юзкейсы}: CRU of users;
    \end{enumerate}
    
    \begin{enumerate}
        \item Просмотр и изменение профиля пользователя;
        
        \textbf{Предусловие}: Пользователь авторизован в системе под аккаунтом с ролью <<Администратор>>;
        
        \textbf{Постусловие}: Данные о пользователе обновлены;
        
        \textbf{Описание}: Авторизованный пользователь переходит на URL: \url{https://charity.infostrategic.com/users/id}, где id - ID одного из пользователей в системе. Администратор меняет поля ФИО, роль, назначенные категории (если есть);
        
        \textbf{Ожидаемый результат}: Система отображает оповещение <<Информация о пользователе успешно изменена>> и перебрасывает на URL: \url{https://charity.infostrategic.com/users}, где можно увидеть измененные данные о пользователе.
        
        \textbf{Покрытые требования}: FR-4 (FR-4.1, FR-4.2);
        
        \textbf{Покрытые юзкейсы}: CRU of users;
    \end{enumerate}
    
    \item Управление новостным контентом;
    
    \begin{enumerate}
        \item Просмотр и изменение новостей фонда;
        
        \textbf{Предусловие}: Пользователь авторизован в системе под аккаунтом с ролью <<Контент-менеджер>>;
        
        \textbf{Постусловие}: Список новостей обновлен;
        
        \textbf{Описание}: Авторизованный пользователь переходит на URL: \url{https://charity.infostrategic.com/news}, где виден список новостей фонда. Далее пользователь переходит на URL: \url{https://charity.infostrategic.com/news/id} где id - ID одной из новостей в системе. Контент-менеджер меняет картинку, название, описание новости и нажимает кнопку <<Сохранить изменения>>;
        
        \textbf{Ожидаемый результат}: Система отображает оповещение <<Информация о новости успешно изменена>> и перебрасывает на URL: \url{https://charity.infostrategic.com/news}, где можно увидеть измененные новости.
        
        \textbf{Покрытые требования}: FR-10 (FR-10.2);
        
        \textbf{Покрытые юзкейсы}: CRUD news;
    \end{enumerate}
    
    \begin{enumerate}
        \item Удаление новости фонда;
        
        \textbf{Предусловие}: Пользователь авторизован в системе под аккаунтом с ролью <<Контент-менеджер>>;
        
        \textbf{Постусловие}: Выбранная новость удалена;
        
        \textbf{Описание}: Авторизованный пользователь переходит на URL: \url{https://charity.infostrategic.com/news}, где виден список новостей фонда. Из списка новостей пользователь удаляет выбранную.
        
        \textbf{Ожидаемый результат}: Система отображает оповещение <<Новость успешно удалена>> и можно увидеть, что удаленной новости в списке больше нет.
        
        \textbf{Покрытые требования}: FR-10 (FR-10.2);
        
        \textbf{Покрытые юзкейсы}: CRUD news;
    \end{enumerate}
    
    \item Чат поддержки;
    
    \begin{enumerate}
        \item Просмотр диалогов с пользователями;
        
        \textbf{Предусловие}: Пользователь авторизован в системе под аккаунтом с ролью <<Оператор>>;
        
        \textbf{Постусловие}: -
        
        \textbf{Описание}: Авторизованный пользователь переходит на URL: \url{https://charity.infostrategic.com/chats}. В списке можно увидеть сообщения от пользователей, которые обновляются в режиме реального времени. Можно увидеть следующую информацию о чате: ФИО собеседника, текст последнего сообщения;
        
        \textbf{Ожидаемый результат}: Чаты обновляются в режиме реального времени;
        
        \textbf{Покрытые требования}: FR-9 (FR-9.1);
        
        \textbf{Покрытые юзкейсы}: CRU support chats;
    \end{enumerate}
    
    \begin{enumerate}
        \item Написать сообщение пользователю;
        
        \textbf{Предусловие}: Пользователь авторизован в системе под аккаунтом с ролью <<Оператор>>;
        
        \textbf{Постусловие}: Сообщение пользователю отправлено;
        
        \textbf{Описание}: Авторизованный пользователь переходит на URL: \url{https://charity.infostrategic.com/chats}. В списке чатов пользователь выбирает диалог и нажимает на него. В текстовом поле оператор набирает сообщение и отправляет пользователю, нажав кнопку <<Отправить>>; 
        
        \textbf{Ожидаемый результат}: Отправленное сообщение появляется в диалоге с пользователем;
        
        \textbf{Покрытые требования}: FR-9 (FR-9.2);
        
        \textbf{Покрытые юзкейсы}: CRU support chats;
    \end{enumerate}
    
    \item Регистрация и просмотр пожертвовании;
    
    \begin{enumerate}
        \item Регистрация пожертвования фонду;
        
        \textbf{Предусловие}: Пользователь авторизован в системе под аккаунтом с ролью <<Член комиссии>>;
        
        \textbf{Постусловие}: Пожертовование зарегистрировано в системе;
        
        \textbf{Описание}: Авторизованный пользователь переходит на URL: \url{https://charity.infostrategic.com/transactions/create}. После заполнения необходимых полей формы, пользователь нажимает кнопку <<Создать транзакцию>>. 
        
        \textbf{Ожидаемый результат}: Зарегистрированная транзакция появляется в системе;
        
        \textbf{Покрытые требования}: FR-6 (FR-6.1, FR-6.2);
        
        \textbf{Покрытые юзкейсы}: Create donation, Read donation;
    \end{enumerate}
    
    \item Категории заявок;
    
    \begin{enumerate}
        \item Создание, изменение и просмотр категорий;
        
        \textbf{Предусловие}: Пользователь авторизован в системе под аккаунтом с ролью <<Член комиссии>>;
        
        \textbf{Постусловие}: Категории фонда обновлены;
        
        \textbf{Описание}: Авторизованный пользователь переходит на URL: \url{https://charity.infostrategic.com/categories}. В UI отображается информация об уже зарегистрированных категориях: ID, название на английском, название на русском и арабском языках, а также галочка - скрыта ли категория. Пользователь добавляет категории и всю необходимую информацию. Пользователь нажимает кнопку <<Сохранить изменения>>;
        
        \textbf{Ожидаемый результат}: Появляется уведомление <<Категории успешно обновлены>>;
        
        \textbf{Покрытые требования}: FR-7 (FR-7.1, FR-7.2, FR-7.3);
        
        \textbf{Покрытые юзкейсы}: CRUD categories;
    \end{enumerate}
    
    \item Просмотр сотрудников фонда;
    
    \begin{enumerate}
        \item Просмотр сотрудников фонда;
        
        \textbf{Предусловие}: Пользователь авторизован в системе под аккаунтом с ролью <<Член комиссии>>;
        
        \textbf{Постусловие}: -
        
        \textbf{Описание}: Авторизованный пользователь переходит на URL: \url{https://charity.infostrategic.com/managers}. 
        
        \textbf{Ожидаемый результат}: В UI отображается информация о сотрудниках фонда с ролями <<Менеджер>>, <<Член комиссии>>, <<Контент-менеджер>>, <<Оператор>>: ФИО, почта, роль. Перейдя на одну из страниц менеджера должна быть видна следующая информация: ФИО, фотография профиля, роль пользователя, заявки, назначенные на пользователя;
        
        \textbf{Покрытые требования}: FR-4 (FR-4.4);
        
        \textbf{Покрытые юзкейсы}: View managers;
    \end{enumerate}
\end{enumerate}



\renewcommand{\labelenumi}{\arabic{enumi}.}


\renewcommand{\labelenumii}{\arabic{enumii}.}
	
	Результаты UI тестов Silenium можно увидеть на рисунке \ref{pic: silenium}
	
	\begin{figure}[H]
		\centering
		\includegraphics[width = \linewidth]{img/silenium.png}
		\caption{Silenium IDE. Проект с тестами}
		\label{pic: silenium}
	\end{figure}
	
	Также было проведено регрессионное ручное тестирование сложных и многоэтапных бизнес-процессов, включающих активацию заявки и голосование по заявке. Данные сценарии невозможно автоматизировать с помощью Silenium IDE, так как их невозможно завершить (проверить полностью) без использования Android-клиента (или совершения запросов через консоль-клиент). 
	
	Для проверки требований и бизнес-сценариев были написаны следующие тест-кейсы:
	
	\renewcommand{\labelenumi}{\textbf{TC-\arabic{enumi}}.}

\renewcommand{\labelenumii}{\textbf{TC-\arabic{enumi}.\arabic{enumii}}.}


\begin{enumerate}
    \setcounter{enumi}{8}
    \item Обработка заявки;
    
    \begin{enumerate}
        \item Обработка заявки членом комиссии;
        
        \textbf{Предусловие}: Пользователь авторизован в системе под аккаунтом с ролью <<Член комиссии>>; Есть хотя бы одна заявка в статусе <<Новая>>;  
        
        \textbf{Постусловие}: Заявка в статусе <<Активна>>;
        
        \textbf{Описание}: \begin{enumerate}
            \item Член комисии переходит на карточку заявки в статусе <<Новая>>;
            \item Член комисии открывает с помощью кнопки в правом углу <<Действия с заявкой>> панель, в которой должна быть обеспечеа возможность выбора следующего статуса для заявки в соответствии с диаграммой, представленной в Приложении \ref{status}.
            \item Член комисии выбирает статус <<В обработке>> и нажимает кнопку <<Подтвердить>>; 
            \item После закрытия панели статус в карточке заявки должен смениться статус на <<В обработке>> и появиться уведомление об успехе операции;
            \item Член комиссии нажимает кнопку <<Изменить>> и редактирует одобренную сумму для сбора;
            \item Далее нужно повторить шаги 2-4 и выбрать новый статус <<Ждет подтверждения>>;
            \item После того, как все члены комиссии проголосовали за активацию заявки должен быть доступен в панели статус <<Ждет активации пользователя>>;
            \item После активации заявки с мобильного устройства пользователем, создавшим заявку (или через консоль-клиент) статус обрабатываемой заявки становится <<Активна>>;
        \end{enumerate}
        
        \textbf{Ожидаемый результат}: Статус заявки будет <<Активна>>;
        
        \textbf{Покрытые требования}: FR-8 (FR-8.1, FR-8.7);
        
        \textbf{Покрытые юзкейсы}: Approve application;
        
        \item Голосование по заявке;
        
        \textbf{Предусловие}: Пользователь авторизован в системе под аккаунтом с ролью <<Член комиссии>>; Есть хотя бы одна заявка в статусе <<Ждет подтверждения>> с категорией, совпадающей с категорией назначенной на члена комиссии;  
        
        \textbf{Постусловие}: Заявка может быть переведена в статус <<Ждет активации>>;
        
        \textbf{Описание}: \begin{enumerate}
        \item Член комисии переходит на карточку заявки в статусе <<Ждет подтверждения>>;
        \item Член комисии выдвигает правую панель по нажатию на кнопку в правом верхнем углу;
        \item На панели голосования видна статистика голосования за активацию заявки;
        \item Член комисии голосует за активацию, статус голосования становится <<Голосование завершено>>;
        \end{enumerate}
        
        \textbf{Ожидаемый результат}: Статус голосования <<Подтвержден>>;
        
        \textbf{Покрытые требования}: FR-8 (FR-8.7, FR-8.8);
        
        \textbf{Покрытые юзкейсы}: Vote for the application;
        
    \end{enumerate}
\end{enumerate}


\renewcommand{\labelenumi}{\arabic{enumi}.}


\renewcommand{\labelenumii}{\arabic{enumii}.}
	
	Для подтверждения успеха проведенного тестирования в Приложении \ref{test_1} и \ref{test_2} представить фотографии экрана Web-приложения по тест кейсу TC-9 и TC-10.
	
	\subsection{Проверка требований к программной документации}
	Вся документация, представленная в требованиях, готова.
	
	\newpage
	%\section{Источники, использованные при разработке}
	%\renewcommand{\refname}{Список источников}
	% \addcontentsline{toc}{subsection}{\refname}
	\patchcmd{\thebibliography}{\section*{\refname}}{}{}{}
	\anonsection{Список источников}
	\begin{thebibliography}{3}
	    \bibitem{md} Markdown Guide URL: \url{https://www.markdownguide.org} (Дата обращения: 16.04.2021).
	    \bibitem{gost}Единая система программной документации – М.: ИПК, Издательство стандартов, 2000, 125 стр.
		\bibitem{chrome} 
		LMS [Электронный ресурс] URL: 
		\url{https://www.google.com/chrome} (Дата обращения: 31.05.2021, режим доступа: свободный)
		\bibitem{api} Swagger Charity API, v0.2 [Электронный ресурс]  URL:\url{https://app.swaggerhub.com/apis/charity-crm/Charity/0.2} (Дата обращения: 31.05.2021, режим доступа: свободный)
		\bibitem{jest} Jest - testing URL:\url{https://jestjs.io} (Дата обращения: 31.05.2021, режим доступа: свободный)
		
		\bibitem{cicd} Gitlab - CI/CD URL:\url{https://docs.gitlab.com/ee/ci/} (Дата обращения: 31.05.2021, режим доступа: свободный)
		
		\bibitem{silenium} Silenium IDE URL:\url{https://www.selenium.dev/selenium-ide/}(Дата обращения: 31.05.2021, режим доступа: свободный)
	\end{thebibliography}
	
	\newpage
	\addition{Макеты интерфейса}{figma} 
	
	\begin{figure}[H]
		\centering
		\includegraphics[width = 0.9\linewidth]{img/chats.png}
		\caption{Макет списка диалогов}
		\label{pic: applications}
\end{figure}

\begin{figure}[H]
		\centering
		\includegraphics[width = 0.9\linewidth]{img/new.png}
		\caption{Макет списка заявок}
		\label{pic: applications}
\end{figure}

\begin{figure}[H]
		\centering
		\includegraphics[width = 0.9\linewidth]{img/in-processing.png}
		\caption{Макет заявки}
		\label{pic: applications}
\end{figure}
	
	\newpage
	\addition{Ключевые прецеденты}{keyUsecase} 
	\begin{figure}[H]
		\centering
		\includegraphics[width = 0.9\linewidth]{img/key-usecase.pdf}
		\caption{Диаграмма прецедентов}
    \end{figure}
    
    \newpage
	\addition{Тест кейс активации заявки}{test_1}
	
	\begin{figure}[H]
		\centering
		\begin{subfigure}[b]{0.475\linewidth}
			\includegraphics[width=\linewidth]{img/test/1.png}
		\end{subfigure}
		\begin{subfigure}[b]{0.475\linewidth}
			\includegraphics[width=\linewidth]{img/test/2.png}
		\end{subfigure}
		\begin{subfigure}[b]{0.475\linewidth}
			\includegraphics[width=\linewidth]{img/test/3.png}
		\end{subfigure}
		\begin{subfigure}[b]{0.475\linewidth}
			\includegraphics[width=\linewidth]{img/test/4.png}
		\end{subfigure}
		\caption{Скриншоты проведенного тестирорования TC-9}
	\end{figure}
	
		\newpage
	\addition{Тест кейс голосования по заявке}{test_2}
	
	\begin{figure}[H]
		\centering
		\begin{subfigure}[b]{0.475\linewidth}
			\includegraphics[width=\linewidth]{img/test/5.png}
		\end{subfigure}
		\begin{subfigure}[b]{0.475\linewidth}
			\includegraphics[width=\linewidth]{img/test/6.png}
		\end{subfigure}
		\begin{subfigure}[b]{0.475\linewidth}
			\includegraphics[width=\linewidth]{img/test/7.png}
		\end{subfigure}
		\caption{Скриншоты проведенного тестирорования TC-10}
	\end{figure}
	
	\newpage
	\addition{Роли сотрудников фонда}{stuff}
	\begin{description}
		\item[\textbf{Администратор}] -- это сотрудник фонда, который имеет доступ к функционалу по управлению пользователями системы.
		\item[\textbf{Член комиссии}] -- у каждого фонда есть комиссия, которая принимает итоговое решение об активации заявок от нуждающихся. В обязанности членов комиссии также входит обработка заявок, просмотр пожертвований поступающих от доноров, создание категорий для заявок и так далее.
		\item[\textbf{Контент-менеджер}] -- отвечает за управление контентом фонда, а именно: часто задаваемыми вопросами, основной информацией фонда и новостями фонда;
		\item[\textbf{Менеджер}] -- большая часть сотрудников фонда состоит из менеджеров фонда. В их обязанности входит только обработка заявок, коммуникация с пользователями и сбор необходимых документов если требуется;
		\item[\textbf{Оператор}] -- оператор фонда отвечает на вопросы пользователей мобильного приложения;
\end{description}
	
	\newpage
	\addition{Диаграмма жизненного цикла заявки}{status} 
	
	\begin{figure}[H]
		\centering
		\includegraphics[width = 0.9\linewidth]{img/statusflow.pdf}
		\caption{Диаграмма статусов}
		\label{pic: status}
	\end{figure}
	
	
	
	\newpage
	\listRegistration
	
\end{document}