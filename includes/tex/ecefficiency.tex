% Ориентировочная экономическая эффективность
% 
% dyurednikina
% version 1.0
% date 11.11.2018

В данной таблице приведен список прямых и косвенных конкурентов социальной сети.
\begin{longtable}{|r|c|l|} 
\caption{Виды конкурентов} \label{t:competetors0}\\
	  \hline
	\textbf{Название} & \textbf{Описание} & \textbf{Вид} \\ \hline
\textbf{Вк} &Социальная сеть& Прямой \\ \hline
\textbf{Твиттер} &Социальная сеть& Прямой \\ \hline
\textbf{Телеграмм} &Социальная сеть& Прямой \\ \hline
\textbf{Ватсап} &Мессенджер& Прямой \\ \hline
\textbf{Фейсбук} &Социальная сеть& Прямой \\ \hline
\textbf{Лмс} &Система организации обучения& Прямой \\ \hline
\textbf{Официальный сайт} &Информационный портал& Прямой \\ \hline
\textbf{Слак} &Социальная сеть& Прямой \\ \hline
\textbf{Вайбер} &Мессенджер& Прямой \\ \hline
\textbf{LinkedIn} &Социальная сеть& Прямой \\ \hline
\textbf{Одноклассники} &Социальная сеть& Прямой \\ \hline
\textbf{Мероприятия} &Конференции& Косвенный \\ \hline
\textbf{Instagram} &Социальная сеть& Прямой \\ \hline
\textbf{Почта} &Способ рассылки сообщений& Косвенный \\ \hline
\textbf{Brainly} &Социальная сеть& Прямой \\ \hline
\textbf{Skype} &Социальная сеть& Прямой \\ \hline
\textbf{ResearcherGate} &Социальная сеть& Прямой \\ \hline
\textbf{Mendeley} &Социальная сеть& Прямой \\ \hline
\textbf{ieee-collabaratec} &Социальная сеть& Прямой \\ \hline
\textbf{Authorea} &Социальная сеть& Прямой \\ \hline
\textbf{Edmodo} &Социальная сеть& Прямой \\ \hline
\end{longtable}


Большинство аналогов не ориентированы на научную деятельность. 
Прямые конкуренты (например, LMS) не удовлетворяют всем перечисленным критериям (не имеют приложения на мобильные устройства, нет возможности предпросмотра кода, бесперебойность работы). 


Таким образом, приложение <<Социальная сеть для сотрудников НИУ ВШЭ>> будет пользоваться спросом среди сотрудников НИУ ВШЭ из-за его узкой направленности (приложение могут использовать только сотрудники университета), ориентации на развитие научной деятельности, наличии приложения и бесперебойной работы социальной сети.


Ниже приведена таблица с детальным анализом конкурентов социальной сети. При подсчете финальной оценки кажого конкурента большим приоритетом в формуле обладали показатели критериев <<Ориентированность на профессионаоьную и исследовательскую деятельность>>, <<Бесперебойность работы>> и <<Категоризация информации>>. Именно эти критерии наиболее релевантны для будущих пользователей разрабатываемого приложения.

\renewcommand{\labelenumi}{\textbf{\Alph{enumi}}.}

\renewcommand{\labelenumii}{\textbf{\alph{enumi}\arabic{enumii}}}

Для вычисления финальной оценки был применен следующий алгоритм: 
критерии оценки были разбиты на следующие кластеры исходя из смысла, для каждой категории был задан вес(от 0 до 10, где 10 -- самое важное) исходя из общих рассуждений о важности: 
\begin{enumerate}
	
	\item Наличие бесед -- 10
	\item Поиск по категориям -- 9
	\item Категоризация информации -- 9
	\item Автоматический выбор релевантной информации -- 9
	\item Ориентированность на профессиональную деятельность -- 8
	\item Подгрузка информации из релевантных источников -- 3
	\item Ориентированность на мобильные устройства -- 9
	\item Наличие десктопного приложения -- 9
	\item Наличие web-версии -- 9
	\item Код -- 8
	\item Мультиплатформенность -- 9
	\item Мультимедиа -- 6
	\item Разделение сообщения/объявления/профиль -- 9
	\item Бесперебойность работы -- 8
\end{enumerate}
Итоговый коэффициент для каждой оценки рассчитывался, как 
\begin{equation}
\frac{n}{\sum_{i = 1}^7 n} \label{eq}
\end{equation} 

%где n - вес конкретной оценки \eqref{eq}\\
Финальная оценка рассчитывалась по следующей формуле:\\

\code{../includes/code/coeff.py}{Функция, рассчитывающая финальную оценку}


\newpage
%\newgeometry{left=0cm, top=0cm,right=0cm,bottom=0cm}
{

\footnotesize

\begin{longtable}{| >{\raggedright\arraybackslash}p{2.7cm}| >{\centering\arraybackslash}p{0.2cm}| p{0.2cm}| p{0.2cm}| p{0.2cm}| p{0.2cm}| p{0.2cm}| p{0.2cm}| p{0.2cm}| p{0.2cm}| p{0.2cm}| p{0.2cm}| p{0.2cm}| p{0.2cm}| p{0.2cm}| p{0.2cm}| p{0.2cm}| p{0.2cm}| p{0.2cm}| p{0.2cm}| p{0.2cm}|}
	\caption{Детальный анализ конкурентов} \label{t:an} \\
\hline
& \rotatebox{-90}{\textbf{Вк}} & \rotatebox{-90}{\textbf{Твиттер}} & \rotatebox{-90}{\textbf{Телеграмм}} & \rotatebox{-90}{\textbf{Ватсап}} & \rotatebox{-90}
{\textbf{Фейсбук}} & \rotatebox{-90}{\textbf{Лмс}} & \rotatebox{-90}{\textbf{Официальный сайт\ }} & \rotatebox{-90}{\textbf{Слак}} & \rotatebox{-90}{\textbf{Вайбер}} & \rotatebox{-90}{\textbf{Mendeley}} & \rotatebox{-90}{\textbf{Одноклассники}} & \rotatebox{-90}{\textbf{Мероприятия}} & \rotatebox{-90}{\textbf{Instagram}} & \rotatebox{-90}{\textbf{Почта}} & \rotatebox{-90}{\textbf{Brainly}} & \rotatebox{-90}{\textbf{Skype}} &\rotatebox{-90}{\textbf{ResearcherGate}} &\rotatebox{-90}{\textbf{ieee-collabratec}} & \rotatebox{-90}{\textbf{Authorea}}& \rotatebox{-90}{\textbf{Edmodo}}\\ \hline


 \endfirsthead

 \caption*{Продолжение Таблицы \ref{t:an}}  \\

 \hline
 \endhead
 \hline

 \caption*{ Продолжение на след. странице} \
 \endfoot
 \endlastfoot
\textbf{Наличие бесед} & 7 & 3 & 10 & 8 & 8 & 0 & 0 & 10 & 7 & 2 & 7 & 10 & 4 & 0 & 0 & 6 & 0& 6& 0& 8 \\ \hline
\textbf{Поиск по категориям} & 7 & 8 & 4 & 2 & 7 & 0 & 4 & 9 & 0 & 5 & 4 & 3 & 7 & 5 & 9 & 0 & 9& 8& 8& 8 \\ \hline
\textbf{Категоризация информации} & 4 & 5 & 2 & 2 & 4 & 7 & 7 & 8 & 0 & 6 & 4 & 8 & 6 & 5 & 8 & 0 & 9& 7& 8& 7 \\ \hline
\textbf{Ориентирован- ность на профессиональную исследовательскую деятельность} & 2 & 2 & 2 & 2 & 2 & 10 & 10 & 9 & 0 & 10 & 2 & 10 & 1 & 8 & 10 & 4 & 10 & 10& 10& 6\\ \hline
\textbf{Подгрузка информации из других релевантных источников} & 3 & 3 & 3 & 0 & 5 & 0 & 0 & 10 & 0 & 8 & 0 & 0 & 0 & 0 & 0 & 0 & 0& 5& 5 & 8\\ \hline
\textbf{Автомати- ческий выбор релевантной информации} & 6 & 6 & 0 & 0 & 6 & 0 & 0 & 0 & 0 & 7 & 0 & 0 & 6 & 0 & 0 & 0 & 8 & 7& 6& 8\\ \hline
\textbf{Ориентирован- ность на мобильные устройства} & 9 & 7 & 9 & 9 & 8 & 0 & 5 & 8 & 6 & 4 & 5 & 0 & 8 & 9 & 7 & 7 & 4 & 4& 5& 10\\ \hline
\textbf{Наличие десктопного приложения} & 6 & 0 & 8 & 3 & 0 & 0 & 0 & 8 & 3 & 8 & 0 & 0 & 2 & 9 & 0 & 7 & 0 & 0 & 0& 0\\ \hline
\textbf{Наличие web-версии} & 10 & 7 & 8 & 3 & 10 & 6 & 10 & 8 & 4 & 6 & 10 & 4 & 2 & 9 & 10 & 5 & 7& 3& 8& 9 \\ \hline

\textbf{Мультимедиа} & 9 & 6 & 7 & 6 & 8 & 2 & 2 & 6 & 5 & 5 & 8 & 4 & 7 & 6 & 4 & 5 & 7& 9 & 9& 8\\ \hline
\textbf{Код} & 0 & 0 & 4 & 0 & 0 & 0 & 0 & 9 & 0 & 0 & 0 & 0 & 0 & 0 & 3 & 0 & 9 & 10& 9& 6\\ \hline
\textbf{Разделение сообщения/ объявления/ профиль} & 8 & 7 & 4 & 4 & 8 & 7 & 3 & 5 & 4 & 4 & 7 & 7 & 4 & 0 & 2 & 5 & 8 & 7& 6& 10\\ \hline
\textbf{Бесперебой- ность работы} & 9 & 8 & 5 & 10 & 10 & 6 & 10 & 10 & 10 & 9 & 10 & 7 & 10 & 10 & 10 & 8 & 10 & 2& 8& 10\\ \hline
\textbf{Финальная оценка} & \textbf{\scriptsize \kern-0.3em 6.7} & \textbf{\scriptsize \kern-0.3em 5.1} & \textbf{\scriptsize \kern-0.3em 5.9} & \textbf{\scriptsize \kern-0.3em 4.7} & \textbf{\scriptsize \kern-0.3em 6.9} &\textbf{\scriptsize \kern-0.3em 2.7} &\textbf{\scriptsize \kern-0.3em 4.2} & \textbf{\scriptsize \kern-0.3em 8.5} & \textbf{\scriptsize \kern-0.3em 4.1} & \textbf{\scriptsize \kern-0.3em 5.7} & \textbf{\scriptsize \kern-0.3em 5.5} &\textbf{\scriptsize \kern-0.3em 5.2} &\textbf{\scriptsize \kern-0.3em 4.5} & \textbf{\scriptsize \kern-0.3em 4.9} & \textbf{\scriptsize \kern-0.3em 5.1} & \textbf{\scriptsize \kern-0.3em 4.5} & \textbf{\scriptsize \kern-0.3em 5.9} & \textbf{\scriptsize \kern-0.3em 5.9} & \textbf{\scriptsize \kern-0.3em 6}& \textbf{\scriptsize \kern-0.3em 8.2}\\ \hline
\end{longtable}
}
\newpage
%\restoregeometry


\renewcommand{\labelenumi}{\arabic{enumi}.}

\renewcommand{\labelenumii}{\arabic{enumii}.}

\renewcommand{\labelenumiii}{\arabic{enumiii}.}
