\renewcommand{\labelenumi}{\textbf{NFR-\arabic{enumi}}.}

\renewcommand{\labelenumii}{\textbf{NFR-\arabic{enumi}.\arabic{enumii}}.}

\renewcommand{\labelenumiii}{\arabic{enumiii}.}
\begin{enumerate}
	\item \textbf{Интерфейс}
	\begin{enumerate}
		\item Приложение должно иметь оптимальный интерфейс, позволяющий пользователю работать с программой с минимальной предварительной подготовкой. 
		\item Так как приложение разрабатывается только под платформу iOS (нативное), то при разработке интерфейса должны быть использованы iOS Design Themes и Design Principles \cite{interface}. 
		\item При разработке приложения должны будут использоваться основные компоненты UIKit\cite{UIKit}. Этот фрэймворк позволяет добиться согласованного внешнего вида приложения. При разработке будут использовать одни из основных элементов для интерфейса: bars, views, controls. Так же интерфейс должен позволять выполнять основные функции приложения (см.~\ref{subsec:requirements}). 
	\end{enumerate}
Образец первоначального прототипа интерфейса приведен в разделе \ref{interface}.
\end{enumerate}
\renewcommand{\labelenumi}{\arabic{enumi}.}

\renewcommand{\labelenumii}{\arabic{enumii}.}

\renewcommand{\labelenumiii}{\arabic{enumiii}.}




