% требования
% dyurednikina
% version 1.2
% last modified 7.05.2019


Программа выполняется в рамках темы курсовой работы в соответствии с учебным планом подготовки бакалавров по направлению 09.03.04 «Программная инженерия» Национального исследовательского университета «Высшая школа экономики», факультет компьютерных наук.

Разработка требований велась совместно с командой и заказчиком в рамках предмета <<Групповая динамика и коммуникации в программной инженерии>>



\textit{Состав команды}:
\begin{itemize}
	\item Анна Михалева - android-developer;
	\item Константин Манежин - web-developer;
	\item Илья Костюченко - backend-developer;
	\item Дарья Редникина - iOS-developer;
\end{itemize}


\textit{Заказчик}:
\begin{itemize}
	\item Девятьярова Анна Дмитриевна, Факультет бизнеса и менеджмента/ 
	Кафедра управления человеческими ресурсами.
\end{itemize}
% \subsection{Клиентские приложения}
%\renewcommand{\labelenumi}{FR-\arabic{enumi})}
\renewcommand{\labelenumi}{\textbf{FR-\arabic{enumi}}.}

\renewcommand{\labelenumii}{\textbf{FR-\arabic{enumi}.\arabic{enumii}}.}

\renewcommand{\labelenumiii}{\arabic{enumiii}.}

\begin{enumerate}
	\item \textbf{Авторизация клиента\\}
	Чтобы использовать программу, клиент должен иметь возможность авторизоваться в системе.
	\begin{enumerate} \label{req: auth}
		\item При регистрации в социальной сети клиенту необходимо заполнить обязательные поля регистрации: \label{FR-1.2}
		\begin{enumerate}
			\item (\textit{Verified, автор: заказчик}) \\ФИО;
			\item (\textit{Verified, автор: заказчик}) \\Факультет;
			\item (\textit{Verified, автор: заказчик})\\Почта;
			\item (\textit{Verified, автор: заказчик})\\Должность; 
			\item (\textit{Verified, автор: заказчик})\\Город; 
		\end{enumerate}
		\item (\textit{Verified, автор: Илья})\\Уже зарегистрированный клиент для входа в социальную сеть должен ввести свою почту с доменом \verb+@hse.ru+.
		\item (\textit{Verified, автор: Анна})\\
		После успешной авторизациии/регистрации пользователю будет выслан код на введенную им почту, который нужно ввести в специальное поле в приложении, только после правильного ввода кода клиент сможет войти в социальную сеть.
	\end{enumerate}
	\item \textbf{Просмотр профиля пользователя}
	\begin{enumerate}
		\item (\textit{Verified, автор: Дарья})\\Должна быть возможность редактирования у пользователей выше перечисленных полей (см. требование \nameref{FR-1.2}), заполненых при регистрации, в настройках профиля. 
		\item (\textit{Verified, автор: команда, Анна})\\
		На странице профиля должна быть возможность просмотра раннее опубликованных постов человека. 
		\item (\textit{Verified, автор: Константин, заказчик})\\ 
		Также на странице должна быть возможность просмотра личных данных пользователя, то есть информации из требования \nameref{FR-1.2}
		\item (\textit{Verified, автор: Анна})\\
		На странице пользователя должна быть возможность написать сообщение пользователю в диалог.
		\item (\textit{Verified, автор: Анна})\\
		На странице пользователя должна быть возможность добавить пользователя в существующий канал.
	\end{enumerate}
	\item \textbf{Публикация постов\\} 
	Клиент имеет возможность опубликовать текстовую информацию от своего имени, чтобы она отображалась в ленте у других пользователей приложения и у него в профиле.
	\begin{enumerate}
		\item (\textit{Verified, автор: Илья})\\
		При написании поста у пользователя есть возможность добавить хэштеги к текстовой информации.
		\begin{enumerate}
			\item Хэштеги должны состоять из одного слова
			\item Хэштеги должны состоять из латинских и русских букв, допустимые символы при написании хэштега: нижнее подчеркивание, цифры.
			\item При неверном формате введенного хэштега он не будет опубликован вместе с написанным постом.
		\end{enumerate}
		\item (\textit{Verified, автор: Константин})\\
		При выборе хэштегов пользователю должны предлагаться autosuggested hashtags, раннее использованные в приложении другими пользователями при публикации постов.
		\item (\textit{Verified, автор: Анна})\\
		Должна быть поддержка разметки markdown, syntax highlighting при написании поста. 
		\item (\textit{Verified, автор: Дарья})\\
		При выходе из раздела создания поста, должна быть возможность сохранить черновик с текущем текстом и набором хэштегов. 
	\end{enumerate}
	\item \textbf{Лента\\}
	Пользователь должен имеет возможность, находясь в ленте, совершать поиск по интересущим его хэштегам, смотреть новости.
	\begin{enumerate}
		\item (\textit{Verified, автор: Дарья})\\
		При входе в основную ленту должна быть возможность отображения всех существующих постов только в хронологическом порядке.
		\item (\textit{Verified, автор: Константин}) \\ 
		Должна быть возможность осуществления перехода при нажатии на какой-либо хэштег в ленту всех постов с выбранным хэштегом. 
		\item (\textit{Verified, автор: Илья})\\
		Должна быть возможность подсказки autosuggested hashtags при поиске нужной информации в поисковой строке в ленте. 
		\item (\textit{Verified, автор: Илья})\\
		Должна быть возможность отображения как preview поста в ленте, так и его полного содержания в отдельном окне при нажатии. 
		\item (\textit{Verified, автор: Анна})\\
		Должна быть возможность перехода на страницы авторов постов, отображенных в ленте. 
		
	\end{enumerate}
	\item \textbf{Каналы [\ref{term: channel}] \\}
	У клиента приложения должна быть возможность сохранять поисковые фильтры для быстрого доступа к просмотру ленты по нужному множеству хэштегов и людей.
	\begin{enumerate}
		\item (\textit{Verified, автор: Константин}) \\
		Должна быть возможноть создания канала, который должен иметь:
		\begin{enumerate}
			\item Название;
			\item Еединое множество людей и хэштегов; 
			\item Функцию <<предпросмотр канала>>;
		\end{enumerate}
		\item (\textit{Verified, автор: Дарья}) \\
		Должна быть возможноть редактирования каналов, где можно:
		\begin{enumerate}
			\item Изменить название канала;
			\item Изменить множество выбранных хэштегов и людей;  
			\item Перейти к предпросмотру канала;
		\end{enumerate}
		\item (\textit{Verified, автор: Константин}) \\
		Должна быть возможность автоподсказки по хэштегам при редактировании канала; 
		\item (\textit{Verified, автор: заказчик, Анна}) \\
		Просмотр содержимого канала;
		\item (\textit{Verified, автор: Константин}) \\
		Удаление каналов;
		\item (\textit{Verified, автор: Анна})\\ 
		Должна быть возможность поиска по названию созданных каналов;
	\end{enumerate}
	\item \textbf{Создание чатов для пользователей\\}
	Клиент должен иметь возможность общаться с другими пользователями социальной сети: обмениваться сообщениями в диалогах и групповых беседах.
	\begin{enumerate}
		\item (\textit{Verified, автор: Анна, заказчик})\\
		Должна быть возможность создать групповую беседы (добавление участников, названия чата) размером от 2 до 50 пользователей
		\item (\textit{Verified, автор: Илья})\\
		При правах администратора беседы клиент должен иметь возможность:
		\begin{enumerate}
			\item Менять состав участников: добавлять или удалять;
			\item Менять название;
			\item Делать администраторами других людей из беседы;
			\item Лишать их возможности быть администратором; 
		\end{enumerate}
		\item (\textit{Verified, автор: Дарья})\\
		Создатель беседы автоматически должен являться администратором.
		\item (\textit{Verified, автор: Константин})\\
		Должна быть возможность просматривать информацию о беседе (состав участников, название беседы, список администраторов).
		\item (\textit{Verified, автор: Илья})\\
		Должна быть возможность выйти из беседы. 
		\item (\textit{Verified, автор: Дарья})\\
		Должна быть возможность написать сообщение другому пользователю. 
	\end{enumerate}
	\end{enumerate}

\renewcommand{\labelenumi}{\arabic{enumi}.}

\renewcommand{\labelenumii}{\arabic{enumii}.}

\renewcommand{\labelenumiii}{\arabic{enumiii}.}
