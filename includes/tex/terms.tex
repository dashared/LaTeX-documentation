\begin{description}
		\item[Социальная сеть] -- это интернет-площадка, сайт, который позволяет зарегистрированным на нем пользователям размещать информацию о себе и коммуницировать между собой, устанавливая социальные связи
		\item[Хэштег] -- ключевое слово или несколько слов сообщения, тег (пометка), используемый в микроблогах и социальных сетях, облегчающий поиск сообщений по теме или содержанию и начинающийся со знака решётки \label{term: hash}
		\item[Пост] -- информационный блок, размещённый пользователем в социальной сети на своей странице и содержащий набор хэштегов, по которым его можно найти \label{term: post}
		\item[Канал] -- сохраненные ранее созданные фильтры новостей (набор хэштегов) по всем публичным постам \label{term: channel}
		\item [Беседа] -- чат для пользователей, в котором одновременно могут присутствовать от 3 до 50 участников \label{term: chat}
		\item [Проект]  -- раздел, в котором сотрудники с любых факультетов по приглашению смогут вместе работать над каким-либо научным исследованием. Проект состоит из timeline (лента с новостями для всех участников проекта) и набором чатов и бесед (с разным количеством участников в каждой) \label{term: project}
		\item [Preview канала]  -- предпросмотр ленты канала с ограничениями на просмотр полной версии поста в ленте, переходом на страницы авторов, нажатия на хэштеги \label{term: preview}
		\item [Preview поста]  -- ограниченный контент (содержание) поста в ленте \label{term: previewPost}
		\item [Model-View-Controller] MVC схема разделения данных приложения, пользовательского интерфейса и управляющей логики на три отдельных компонента: модель, представление и контроллер — таким образом, что модификация каждого компонента может осуществляться независимо
		\item [Xcode] интегрированная среда разработки (IDE) программного обеспечения для платформ macOS, iOS, watchOS и tvOS, разработанная корпорацией Apple
		\item [Dependency manager] программный модуль, который координируют интеграцию внешних библиотек или пакетов в стек приложения
		\item [Constraints] ограничения на размеры и положения объектов на view,  необходимые для правильного определения размеров и позиций контейнеров
		\item [Storyboard] удобный механизм разработки интерфейса программы
	\end{description}