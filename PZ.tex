\documentclass[a4paper,12pt]{article}
\usepackage{styledoc19}


\begin{document} % конец преамбулы, начало документа
	
	\docNumber{RU.17701729.04.03-01 81 01-1-ЛУ}
	\docFormat{Пояснительная записка}
	\student{БПИ 174}{Д. Ю. Редникина}
	\supervisor{Преподаватель департамента \vfill программной инженерии  факультета компьютерных наук}
	{М. К. Горденко}
	
	\firstPage
						\newpage
	\secondPage
						\newpage
	\thirdPage
						\newpage
	\section{Введение}
	\subsection{Наименование программы}
	\subsection{Документы, на основании которых ведется разработка}
	\newpage
	\section{Назначение и область применения}
	\subsection{Назначение программы }
	\subsubsection{Функциональное назначение }
	\subsubsection{Эксплуатационное назначение}
	\subsubsection{Область применения}
	
					\newpage 
	\section{Технические характеристики}
	\subsection{Постановка задачи на разработку программы}
	\subsection{Описание алгоритмов и функционирования программы}
	\subsubsection{Описание алгоритмов программы}
	\subsubsection{Описание схемы функционирования программы}
	\subsubsection{Возможные взаимодействия программы с другими программами}
	\subsection{Описание и обоснование выбора метода организации входных и выходных данных}
	\subsubsection{Описание метода организации входных и выходных данных}
	\subsubsection{Обоснование выбора метода организации входных и выходных данных}
	\subsection{Описание и обоснование выбора состава технических и программных средств}
	\subsubsection{Состав технических и программных средств}
	\subsubsection{Обоснование выбора состава технических средств}
	
						\newpage
	\section{Технико-экономические показатели}
	\subsection{Предполагаемая потребность}
	\subsubsection{Экономические преимущества по сравнению с отечественными и зарубежными аналогами}

	\addition{Используемые понятия и определения}
	
	\addition{Статус требований}
	
	\addition{Описание классов, структур, методов, полей}
	
						\newpage
	%\section{Источники, использованные при разработке}
	\renewcommand{\refname}{Список источников}
	\addcontentsline{toc}{section}{\refname}
	\begin{thebibliography}{7}
		\bibitem{iOS} The Swift Programming Language Documentation [Электронный ресурс] URL: \url{https://swift.org/documentation/#the-swift-programming-language} (Дата обращения: 16.05.2019, режим доступа: свободный)
		\bibitem{documentation}Единая система программной документации – М.: ИПК, Издательство стандартов, 2000, 125 стр.
	\end{thebibliography}
	
						\newpage
	\listRegistration
	\addcontentsline{toc}{section}{Лист регистрации изменений}
\end{document} % конец документа