\documentclass[a4paper,12pt]{article}
\usepackage{styledoc19}


\begin{document} % конец преамбулы, начало документа
	
	\itsHSE
	\academicTeacher{Доцент департамента больших данных и информационного поиска, к.ф.-м.н.}
	{В. Л. Чернышев}
	
	\projectName{РАЗРАБОТКА ПРОГРАММНОГО ПРОДУКТА ДЛЯ ПОСТРОЕНИЯ И ИЗУЧЕНИЯ МНОГОЧЛЕНОВ, СООТВЕТСТВУЮЩИХ ГЕОМЕТРИЧЕСКИМ ГРАФАМ}
	
	\titleList{Техническое задание}{RU.17701729.04.15-01 51 01-1-ЛУ}
	\par\vspace{60mm}
	\nameOfAuthor{БПИ174}{Д. Ю. Редникина}
	\tabForFirstPage
	
						\newpage
	
	\pagestyle{fancy}
	\lhead{УТВЕРЖДЕН \newline
	 	RU.17701729.04.15-01 34 01-1-ЛУ}
	\vspace*{\fill}
	\begingroup
	\centering
	\tabForFirstPage
	\projectName{РАЗРАБОТКА ПРОГРАММНОГО ПРОДУКТА ДЛЯ ПОСТРОЕНИЯ И ИЗУЧЕНИЯ МНОГОЧЛЕНОВ, СООТВЕТСТВУЮЩИХ ГЕОМЕТРИЧЕСКИМ ГРАФАМ}
	\titleList{Техническое задание}{RU.17701729.04.15-01 51 01-1-ЛУ}
	\listNumber{2}
	
	\endgroup
	\vspace*{\fill}
	
	
						\newpage
	\lhead{ }
	\chead{\vfill \thepage \vfill  RU.17701729.04.15-01 ТЗ 01-1 }
	\rhead{ }
	\cfoot{ }
	%delete this if you are not writing a TZ
	\cfoot{\tabForTZ}
	\tableofcontents
	\newpage
	\section{Введение}
	\subsection{Наименование программы}
	\subsection{Краткая характеристика области применения}
	\newpage
	\section{Основания для разработки}
	\subsection{Документы, на основании которых ведется разработка}
	
	
	Приказ декана факультета компьютерных наук Национального Исследовательского университета <<Высшая школа экономики>>...
	
	\subsection{Наименование темы разработки}
	Программа выполняется в рамках темы курсовой работы в соответствии с учебным планом подготовки бакалавров по направлению 09.03.04 «Программная инженерия» Национального исследовательского университета «Высшая школа экономики», факультет компьютерных наук, департамент программной инженерии.
	
	\newpage 
	\section{Назначение разработки}
	\subsection{Функциональное назначение}
	\subsection{Эксплуатационное назначение}
	
						\newpage
	\section{Требования к программе}
	\subsection{Требования к функциональным характеристикам}
	\subsubsection{Требования к составу выполняемых функций}
	\subsubsection{Требования к организации входных данных}
	\subsubsection{Требования к организации выходных данных}
	\subsection{Требования к надежности}
	\subsection{Требования к интерфейсу}
	\subsection{Условия эксплуатации}
	\subsection{Требования к составу и параметру технических средств}
	\subsection{Требования к информационной и программной совместимости}
	\subsection{Требования к маркировке и упаковке}
	\subsection{Требования к транспортированию и хранению}
	
						\newpage
	\section{Требования к программной документации}
	Состав программной документации должен включать в себя следующие компоненты:
	\begin{enumerate}
		\item Техническое задание (ГОСТ 19.201-78)
		\item Программа и методика испытаний (ГОСТ 19.301-78)
		\item Пояснительная записка (ГОСТ 19.404-79)
		\item Руководство оператора (ГОСТ 19.505-79)
		\item Текст программы (ГОСТ 19.401-78)
	\end{enumerate}
	Вся документация должна быть составлена согласно ЕСПД (ГОСТ 19.101-77, 19.104-78, 19.105-78, 19.106-78 и ГОСТ к соответствующим документам (см. выше)) [3]. Вся документация сдаётся в печатном виде, с подписанными листами утверждения и в электронном виде в составе курсовой работы в систему LMS НИУ ВШЭ. 
	
						\newpage
	\section{Технико-экономические показатели}
	\subsection{Предполагаемая потребность}
	\subsection{Ориентировочная экономическая эффективность}
	\subsection{Экономические преимущества разработки по сравнению с отечественными и зарубежными аналогами}
	
						\newpage
	\section{Стадии и этапы разработки}
	\subsection{Необходимые стадии разработки, этапы и содержание работ}
	\subsection{Сроки и исполнители}
	Программа и документация к ней разрабатываются к утвержденным срокам защиты курсовой работы (20 – 30 мая 2019 года).
	Исполнителем является студент НИУ ВШЭ группы БПИ174 Редникина Дарья Юрьевна.
	
						\newpage
	\section{Порядок контроля и приемки}
	\subsection{Виды испытаний}
	\subsection{Общие требования к приемке работы}
	
						\newpage
	\listRegistration
	\addcontentsline{toc}{section}{Лист регистрации изменений}

\end{document} % конец документа