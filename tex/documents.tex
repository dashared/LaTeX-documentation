Состав программной документации должен включать в себя следующие компоненты:
\begin{enumerate}
	\item Техническое задание <<CRM-система для благотворительного фонда <<AIAIN>>. Web-приложение для сотрудников фонда>> (ГОСТ 19.201-78) \label{tz}
	\item Программа и методика испытаний <<CRM-система для благотворительного фонда <<AIAIN>>. Web-приложение для сотрудников фонда>> (ГОСТ 19.301-78) \label{pmi}
	\item Руководство оператора <<CRM-система для благотворительного фонда <<AIAIN>>. Web-приложение для сотрудников фонда>> (ГОСТ 19.505-79) \label{ro}
	\item Текст программы <<CRM-система для благотворительного фонда <<AIAIN>>. Web-приложение для сотрудников фонда>> (ГОСТ 19.401-78) \label{tp}
\end{enumerate}

\indent
Вся документация должна быть составлена согласно ЕСПД (ГОСТ 19.101-77, 19.104-78, 19.105-78, 19.106-78 и ГОСТ к соответствующим документам (см. выше)) \cite{gost}. Все документы сдаются в электронном виде в составе выпускной квалификационной работы LMS НИУ ВШЭ.

% Пояснительная записка <<CRM-система для благотворительного фонда <<AIAIN>>. Web-приложение для сотрудников фонда>> должна быть проверена на плагиат ($< 40\% $ заимствований). Документ, подтвержадющий проверку Пояснительной записки сдается в печатном виде вместе с подписанным отзывом от научного руководителя.
