\begin{center}
\begin{longtable}{|p{0.15\linewidth}|p{0.75\linewidth}|}
\caption{Акторы диаграммы прецедентов} \label{table:actors} \\

\hline \multicolumn{1}{|c|}{\textbf{Актор}} & \multicolumn{1}{c|}{\textbf{Описание}} \\ \hline
\endfirsthead

\multicolumn{2}{r}%
{{\bfseries \tablename\ \thetable{} -- продолжение таблицы}} \\
\hline \multicolumn{1}{|c|}{\textbf{Актор}} & \multicolumn{1}{c|}{\textbf{Описание}} \\ \hline 
\endhead

\hline \multicolumn{2}{r}{{Продолжение на следующей странице}} \\ 
\endfoot

\hline \hline
\endlastfoot

General user & Обобщение для всех пользователей CRM. \\ \hline
Admin & \textbf{Администратор} управляет всеми учетными записями в системе. У него есть доступ к логам системы, а также возможность просматривать пользователей системы и регистрировать их. \\ \hline
Operator & \textbf{Оператор} отвечает на сообщения пользователей, отправленные через чат. \\ \hline
Content Manager & 
\textbf{Менеджер контента} системы имеет возможность редактировать описание фонда, а также раздел часто задаваемых вопросов. \\ \hline
Manager & \textbf{Менеджер фонда} имеет доступ к заявкам фонда, а также может обрабатывать данные заявки: менять статусы, редактировать и общаться с пользователем в рамках одной заявки. Также менеджер может просматривать транзакции фонда. \\ \hline
SuperManager & \textbf{Комиссия фонда} имеет доступ к расширенному функционалу управления заявками, а именно к активации заявки (активная заявка становится видимой для всех пользователей системы, на нее можно жертвовать деньги). \\ \hline
Firebase & Система рассылки уведомлений. \\ \hline





\end{longtable}
\end{center}