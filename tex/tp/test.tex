\subsection{controllers/ApplicationSpec.scala}
\begin{lstlisting}
package controllers


import utils.{AuthSpecification,DatabaseCleaner}
import play.api.test._
import com.mohiva.play.silhouette.test._
import play.api.test.Helpers._



class ApplicationSpec extends AuthSpecification
                      with DatabaseCleaner {


  import utils.data.UserData._

  private val controller = app.injector.instanceOf[ApplicationController]

  "ApplicationController Logout GET" should {

    "should be unauthorized error" in {
      val result = controller.logout.apply(FakeRequest(method = "GET", "/"))
      status(result) mustBe UNAUTHORIZED
    }

    "redicrect if user was found" in {
      val request = FakeRequest().withAuthenticator(loginInfo)
      val result = controller.logout(request)

      status(result) mustBe SEE_OTHER
      redirectLocation(result) mustBe Some("/api/helloworld")
    }
  }
}
\end{lstlisting}
\subsection{controllers/AuthorizationSpec.scala}
\begin{lstlisting}
package controllers

import akka.stream.Materializer
import com.mohiva.play.silhouette.api.LoginInfo
import forms.SignIn
import play.api.libs.json.{JsString, Json}
import play.api.test.FakeRequest
import play.api.test.Helpers._
import utils.{AuthSpecification, DatabaseCleaner}

class AuthorizationSpec extends AuthSpecification
                        with DatabaseCleaner {

  import utils.data.UserData._

  private val controllerSignUp = app.injector.instanceOf[SignUpController]
  private val controllerSignIn = app.injector.instanceOf[SignInController]
  implicit lazy val materializer: Materializer = app.materializer

  "Authorize person" should {

    "signUp: OK" in {

      val result = controllerSignUp.signUp()(FakeRequest().withBody(signUpForm))

      status(result) mustBe OK
      contentAsJson(result) mustEqual Json.toJson(Map("user" -> LoginInfo("credentials", authorizationEmail)))

      val resultSignIn = controllerSignIn.signIn()(FakeRequest().withBody(credentials))
      status(resultSignIn) mustBe OK
    }

    "signUp: userAlreadyExists" in {

      val result = controllerSignUp.signUp()(FakeRequest().withBody(signUpForm.copy(email = email)))

      status(result) mustBe CONFLICT
      contentAsJson(result) mustBe JsString(controllerSignUp.UserAlreadyExistsMessage)
    }

    "signUp: invalid email format" in {
      val invalidCredentials = signUpForm.copy(email = "wrongformat@ j j j")
      val result = call(controllerSignUp.signUp(), FakeRequest(POST, "/api/auth/signup").withJsonBody(Json.toJson(invalidCredentials)))
      status(result) mustBe FORBIDDEN
    }

    "signIn: wrong credentials" in {

      val wrongCredentials = Json.toJson(SignIn("ddd@kkk.d", "fff"))
      val result = call(controllerSignIn.signIn(), FakeRequest(POST, "/api/auth/signin").withJsonBody(wrongCredentials))

      status(result) mustBe FORBIDDEN
    }
  }
}
\end{lstlisting}
\subsection{controllers/CrawlerSpec.scala}
\begin{lstlisting}
package controllers

import models.tables.Crawler
import play.api.libs.json.{Json, OFormat}
import play.api.test.FakeRequest
import play.api.test.Helpers.{contentAsJson, status}
import utils.{AuthSpecification, DatabaseCleaner, TestHelper}
import play.api.test.Helpers._
import utils.data.UserData._
import com.mohiva.play.silhouette.test._
import forms.SpiderChangeForm
import models.services.SecurityService
import utils.data.ProjectTestData

class CrawlerSpec extends AuthSpecification
                  with DatabaseCleaner {

  implicit val crawlerFormat: OFormat[Crawler] = Json.format[Crawler]

  private val projectsController = app.injector.instanceOf[ProjectsController]
  private val crawlersController = app.injector.instanceOf[CrawlersController]

  private val readonlyProject = ProjectTestData.Access.readOnly
  private val readAndWriteProject = ProjectTestData.Access.readAndWrite
  private val noAccessProject = ProjectTestData.Access.noAccess

  private def getPutRequest: FakeRequest[SpiderChangeForm] = {

    val changeBody = SpiderChangeForm(settings = Map("DOWNLOAD_DELAY" -> "300"))
    FakeRequest().withAuthenticator(loginInfo).withBody(changeBody)
  }

  "GET crawlers: OK" in {

    val projectId = 15
    val deployResult = projectsController.deployProject(projectId)(TestHelper.requestWithMetadata("egg1.egg"))

    status(deployResult) mustBe OK
    contentAsJson(deployResult) mustBe Json.toJson(TestHelper.crawlers)

    val getResult = crawlersController.listSpiders(projectId)(FakeRequest().withAuthenticator(loginInfo))

    status(getResult) mustBe OK
    contentAsJson(deployResult) mustBe Json.toJson(TestHelper.crawlers)
  }

  "GET crawlers: no access" in {

    val getResult = crawlersController.listSpiders(noAccessProject)(FakeRequest().withAuthenticator(loginInfo))

    status(getResult) mustBe FORBIDDEN
  }

  "PUT crawlers" in {

    val putResult = crawlersController.updateSpider(readAndWriteProject, 2)(getPutRequest)

    status(putResult) mustBe OK
  }

  "PUT crawlers: ReadOnly access" in {

    val putResult = crawlersController.updateSpider(readonlyProject, 1)(getPutRequest)

    status(putResult) mustBe FORBIDDEN
    contentAsString(putResult) mustBe SecurityService.UserAccessMessage
  }

  "PUT crawlers: spider not found" in {

    val putResult = crawlersController.updateSpider(readAndWriteProject, 40)(getPutRequest)

    status(putResult) mustBe FORBIDDEN
    contentAsString(putResult) mustBe SecurityService.CrawlerMessage
  }
}
\end{lstlisting}
\subsection{controllers/jobs/JobTestCase.scala}
\begin{lstlisting}
package controllers.jobs

import controllers.{JobsController, PeriodicJobsController, ProjectsController}
import models.tables.Crawler
import play.api.libs.json.{Json, OFormat}
import play.api.test.Helpers.{status, _}
import utils.{AuthSpecification, DatabaseCleaner, TestHelper}

class JobTestCase extends AuthSpecification
                  with DatabaseCleaner {

  // MARK: - Controllers

  val periodicJobsController: PeriodicJobsController = app.injector.instanceOf[PeriodicJobsController]
  val projectController: ProjectsController = app.injector.instanceOf[ProjectsController]
  val jobsController: JobsController = app.injector.instanceOf[JobsController]

  implicit val crawlerFormat: OFormat[Crawler] = Json.format[Crawler]

  def postEggFile(projectId: Long): Unit = {
    val deployResult = projectController.deployProject(projectId)(TestHelper.requestWithMetadata("egg1.egg"))

    status(deployResult) mustBe OK
  }

}
\end{lstlisting}
\subsection{controllers/jobs/specs/JobExecutionSpec.scala}
\begin{lstlisting}
package controllers.jobs.specs

import java.util.UUID

import com.mohiva.play.silhouette.test._
import controllers.jobs.JobTestCase
import models.common.enums.JobExecutionStatus
import models.responses.SimpleJob
import models.services.SecurityService
import play.api.libs.json._
import play.api.test.FakeRequest
import play.api.test.Helpers._
import utils.data.UserData._
import utils.TestHelper
import utils.data.{JobTestData, ProjectTestData}

class JobExecutionSpec extends JobTestCase {

  private val projectId: Long = JobTestData.OnetimeJob.Owner.projectId
  private val jobId: Long = JobTestData.OnetimeJob.ids.length
  private val scheduledJobId: Long = JobTestData.OnetimeJob.ids.length + 1

  private val request = FakeRequest().withAuthenticator(loginInfo)

  private val readAndWriteProject = ProjectTestData.Access.readAndWrite
  private val readonlyProject = ProjectTestData.Access.readOnly

  private def scheduleJob(projectId: Long, scheduledJobId: Long): UUID = {
    /** Post egg file with crawlers for project */
    postEggFile(projectId)

    /** Schedule job */
    val request = FakeRequest().withAuthenticator(loginInfo).withBody(TestHelper.onetimeJobForm)
    val result = jobsController.schedule(projectId)(request)

    val responseObject = contentAsJson(result).as[SimpleJob]

    responseObject.id mustBe scheduledJobId
    status(result) mustBe OK

    responseObject.scrapydId
  }

  "JobsController" should {

    "GET jobs" in {

      val limit = 10
      val result = jobsController.getJobsExecutions(limit, JobExecutionStatus.Finished)(request)

      status(result) mustBe OK
      contentAsJson(result) match {
        case JsArray(value) =>
          value must have length limit
          (value(0) \ "id").get must equal(JsNumber(20))
        case _ => assertTypeError("Error type")
      }

      val resultRunning = jobsController.getJobsExecutions(limit, JobExecutionStatus.Running)(request)

      status(resultRunning) mustBe OK
      contentAsJson(resultRunning) match {
        case JsArray(value) => value must have length 0
        case _ => assertTypeError("Error type")
      }
    }

    "GET jobs: pagination" in {

      val limit = 10
      val result = jobsController.getJobsExecutions(limit, JobExecutionStatus.Finished, Some(3))(request)

      status(result) mustBe OK
      status(result) mustBe OK
      contentAsJson(result) match {
        case JsArray(value) =>
          value must have length 2
          (value(0) \ "id").get must equal(JsNumber(2))
        case _ => assertTypeError("Error type")
      }
    }

    "POST schedule: ordinary" in {

      scheduleJob(projectId, scheduledJobId)

      /** Get 1 running/pending job execution */
      val getResult = jobsController.getJobsExecutions(10, JobExecutionStatus.Pending)(request)

      status(getResult) mustBe OK
      contentAsJson(getResult) match {
        case JsArray(value) =>
          value must have length 1
          (value(0) \ "id").get must equal(JsNumber(scheduledJobId))
          (value(0) \ "status").get must equal(JsString("Pending"))
        case _ => assertTypeError("Error type")
      }
    }


    "POST schedule: ReadOnly access" in {

      /** Schedule job with ReadOnly access */
      val request = FakeRequest().withAuthenticator(loginInfo).withBody(TestHelper.onetimeJobForm)
      val result = jobsController.schedule(readonlyProject)(request)

      status(result) mustBe FORBIDDEN
      contentAsString(result) mustBe SecurityService.UserAccessMessage
    }

    "POST schedule: ProjectId doesn't match CrawlerId" in {

      val request = FakeRequest().withAuthenticator(loginInfo).withBody(TestHelper.onetimeJobForm)
      val result = jobsController.schedule(readAndWriteProject)(request)

      status(result) mustBe FORBIDDEN
      contentAsString(result) mustBe SecurityService.CrawlerMessage
    }

    "PUT cancel" in {

      val jobScrapydId = scheduleJob(projectId, scheduledJobId)

      // for job to start running/or be still pending
      Thread.sleep(5000)

      val cancelJobResponse = jobsController.cancel(projectId, jobScrapydId, scheduledJobId)(request)

      status(cancelJobResponse) mustBe OK

      // for scrapyd to finish task
      Thread.sleep(1000)

      val getJobResponse = jobsController.getJobsExecutions(1, JobExecutionStatus.Finished)(request)

      status(getJobResponse) mustBe OK
      contentAsJson(getJobResponse) match {
        case JsArray(value) =>
          value must have length 1
          (value(0) \ "id").get must equal(JsNumber(scheduledJobId))
          (value(0) \ "status").get must equal(JsString("Finished"))
        case _ => assertTypeError("Error type")
      }
    }

    "PUT cancel: jobId doesn't match to projectId" in {

      val jobScrapydId = UUID.randomUUID()

      val cancelJobResponse = jobsController.cancel(projectId, jobScrapydId, jobId)(request)

      status(cancelJobResponse) mustBe FORBIDDEN
      contentAsString(cancelJobResponse) mustBe SecurityService.JobExecutionToProjectMessage
    }

    "PUT cancel: ReadOnly access" in {

      val jobScrapydId = UUID.randomUUID()

      val cancelJobResponse = jobsController.cancel(readonlyProject, jobScrapydId, jobId)(request)

      status(cancelJobResponse) mustBe FORBIDDEN
      contentAsString(cancelJobResponse) mustBe SecurityService.UserAccessMessage
    }

    "DELETE job" in {

      val (projectId, lastJobId, jobScrapydId) = TestHelper.insertedJobExecutions.last

      lastJobId mustBe jobId

      // delete job
      val deleteJobResponse = jobsController.deleteJob(projectId, jobScrapydId, lastJobId)(request)

      status(deleteJobResponse) mustBe OK

      // check if job was completely deleted
      val getJobResponse = jobsController.getJobsExecutions(1, JobExecutionStatus.Finished)(request)

      status(getJobResponse) mustBe OK
      contentAsJson(getJobResponse) match {
        case JsArray(value) =>
          value must have length 1
          (value(0) \ "id").get must equal(JsNumber(lastJobId - 1))
          (value(0) \ "status").get must equal(JsString("Finished"))
        case _ => assertTypeError("Error type")
      }
    }

    "DELETE job: still running" in {

      val jobScrapydId = scheduleJob(projectId, scheduledJobId)

      val deleteJobResponse = jobsController.deleteJob(projectId, jobScrapydId, scheduledJobId)(request)

      status(deleteJobResponse) mustBe UNPROCESSABLE_ENTITY
      contentAsString(deleteJobResponse) mustBe "There was no finished job with given id"
    }

    "DELETE job: ReadOnly access" in {

      val (_, jobId, jobScrapydId) = TestHelper.insertedJobExecutions.last
      val cancelJobResponse = jobsController.cancel(readonlyProject, jobScrapydId, jobId)(request)

      status(cancelJobResponse) mustBe FORBIDDEN
      contentAsString(cancelJobResponse) mustBe SecurityService.UserAccessMessage
    }

    "DELETE job: jobId doesn't match projectId" in {

      val (projectId, jobId, _) = TestHelper.insertedJobExecutions.last
      val cancelJobResponse = jobsController.cancel(projectId, UUID.randomUUID(), jobId)(request)

      status(cancelJobResponse) mustBe FORBIDDEN
      contentAsString(cancelJobResponse) mustBe SecurityService.JobExecutionToProjectMessage
    }
  }

}
\end{lstlisting}
\subsection{controllers/jobs/specs/PeriodicJobSpec.scala}
\begin{lstlisting}
package controllers.jobs.specs


import com.mohiva.play.silhouette.test._
import controllers.jobs.JobTestCase
import models.common.enums.{JobExecutionStatus, RunningStatus}
import models.responses.ProjectData
import models.services.SecurityService
import models.tables.JobInstance
import play.api.libs.json.JsArray
import play.api.test.FakeRequest
import play.api.test.Helpers.{status, _}
import utils.data.UserData._
import utils.TestHelper
import utils.data.{JobTestData, ProjectTestData}

class PeriodicJobSpec extends JobTestCase {

  private val projectId: Long = JobTestData.PeriodicJob.Owner.projectId
  private val crawlerId: Long = JobTestData.PeriodicJob.Owner.crawlerId
  private val periodicJobId: Long = JobTestData.PeriodicJob.Status.disabledId
  private val scheduledJobId: Long = JobTestData.PeriodicJob.idsDisabled.max + 1

  private val readonlyProject: Long = ProjectTestData.Access.readOnly
  private val request = FakeRequest().withAuthenticator(loginInfo)

  private def addActivePeriodicJob(projectId: Long, periodicJobId: Long): Unit  = {

    postEggFile(projectId)

    /** Schedule job */
    val request = FakeRequest().withAuthenticator(loginInfo).withBody(TestHelper.periodicJobForm)
    val result = periodicJobsController.addPeriodicJob(projectId)(request)

    val responseObject = contentAsJson(result).as[Long]

    responseObject mustBe periodicJobId
    status(result) mustBe OK
  }

  private def cancel(projectId: Long, jobId: Long) = {

    val cancelJobResponse = periodicJobsController.disable(projectId, jobId)(request)
    status(cancelJobResponse) mustBe OK
  }

  "PeriodicJobController" should {

    "GET periodic jobs: basic" in {

      val getResult = periodicJobsController.getPeriodicJobs(projectId, 1)(request)

      status(getResult) mustBe OK
      contentAsJson(getResult).as[Seq[JobInstance]] must equal (Seq(TestHelper.insertedPeriodicJobs(periodicJobId)))
    }

    "GET periodic jobs: pagination" in {

      val getResult = periodicJobsController.getPeriodicJobs(projectId, 5, exclusiveFrom = Some(23))(request)

      status(getResult) mustBe OK
      contentAsJson(getResult).as[Seq[JobInstance]] must equal (Seq(TestHelper.insertedPeriodicJobs(22), TestHelper.insertedPeriodicJobs(21)))
    }

    "GET periodic jobs: no access to project" in {

      val noAccessId = ProjectTestData.Access.noAccess

      val getResult = periodicJobsController.getPeriodicJobs(noAccessId, 5)(request)

      status(getResult) mustBe FORBIDDEN
      contentAsString(getResult) mustBe SecurityService.UserAccessMessage
    }

    "POST periodic job: basic" in {

      val projectId = 12

      addActivePeriodicJob(projectId, scheduledJobId)

      Thread.sleep(5000)

      val getResponse = jobsController.getJobsExecutions(1, JobExecutionStatus.Pending)(request)

      contentAsJson(getResponse) match {
        case JsArray(value) =>
          (value(0) \ "jobInstanceId").as[Long] mustBe scheduledJobId
          (value(0) \ "project").as[ProjectData] mustBe ProjectData(projectId, s"${projectId} project")
        case _ => assertTypeError("Error type")
      }

      val cancelResponse = periodicJobsController.disable(projectId, scheduledJobId)(request)

      status(cancelResponse) mustBe OK
    }

    "POST periodic job: invalid cron-expression" in {

      val invalidCronBody = TestHelper.periodicJobForm.copy(cronExpression = "test", crawlerId = crawlerId)
      val createResponse = periodicJobsController.addPeriodicJob(projectId)(request.withBody(invalidCronBody))

      status(createResponse) mustBe UNPROCESSABLE_ENTITY
      contentAsString(createResponse) mustBe "invalid cron expression"
    }

    "POST periodic job: crawler doesn't correspond to project" in {

      val createResponse = periodicJobsController.addPeriodicJob(projectId)(request.withBody(TestHelper.periodicJobForm))

      status(createResponse) mustBe FORBIDDEN
      contentAsString(createResponse) mustBe SecurityService.CrawlerMessage
    }

    "POST periodic job: ReadOnly access" in {

      val createResponse = periodicJobsController.addPeriodicJob(readonlyProject)(request.withBody(TestHelper.periodicJobForm))

      status(createResponse) mustBe FORBIDDEN
      contentAsString(createResponse) mustBe SecurityService.UserAccessMessage
    }

    "PUT periodic job: basic" in {

      val jobInstance = TestHelper.insertedPeriodicJobs(periodicJobId)
      val projectId = jobInstance.projectId
      val changePeriodicJob = TestHelper.changePeriodicJobForm.copy(crawlerId = jobInstance.spider)

      val putResponse = periodicJobsController.changePeriodicJob(projectId, periodicJobId)(request.withBody(changePeriodicJob))

      status(putResponse) mustBe OK
    }

    "PUT periodic job: crawler does not correspond to project" in {

      val changePeriodicJob = TestHelper.changePeriodicJobForm

      val putResponse = periodicJobsController.changePeriodicJob(projectId, periodicJobId)(request.withBody(changePeriodicJob))

      status(putResponse) mustBe FORBIDDEN
      contentAsString(putResponse) mustBe SecurityService.CrawlerMessage
    }

    "PUT periodic job: wrong cron-expression" in {

      val changePeriodicJob = TestHelper.changePeriodicJobForm.copy(cronExpression = "test", crawlerId = crawlerId)

      val putResponse = periodicJobsController.changePeriodicJob(projectId, periodicJobId)(request.withBody(changePeriodicJob))

      status(putResponse) mustBe UNPROCESSABLE_ENTITY
      contentAsString(putResponse) mustBe "invalid cron expression"
    }

    "DELETE periodic job: enabled status" in {

      addActivePeriodicJob(projectId, scheduledJobId)

      // check GET after creating periodic job and before deleting it
      val getBeforeResponse = periodicJobsController.getPeriodicJobs(projectId, 1)(request)
      val lastPeriodicScheduled = contentAsJson(getBeforeResponse).as[Seq[JobInstance]].head

      status(getBeforeResponse) mustBe OK
      lastPeriodicScheduled.id mustBe scheduledJobId

      // delete created periodic job
      val deleteResponse = periodicJobsController.deletePeriodicJob(projectId, scheduledJobId)(request)

      status(deleteResponse) mustBe OK

      // check GET after deletion
      val getAfterResponse = periodicJobsController.getPeriodicJobs(projectId, 1)(request)
      val lastScheduled = contentAsJson(getAfterResponse).as[Seq[JobInstance]].head

      status(getAfterResponse) mustBe OK
      lastScheduled.id mustBe (scheduledJobId - 1)
    }

    "DELETE periodic job: disabled status" in {

      val deleteResponse = periodicJobsController.deletePeriodicJob(projectId, periodicJobId)(request)

      status(deleteResponse) mustBe OK
    }

    "DELETE periodic job: readonly access" in {

      val readonly = ProjectTestData.Access.readOnly
      val deleteResponse = periodicJobsController.deletePeriodicJob(readonly, periodicJobId)(request)

      status(deleteResponse) mustBe FORBIDDEN
      contentAsString(deleteResponse) mustBe SecurityService.UserAccessMessage
    }

    "DELETE periodic job: no existing job with id" in {
      val deleteResponse = periodicJobsController.deletePeriodicJob(projectId, scheduledJobId)(request)

      status(deleteResponse) mustBe FORBIDDEN
      contentAsString(deleteResponse) mustBe SecurityService.JobInstanceToProjectMessage
    }

    "PUT cancel periodic job: basic" in {

      addActivePeriodicJob(projectId, scheduledJobId)

      // check GET after creating periodic job and before deleting it
      val getBeforeResponse = periodicJobsController.getPeriodicJobs(projectId, 1)(request)
      val lastPeriodicScheduled = contentAsJson(getBeforeResponse).as[Seq[JobInstance]].head

      status(getBeforeResponse) mustBe OK
      lastPeriodicScheduled.id mustBe scheduledJobId
      lastPeriodicScheduled.status mustBe RunningStatus.Enabled

      // cancel periodic job
      cancel(projectId, scheduledJobId)

      // check GET after cancelling
      val getAfterResponse = periodicJobsController.getPeriodicJobs(projectId, 1)(request)
      val lastScheduled = contentAsJson(getAfterResponse).as[Seq[JobInstance]].head

      status(getAfterResponse) mustBe OK
      lastScheduled.status mustBe RunningStatus.Disabled
    }

    "PUT cancel periodic job: already disabled" in {

      val cancelJobResponse = periodicJobsController.disable(projectId, periodicJobId)(request)

      status(cancelJobResponse) mustBe UNPROCESSABLE_ENTITY
    }

    "PUT cancel periodic job: readonly access" in {

      val cancelJobResponse = periodicJobsController.disable(readonlyProject, periodicJobId)(request)

      status(cancelJobResponse) mustBe FORBIDDEN
      contentAsString(cancelJobResponse) mustBe SecurityService.UserAccessMessage

    }

    "PUT cancel periodic job: no existing job with id" in {

      val cancelJobResponse = periodicJobsController.disable(projectId, 4920)(request)

      status(cancelJobResponse) mustBe FORBIDDEN
      contentAsString(cancelJobResponse) mustBe SecurityService.JobInstanceToProjectMessage
    }


    "PUT enable periodic job: basic" in {

      addActivePeriodicJob(projectId, scheduledJobId)

      // cancel periodic job
      cancel(projectId, scheduledJobId)

      val enableJobResponse = periodicJobsController.enable(projectId, scheduledJobId)(request)
      status(enableJobResponse) mustBe OK

      val getResponse = periodicJobsController.getPeriodicJobs(projectId, 1)(request)
      val periodicJob = contentAsJson(getResponse).as[Seq[JobInstance]].head

      status(getResponse) mustBe OK
      periodicJob.id mustBe scheduledJobId
      periodicJob.status mustBe RunningStatus.Enabled

      // cancel periodic job
      cancel(projectId, scheduledJobId)
    }

    "PUT enable periodic job: already enabled" in {

      val enabledJobId = JobTestData.PeriodicJob.Status.enabledId

      val cancelEnabledResponse = periodicJobsController.enable(projectId, enabledJobId)(request)

      status(cancelEnabledResponse) mustBe UNPROCESSABLE_ENTITY
    }

    "PUT enable periodic job: readonly access" in {

      val enableResponse = periodicJobsController.enable(readonlyProject, periodicJobId)(request)

      status(enableResponse) mustBe FORBIDDEN
      contentAsString(enableResponse) mustBe SecurityService.UserAccessMessage
    }

    "PUT enable periodic job: no existing job found with id" in {

      val enableResponse = periodicJobsController.enable(projectId, scheduledJobId)(request)

      status(enableResponse) mustBe FORBIDDEN
      contentAsString(enableResponse) mustBe SecurityService.JobInstanceToProjectMessage
    }
  }
}
\end{lstlisting}
\subsection{controllers/MembershipSpec.scala}
\begin{lstlisting}
package controllers

import play.api.test.FakeRequest
import utils.{AuthSpecification, DatabaseCleaner, TestHelper}
import play.api.test.Helpers._
import utils.data.UserData._
import com.mohiva.play.silhouette.test._
import models.common.enums.MembershipAccessRight
import models.responses.Member
import models.services.SecurityService
import utils.data.ProjectTestData


class MembershipSpec extends AuthSpecification
                     with DatabaseCleaner {

  val controller: MembershipController = app.injector.instanceOf[MembershipController]

  private val readAndWriteProject = ProjectTestData.Access.readAndWrite
  private val noAccessProject = ProjectTestData.Access.noAccess
  private val ownerProject = ProjectTestData.Access.owner

  def checkGETLength(projectId: Long, expectedLength: Int): Unit = {

    val getResponse = controller.getParticipants(projectId)(FakeRequest().withAuthenticator(loginInfo))

    status(getResponse) mustBe OK
    val members = contentAsJson(getResponse).as[Seq[Member]]

    members.length mustBe expectedLength
  }

  "MembershipController" should {

    "GET members: ReadAndWrite access" in {

      checkGETLength(readAndWriteProject, 2)
    }

    "GET members: no access to project" in {

      val getResponse = controller.getParticipants(noAccessProject)(FakeRequest().withAuthenticator(loginInfo))

      status(getResponse) mustBe FORBIDDEN
      contentAsString(getResponse) mustBe SecurityService.UserAccessMessage
    }

    "DELETE member: Owner access" in {

      val requestWithAuthenticator = FakeRequest().withAuthenticator(loginInfo)
      val deleteResponse = controller.deleteParticipant(ownerProject, sampleUser.id)(requestWithAuthenticator)

      status(deleteResponse) mustBe OK

      checkGETLength(ownerProject, 1)
    }

    "DELETE member: ReadAndWrite access" in {

      val requestWithAuthenticator = FakeRequest().withAuthenticator(loginInfo)
      val deleteResponse = controller.deleteParticipant(readAndWriteProject, sampleUser.id)(requestWithAuthenticator)

      status(deleteResponse) mustBe FORBIDDEN
      contentAsString(deleteResponse) mustBe SecurityService.UserAccessMessage

      checkGETLength(readAndWriteProject, 2)
    }

    "PUT member: Owner access" in {

      checkGETLength(ownerProject, 1)

      val requestWithAuthenticator = FakeRequest().withAuthenticator(loginInfo)
      val putResponse = controller.addParticipants(ownerProject, sampleUser.id, MembershipAccessRight.Readonly)(requestWithAuthenticator)

      status(putResponse) mustBe OK

      // after adding new participant
      checkGETLength(ownerProject, 2)
    }

    "PUT member: ReadAndWrite access" in {

      val requestWithAuthenticator = FakeRequest().withAuthenticator(loginInfo)
      val putResponse = controller.addParticipants(readAndWriteProject, sampleUser.id, MembershipAccessRight.Readonly)(requestWithAuthenticator)

      status(putResponse) mustBe FORBIDDEN
      contentAsString(putResponse) mustBe SecurityService.UserAccessMessage
    }
  }

}
\end{lstlisting}
\subsection{controllers/ProjectSpec.scala}
\begin{lstlisting}
package controllers


import akka.stream.Materializer
import utils.{AuthSpecification, DatabaseCleaner, TestHelper}
import play.api.test._
import com.mohiva.play.silhouette.test._
import forms.project.{ProjectChangeForm, ProjectForm}
import models.tables.Crawler
import play.api.test.Helpers._
import play.api.libs.json.{JsArray, JsNumber, JsString, Json, OFormat}
import utils.data.ProjectTestData
import utils.data.UserData._


class ProjectSpec extends AuthSpecification
                  with DatabaseCleaner  {

  implicit val crawlerFormat: OFormat[Crawler] = Json.format[Crawler]

  private val controller = app.injector.instanceOf[ProjectsController]
  implicit lazy val materializer: Materializer = app.materializer
  
  private val readAndWriteProject = ProjectTestData.Access.readAndWrite
  private val readonlyProject = ProjectTestData.Access.readOnly

  object Data {
    val projectForm: ProjectForm = ProjectForm(
      name = "New project",
      description = Some("Hello world description")
    )
    val changeProjectForm: ProjectChangeForm = ProjectChangeForm(
      name = "New name",
      description = Some("New description"),
      spiderSettings = Map("XYZ" -> "Z"),
      spiderArgs = Map("args" -> "something")
    )
  }

  "ProjectsController" should {

    "GET list of projects for user" in {

      val result = controller.getProjects(10, None)(FakeRequest().withAuthenticator(loginInfo))

      status(result) mustBe OK

      contentAsJson(result) match {
        case JsArray(value) =>
          value must have length 10
          (value(0) \ "id").get must equal(JsNumber(16))
        case _ => assertTypeError("Error type")
      }
    }

    "GET list of projects: with pagination" in {

      val result = controller.getProjects(10, Some(13))(FakeRequest().withAuthenticator(loginInfo))

      status(result) mustBe OK

      contentAsJson(result) match {
        case JsArray(value) =>
          value must have length 10
          (value(0) \ "id").get must equal(JsNumber(12))
        case _ => assertTypeError("Error type")
      }
    }

    "CREATE project" in {

      val request = FakeRequest()
        .withAuthenticator(loginInfo)
        .withBody(Data.projectForm)
      val createResult = controller.createProject(request)

      status(createResult) mustBe OK
      contentAsString(createResult) mustBe "17"

      val getResult = controller.getProjects(1, None)(FakeRequest().withAuthenticator(loginInfo))

      status(getResult) mustBe OK
      contentAsJson(getResult) match {
        case JsArray(value) =>
          value must have length 1
          (value(0) \ "id").get must equal(JsNumber(17))
          (value(0) \ "name").get must equal(JsString(Data.projectForm.name))
          (value(0) \ "description").get must equal(JsString(Data.projectForm.description.getOrElse("")))
        case _ => assertTypeError("Error type")
      }
    }


    "PUT project's metadata" in {

      val request = FakeRequest()
        .withAuthenticator(loginInfo)
        .withBody(Data.changeProjectForm)

      val putResult = controller.updateProjectMetadata(16)(request)

      status(putResult) mustBe OK

      val getResult = controller.getProjects(1)(FakeRequest().withAuthenticator(loginInfo))

      status(getResult) mustBe OK
      contentAsJson(getResult) match {
        case JsArray(value) =>
          value must have length(1)
          (value(0) \ "id").get must equal(JsNumber(16))
          (value(0) \ "name").get must equal(JsString(Data.changeProjectForm.name))
          (value(0) \ "description").get must equal(JsString(Data.changeProjectForm.description.getOrElse("")))
          (value(0) \ "spidersSettings").as[Map[String, String]] must equal(Data.changeProjectForm.spiderSettings)
          (value(0) \ "args").as[Map[String, String]] must equal(Data.changeProjectForm.spiderArgs)
        case _ => assertTypeError("Error type")
      }
    }

    "PUT project's metadata: Readonly access - no permission" in {

      val request = FakeRequest()
        .withAuthenticator(loginInfo)
        .withBody(Data.changeProjectForm)

      val putResult = controller.updateProjectMetadata(readonlyProject)(request)

      status(putResult) mustBe FORBIDDEN
    }

    "PUT project's metadata: ReadAndWrite access" in {

      val request = FakeRequest()
        .withAuthenticator(loginInfo)
        .withBody(Data.changeProjectForm)

      val putResult = controller.updateProjectMetadata(readAndWriteProject)(request)

      status(putResult) mustBe OK
    }

    "DELETE project: Owner access" in {

      val request = FakeRequest().withAuthenticator(loginInfo)
      val deleteRequest = controller.deleteProject(16)(request)

      status(deleteRequest) mustBe OK

      val getRequest = controller.getProjects(20)(request)

      status(getRequest) mustBe OK
      contentAsJson(getRequest) match {
        case JsArray(value) =>
          value must have length(14)
          (value(0) \ "id").get must equal(JsNumber(15))
        case _ => assertTypeError("Error type")
      }
    }

    "DELETE project: NOT Owner access" in {

      val request = FakeRequest().withAuthenticator(loginInfo)
      val deleteRequest = controller.deleteProject(readAndWriteProject)(request)

      status(deleteRequest) mustBe FORBIDDEN
    }

    "DELETE project: doesn't exist" in {

      val request = FakeRequest().withAuthenticator(loginInfo)
      val deleteRequest = controller.deleteProject(122)(request)

      status(deleteRequest) mustBe FORBIDDEN
    }

    "PUT deploy" in {

      val deployResult = controller.deployProject(15)(TestHelper.requestWithMetadata("egg1.egg"))

      status(deployResult) mustBe OK
      contentAsJson(deployResult) mustBe(Json.toJson(TestHelper.crawlers))
    }

    "PUT deploy: wrong format file" in {

      val deployResult = controller.deployProject(15)(TestHelper.requestWithMetadata("invalid.egg"))

      status(deployResult) mustBe UNPROCESSABLE_ENTITY
    }

    "PUT deploy: no access" in {

      val deployResult = controller.deployProject(readonlyProject)(TestHelper.requestWithMetadata("egg1.egg"))

      status(deployResult) mustBe FORBIDDEN
    }
  }
}
\end{lstlisting}
\subsection{unit/EmailValidatorTest.scala}
\begin{lstlisting}
package unit

import org.scalatest.FunSuite
import models.common.extensions.ValidateString

class EmailValidatorTest extends FunSuite {

  test("EmailString.wrongEmail") {
    assert(!"helpmail.ru".isEmail)
    assert(!"helpп@mail.ru".isEmail)
    assert(!"help123@.ru".isEmail)
    assert(!"help@mail.".isEmail)
    assert(!"help@ma il.go".isEmail)
  }

  test("EmailString.validEmail") {
    assert("help@mail.ru".isEmail)
    assert("help123@mail.ru".isEmail)
  }
}
\end{lstlisting}
\subsection{unit/SettingsMergerTest.scala}
\begin{lstlisting}
package unit


import models.common.extensions._
import org.scalatest.FunSuite
import play.api.libs.json._
import models.common.settings.{ScrapydSettings, SettingsFromDB, SettingsMerger}

class SettingsMergerTest extends FunSuite {

  object Data {
    val settings: JsValue = Json.parse("""{"DOWNLOAD_DELAY":"2", "XYZ":"S"}""")
    val settings2: JsValue = Json.parse("""{"DOWNLOAD_DELAY":"200", "ABD":"S"}""")
    val settings3: JsValue = Json.parse("""{"DOWNLOAD_DELAY":"3030", "X":"XJS"}""")
  }


  test("SettingsMerger.basic") {

    val jsonToSettings = Data.settings

    assert(Some(jsonToSettings).toMap === Map("DOWNLOAD_DELAY" -> "2", "XYZ" -> "S"))
  }

  test("SettingsMerger.wrongInput") {
    val jsonToArgs = Json.parse("""["hello", "hehehe"]""")

    assert(Some(jsonToArgs).toMap === Map.empty)
  }

  test("SettingsMerger.testPriority") {

    val projectData = SettingsFromDB(settings = Some(Data.settings), args = Some(Data.settings))
    val spiderData = SettingsFromDB(settings = Some(Data.settings2), args = Some(Data.settings2))
    val jobData = SettingsFromDB(settings = Some(Data.settings3), args = Some(Data.settings3))

    val resultSettingsMap = Map(
      "DOWNLOAD_DELAY" -> "3030",
      "XYZ" -> "S",
      "ABD" -> "S",
      "X" -> "XJS"
    )
    val resultSettingsSeq = resultSettingsMap.map { case (str, str1) => s"$str=$str1" }.toSeq

    val result = SettingsMerger.mergeSettings(projectData, spiderData, jobData)
    assert(result === ScrapydSettings(setting = resultSettingsSeq, args = resultSettingsMap))
  }

}
\end{lstlisting}
\subsection{utils/AuthSpecification.scala}
\begin{lstlisting}
package utils

import java.util.UUID

import com.mohiva.play.silhouette.api.actions.{SecuredErrorHandler, SecuredErrorHandlerModule, UnsecuredErrorHandler, UnsecuredErrorHandlerModule}
import com.mohiva.play.silhouette.api.{Environment, LoginInfo, Silhouette, SilhouetteProvider}
import com.mohiva.play.silhouette.impl.providers.CredentialsProvider
import com.mohiva.play.silhouette.test.FakeEnvironment
import forms.{SignIn, SignUp}
import models.tables.User
import net.codingwell.scalaguice.ScalaModule
import play.api.Application
import play.api.inject.guice.GuiceApplicationBuilder
import utils.data.UserData

import scala.concurrent.ExecutionContext.Implicits.global

/**
 * A specification which contains some auth specific configuration.
 */
abstract class AuthSpecification extends BaseSpecification {

  import UserData._

  /**
   * The fake environment.
   */
  implicit val fakeEnv: FakeEnvironment[DefaultEnv] = FakeEnvironment[DefaultEnv](Seq(loginInfo -> userExample))

  /**
   * The silhouette module used to instantiate the application.
   */
  def silhouetteModule: ScalaModule = new ScalaModule {
    def configure(): Unit = {
      bind[Environment[DefaultEnv]].toInstance(fakeEnv)
      bind[Silhouette[DefaultEnv]].to[SilhouetteProvider[DefaultEnv]]
    }
  }

  /**
   * The application builder.
   */
  override def fakeApplication(): Application = {
    val builder = overrideDependencies(
      new GuiceApplicationBuilder()
        .overrides(silhouetteModule)
    )
    builder.build()
  }

  def overrideDependencies(application: GuiceApplicationBuilder): GuiceApplicationBuilder = {
    application
  }
}
\end{lstlisting}
\subsection{utils/BaseSpecification.scala}
\begin{lstlisting}
package utils

import java.time.{Clock, Instant, ZoneId}

import net.codingwell.scalaguice.ScalaModule
import org.scalatestplus.play.PlaySpec
import org.scalatestplus.play.guice.GuiceOneAppPerSuite
import org.specs2.specification.Scope
import play.api.Application
import play.api.i18n.{Lang, Messages, MessagesApi}
import play.api.inject.Injector
import play.api.inject.guice.GuiceApplicationBuilder
import play.api.test.PlaySpecification

/**
 * A specification which contains some helpers.
 */
abstract class BaseSpecification extends PlaySpec with GuiceOneAppPerSuite {

  /**
   * The fake module used to instantiate the application.
   */
  def fakeModule: ScalaModule = new ScalaModule {
    def configure(): Unit = {}
  }

  /**
   * The application builder.
   */
  def applicationBuilder: GuiceApplicationBuilder = new GuiceApplicationBuilder()
    .overrides(fakeModule)

  /**
   * The application.
   */
  def application: Application = applicationBuilder.build()

  /**
   * The Guice injector.
   */
  def injector: Injector = application.injector

  /**
   * The Play lang.
   */
  def lang: Lang = Lang("en-US")


  /**
   * The current clock.
   */
  lazy val clock = Clock.fixed(Instant.now(), ZoneId.of("UTC"))
}
\end{lstlisting}
\subsection{utils/data/CrawlerData.scala}
\begin{lstlisting}
package utils.data

import models.tables.Crawler

object CrawlerData {

  def getCrawlers: Seq[Crawler] = {

    val pIds = Seq(1,2, JobTestData.OnetimeJob.Owner.projectId, JobTestData.PeriodicJob.Owner.projectId)
    pIds.map(id => getCrawlerForProjectId(id))
  }

  private def getCrawlerForProjectId(projectId: Long): Crawler = {
    Crawler(projectId = projectId, name = s"crawler in project ${projectId}")
  }

}
\end{lstlisting}
\subsection{utils/data/JobTestData.scala}
\begin{lstlisting}
package utils.data

import models.common.enums.RunningStatus.RunningStatus
import models.common.enums.RunningType
import models.tables.JobInstance

object JobTestData {

  object OnetimeJob {

    val ids: Array[Int] = 1 to 20 toArray

    object Owner {
      val projectId: Long = 13
      val crawlerId: Long = 3
    }

    def getJobInstance: JobInstance = {
      JobInstance(projectId = Owner.projectId, spider = Owner.crawlerId, runType = RunningType.Onetime)
    }
  }


  object PeriodicJob {

    val idsEnabled: Array[Int] = 21 to Status.enabledId toArray
    val idsDisabled: Array[Int] = 31 to Status.disabledId toArray

    def getJobInstance(id: Long, status: RunningStatus): JobInstance = {
      JobInstance(
        id = id,
        projectId = Owner.projectId,
        spider = Owner.crawlerId,
        runType = RunningType.Periodic,
        status = status)
    }

    val validCronExpression = "* * * * * ? *"

    object Status {
      val enabledId = 30
      val disabledId = 40
    }

    object Owner {
      val projectId: Long = 13
      val crawlerId: Long = 4
    }

  }
}
\end{lstlisting}
\subsection{utils/data/ProjectTestData.scala}
\begin{lstlisting}
package utils.data

import java.time.Instant

import models.tables.Project

object ProjectTestData {

  object Access {
    val readOnly = 1
    val readAndWrite = 2
    val noAccess = 3
    val owner = 4
  }

  def getProject(withId: Long = 0): Project = {
    Project(
      name = s"${withId} project",
      description = Some("projectForm.description"),
      ownerId = UserData.userExample.id,
      createdAt = Instant.now(),
      changedBy = UserData.userExample.id,
      changedAt = Instant.now()
    )
  }

}
\end{lstlisting}
\subsection{utils/data/UserData.scala}
\begin{lstlisting}
package utils.data

import java.util.UUID

import com.mohiva.play.silhouette.api.LoginInfo
import com.mohiva.play.silhouette.impl.providers.CredentialsProvider
import forms.{SignIn, SignUp}
import models.tables.User

object UserData {

  val email = "zdaria99@mail.ru"
  val authorizationEmail = "exampleEmail@mail.ru"
  val sampleUserEmail = "hello@mail.ru"
  val loginInfo = LoginInfo(CredentialsProvider.ID, email)

  val credentials = SignIn(authorizationEmail, "password")

  /**
   * Form to create new user in DB.
   * */
  val signUpForm = SignUp(
    name = "Dasha",
    login = "unique_name",
    email = credentials.email,
    password = credentials.password
  )

  val userExample = User(
    id = UUID.fromString("0375dc2c-6d44-4096-a35b-152b8c2568dc"),
    name = "Dasha",
    login = "dashatest",
    email = email,
    providerID = "credentials",
    providerKey = email)

  val sampleUser = User(
    id = UUID.randomUUID(),
    name = "User",
    login = "some_user",
    email = sampleUserEmail,
    providerID = "credentials",
    providerKey = sampleUserEmail
  )

}
\end{lstlisting}
\subsection{utils/DatabaseCleaner.scala}
\begin{lstlisting}
package utils

import models.common.{DBCreator, PGProfile}
import org.scalatest.{BeforeAndAfterAll, BeforeAndAfterEach, Suite}
import play.api.db.slick.{DatabaseConfigProvider, HasDatabaseConfigProvider}

import scala.concurrent.ExecutionContext.Implicits.global
import scala.util.Try

trait DatabaseCleaner extends HasDatabaseConfigProvider[PGProfile]
                                with BeforeAndAfterEach
                                with BeforeAndAfterAll {
  this: Suite with BaseSpecification =>


  override lazy val dbConfigProvider: DatabaseConfigProvider =
    app.injector.instanceOf[DatabaseConfigProvider]

  override protected def beforeEach(): Unit = {
    super.beforeEach()
    createDB()
  }

  override protected def afterEach(): Unit = {
    dropDB()
    super.afterEach()
  }

  override protected def beforeAll(): Unit = {
    super.beforeAll()
    Try(DBCreator.downTypes(db))
    DBCreator.upTypes(db)
  }

  def createDB(): Unit = {
    Try(dropDB())
    DBCreator.up(db)
    DBTestFiller.fillData(db)
  }

  def dropDB(): Unit = {
    DBCreator.down(db)
  }
}
\end{lstlisting}
\subsection{utils/DBTestFiller.scala}
\begin{lstlisting}
package utils


import java.time.Instant

import models.common.PGProfile.api._
import models.tables._
import models.common.extensions._
import models.common.enums.{JobExecutionStatus, MembershipAccessRight, RunningStatus}
import utils.data.{CrawlerData, JobTestData, ProjectTestData}

import scala.concurrent.ExecutionContext

object DBTestFiller {

  import utils.data.UserData._
  import slickProfile.api._
  import utils.data.JobTestData._

  val user = User.dbUsers
  val password = Password.dbPasswords
  val project = Project.dbProjects
  val crawler = Crawler.dbCrawlers
  val jInstance = JobInstance.dbJobInstances
  val jExecution = JobExecution.dbJobExecutions
  val membership = Membership.dbMembership

  def fillData(db: Database)(implicit ec: ExecutionContext): Unit = {
    db.run(generateUserData andThen generateProjectData andThen generateCrawlerData
      andThen generateJobExecutionData).awaitForResult
  }

  private def generateUserData() = {
    user ++= Seq(userExample, sampleUser)
  }

  /**
   * Generate 15 projects which user owns
   */
  private def generateProjectData()(implicit ec: ExecutionContext): DBIO[Unit] = {

    val insertProject = project returning project.map(_.id)
    val userId = userExample.id

    val projectForTestReadonly = ProjectTestData.getProject(1)
    val projectForTestReadAndWrite = ProjectTestData.getProject(2)
    val projectNoAccess = ProjectTestData.getProject(3)

    val projectsForTest = (4 to 16 toArray).map { pId => ProjectTestData.getProject(pId) }

    (for {
      pIdReadonly <- insertProject += projectForTestReadonly // 1
      pIdReadAndWrite <- insertProject += projectForTestReadAndWrite // 2
      pIdNoAccess <- insertProject += projectNoAccess // 3
      projectIds <- insertProject ++= projectsForTest

      memberships = projectIds.map { id =>  Membership(userId, id, MembershipAccessRight.Owner) }

      _ <- membership ++= memberships

      _ <- membership += Membership(sampleUser.id, pIdReadonly, MembershipAccessRight.Owner)
      _ <- membership += Membership(sampleUser.id, pIdReadAndWrite, MembershipAccessRight.Owner)
      _ <- membership += Membership(sampleUser.id, pIdNoAccess, MembershipAccessRight.Owner)

      _ <- membership += Membership(userId, pIdReadonly, MembershipAccessRight.Readonly)
      _ <- membership += Membership(userId, pIdReadAndWrite, MembershipAccessRight.ReadAndWrite)

    } yield ()).transactionally

  }

  /**
   * Data for testing CrawlerController
   */
  private def generateCrawlerData()(implicit ec: ExecutionContext): DBIO[Unit] = {
    (for {
      _ <- crawler ++= CrawlerData.getCrawlers
    } yield ()).transactionally
  }

  /**
   * Data for testing job executions
   */
  private def generateJobExecutionData()(implicit ec: ExecutionContext): DBIO[Unit] = {

    val jInstanceInsert = jInstance returning jInstance.map(_.id)
    val jExecutionInsert = jExecution returning jExecution.map(job => (job.id, job.scrapydId))

    val onetimeJobs = OnetimeJob.ids.map(_ => JobTestData.OnetimeJob.getJobInstance)
    val periodicJobsEnabled = PeriodicJob.idsEnabled.map(id => PeriodicJob.getJobInstance(id, RunningStatus.Enabled))
    val periodicJobsDisabled = PeriodicJob.idsDisabled.map(id => PeriodicJob.getJobInstance(id, RunningStatus.Disabled))

    val periodicJobs = (periodicJobsEnabled ++ periodicJobsDisabled)

    for {
      ids1 <- jInstanceInsert ++= onetimeJobs
      ids2 <- jInstanceInsert ++= periodicJobs

      jobExecutions = ids1.map { id =>
        JobExecution(jobInstanceId = id, createTime = Instant.now(), status = JobExecutionStatus.Finished)
      }
      insertedJobExecutions <- jExecutionInsert ++= jobExecutions
    } yield {
      TestHelper.insertedJobExecutions = insertedJobExecutions.map { case (l, uuid) => (JobTestData.OnetimeJob.Owner.projectId, l, uuid) }
      TestHelper.insertedPeriodicJobs = ids2.zip(periodicJobs).toMap
    }
  }

}
\end{lstlisting}
\subsection{utils/TestHelper.scala}
\begin{lstlisting}
package utils

import java.nio.file.Paths
import java.util.UUID

import com.mohiva.play.silhouette.api.Environment
import models.tables.{Crawler, JobInstance}
import play.api.libs.Files
import play.api.libs.Files.SingletonTemporaryFileCreator
import play.api.libs.json.{Json, OFormat}
import play.api.mvc.MultipartFormData
import play.api.mvc.MultipartFormData.FilePart
import play.api.test._
import utils.data.UserData._
import com.mohiva.play.silhouette.test._
import forms.OnetimeJobForm
import forms.periodic.{PeriodicJobChangeForm, PeriodicJobCreateForm}
import models.common.enums.JobPriority
import utils.data.JobTestData


object TestHelper {

  private val baseDir = "test/data/egg/"

  val projectWithCrawlers = 15

  var insertedJobExecutions: Seq[(Long, Long, UUID)] = Seq.empty
  var insertedPeriodicJobs: Map[Long, JobInstance] = Map.empty

  implicit val crawlerFormat: OFormat[Crawler] = Json.format[Crawler]

  val crawlers: Seq[Crawler] = Seq(
    Crawler(5, projectWithCrawlers, "toscrape-css"),
    Crawler(6, projectWithCrawlers, "toscrape-xpath")
  )

  val onetimeJobForm = OnetimeJobForm(
    crawlerId = crawlers.head.id,
    priority = JobPriority.Normal,
    settings = Map("DOWNLOAD_DELAY" -> "100")
  )

  val periodicJobForm = PeriodicJobCreateForm(
    title = "Periodic Job",
    crawlerId = crawlers.head.id,
    cronExpression = JobTestData.PeriodicJob.validCronExpression
  )

  val changePeriodicJobForm = PeriodicJobChangeForm(
    title = "Periodic Job",
    crawlerId = crawlers.head.id,
    cronExpression = JobTestData.PeriodicJob.validCronExpression
  )

  def requestWithMetadata(fileName: String)(implicit env: Environment[DefaultEnv]): FakeRequest[MultipartFormData[Files.TemporaryFile]] = {
    val filePart = FilePart(
      "eggFile",
      baseDir + fileName,
      Some("text/plain;charset=UTF-8"),
      SingletonTemporaryFileCreator.create(Paths.get(baseDir + fileName))
    )

    FakeRequest()
      .withAuthenticator(loginInfo)
      .withBody(MultipartFormData(
        dataParts = Map.empty,
        files = Seq(filePart),
        badParts = Nil))
  }
}
\end{lstlisting}
