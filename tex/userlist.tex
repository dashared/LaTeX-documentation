\label{subsub: userlist}

\begin{center}
\begin{longtable}{|p{0.2\linewidth}|p{0.35\linewidth}|p{0.4\linewidth}|}
\caption{Список пользователей} 
\label{table: userlist} \\

\hline
\multicolumn{1}{|c|}{\textbf{Роль}} & \multicolumn{1}{c|}{\textbf{Описание}} & \multicolumn{1}{c|}{\textbf{Способ работы}} \\ \hline
\endfirsthead

\multicolumn{3}{r}%
{{ \tablename\ \thetable{} -- продолжение}} \\ 
\hline 
\multicolumn{1}{|c|}{\textbf{Роль}} & \multicolumn{1}{c|}{\textbf{Описание}} & \multicolumn{1}{c|}{\textbf{Способ работы}} \\
\hline
\endhead

\multicolumn{3}{r}{{Продолжение на следующей странице}} \\ 
\endfoot

\hline 
\endlastfoot

Администратор & Сотрудник фонда, задача которого упарвлять пользователями системы, а также мониторить логи. Возраст 20-70 лет. Работает на 1/2 ставку.  & Управляет всеми учетными записями в системе. У него есть доступ к логам системы, а также возможность просматривать пользователей системы и регистрировать их. \\ \hline
Оператор &  Является сотрудником службы поддержки доноров и нуждающихся. Возраст 20-70 лет. Работает на 1/2 ставку. & Отвечает на сообщения пользователей, отправленные через чат. \\ \hline
Контент-менеджер & Отвечает за контент, размещаемый на основной странице фонда. Возраст 20-70 лет. Работает на 1/2 ставку. &  Имеет возможность редактировать описание фонда, а также раздел часто задаваемых вопросов и новости фонда. \\ \hline
Менеджер & Обрабатывает заявки, поступающие в фонд. Возраст 20-70 лет. Полная занятость.  & Имеет доступ к заявкам фонда, а также может обрабатывать данные заявки: менять статусы, редактировать и общаться с пользователем в рамках одной заявки. Также менеджер может просматривать транзакции фонда. \\ \hline
Член комиссии & Члены комиссии фонда принимают решение об опубликовании заявок. Возраст 20-70 лет. Полная занятость. & Имеет доступ к расширенному функционалу управления заявками, а именно к активации заявки (активная заявка становится видимой для всех пользователей системы, на нее можно жертвовать деньги).\\ \hline
\end{longtable}
\end{center}