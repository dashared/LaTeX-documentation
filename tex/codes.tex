\label{subsub: userlist}

\begin{center}
\begin{longtable}{|p{0.2\linewidth}|p{0.35\linewidth}|p{0.4\linewidth}|}
\caption{Статус-коды ответов от сервера} 
\label{table: userlist} \\

\hline
\multicolumn{1}{|c|}{\textbf{Код}} & \multicolumn{1}{c|}{\textbf{Описание}} & \multicolumn{1}{c|}{\textbf{Обработка}} \\ \hline
\endfirsthead

\multicolumn{3}{r}%
{{ \tablename\ \thetable{} -- продолжение}} \\ 
\hline 
\multicolumn{1}{|c|}{\textbf{Код}} & \multicolumn{1}{c|}{\textbf{Описание}} & \multicolumn{1}{c|}{\textbf{Обработка}} \\
\hline
\endhead

\multicolumn{3}{r}{{Продолжение на следующей странице}} \\ 
\endfoot

\hline 
\endlastfoot
\label{table: codes}
200 & Успешный ответ  & В случае успешного ответа обрабатываем полученное тело ответа и демонстрируем его в UI \\ \hline
400 & Плохие входные данные  & В случае некорректного введенного и отправленного значения высвечивается уведомление на UI \\ \hline

401 & Пользователь не авторизован & В случае получения данной ошибки пользователь не авторизован, либо закончился срок действия авторизационного токена. Либо отображается страница с авторизационной формой, либо перезапрашивается авторизационный токен через эндпойнт \texttt{refresh} \\ \hline

403 & Недостаточно прав доступа & В случае получения данной ошибки пользователь перешел на url, к которому у него нет доступа. В данном случае высвечивается соответствующая ошибка в виде уведомления в UI \\ \hline
404 & Запрашиваемый контент не найден & В случае получения данной ошибки либо url запроса некорретный, либо в базе данных не найдена сущность по указанным в запросе параметрам. В данной ситуации обработка подразумевает демонстрацию ошибки на UI \\ \hline

500 & Ошибка сервера & Данная ошибка демонстрируется в виде нотификации на UI \\ \hline


\end{longtable}
\end{center}