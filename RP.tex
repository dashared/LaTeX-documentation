\documentclass[a4paper,12pt]{article}
\usepackage{styledoc19}


\begin{document} % конец преамбулы, начало документа
    \year{2020}
    \docNumber{RU.17701729.04.13-01 34 01-1}
	\docFormat{Руководство программиста}
	\student{БПИ 174}{Д. Ю. Редникина}
	\supervisor{Профессор департамента \vfill программной инженерии  факультета компьютерных наук, к.т.н}
	{Е. М. Гринкруг}
	\project{СИСТЕМА УПРАВЛЕНИЯ ЗАДАНИЯМИ ПО АВТОМАТИЧЕСКОМУ СБОРУ ДАННЫХ ИЗ СЕТИ ИНТЕРНЕТ}
	
	\firstPage
						\newpage
	\secondPage
						\newpage
	\thirdPage
						\newpage
	\section{Назначение программы}
	\subsection{Функциональное назначение}
	Система будет применяться как средство управления проектами по созданию, редактированию и запуску веб краулеров для сбора данных в сети интернет. Продукт позволит следить за запусками в режиме реального времени, а также создавать периодические запуски по расписанию.

	\subsection{Эксплуатационное назначение}
	Web-приложение является компонентом CRM-системы для благотворительного фонда <<AIAIN>>, позволяющей облегчить бизнес процессы работы фонда с благополучателями и донорами. Web-приложение призвано обеспечить необходимую сотрудникам фонда функциональность для работы с системой. Им будут пользоваться как менеджеры фонда, так и администраторы, члены комиссий, операторы фонда, контент-менеджеры фонда. Каждый из пользователей будет иметь доступ к необходимой ему функциональности по обработке заявок, управлению фондом, администрированию и т.д.
	\subsection{Состав выполняемых функций}
	Следующие требования зафиксированы в документе <<Cистема управления заданиями по автоматическому сбору данных из сети Интернет. Техническое задание>> к составу выполняемых функций:
	

\begin{enumerate}
		
		\item \textbf{Авторизация\\}
		Чтобы использовать сервис, клиентская программа должна иметь возможность авторизоваться в системе с помощью REST API
		\begin{enumerate}
			\item Для регистрации пользователю нужно указать следующие данные 
			\begin{enumerate}
			    \item Почта - уникальна для каждого зарегистрированного пользователя; 
			    \item Имя - длина больше 1 символ;
			    \item Логин - длина больше 2 символов;
			    \item Пароль - длина больше 2 символов;
			\end{enumerate}
			\item Для авторизации пользователя в системе должны быть указаны следующие данные
			\begin{enumerate}
			    \item Почта;
			    \item Пароль;
			\end{enumerate}
		\end{enumerate}
		
		\item \textbf{Проекты\\}
		Должны быть реализованы запросы REST API для предоставления клиенту следующей функциональности
		\begin{enumerate}
	    	\item Создание проекта со следующей информацией
	    	\begin{enumerate}
	    	    \item Имя проекта;
	    	    \item Описание проекта - опциональное поле;
	    	\end{enumerate}
			\item Обновление метаданных о проекте (редактирование) могут быть обновлены только участником с минимальным уровнем дотупа \texttt{ReadAndWrite}. Следующие данные могут быть обновлены:
			\begin{enumerate}
			    \item Имя проекта;
			    \item Описание проекта;
			    \item Настройки проекта для запуска краулеров;
			    \item Аргументы для запуска краулеров проекта;
			\end{enumerate}
			\item Обновление \texttt{egg} файла проекта (редактирование) -- минимальный уровень доступа участника, обновляющий данные о проекте \texttt{ReadAndWrite}.  
			\item Удаление данных о проекте. Удалить проект может только владелец \texttt{Owner}.
			\item Просмотр списка проектов (с пагинацией), к которым у пользователя есть как минимум \texttt{ReadOnly} доступ.
		\end{enumerate}
		
		\item \textbf{Участники проектов\\}
		Должны быть реализованы запросы REST API для предоставления клиенту следующей функциональности
		\begin{enumerate}
	    	\item Просмотр информации об участниках проекта;
	    	\begin{enumerate}
	    	    \item Имя, почта, логин участника;
	    	    \item Статус участника в проекте (\texttt{ReadOnly}, \texttt{ReadAndWrite} или \texttt{Owner});
	    	\end{enumerate}
			\item Обновление статуса участника проекта. Это действие совершать может только владелец проекта; 
			\item Удаление участника из проекта. Данное действие может совершать только владелец проекта;
			\item Добавление нового участника с указанными правами на редактирование. Данное действие может совершать только владелец проекта;
		\end{enumerate}
		
		\item \textbf{Краулеры\\}
		Должны быть реализованы запросы REST API для предоставления клиенту следующей функциональности
		\begin{enumerate}
			\item Просмотр списка краулеров проекта;
			\item Редактирование информации о краулере для последующих запусков. Следующая информация может быть изменена
			\begin{enumerate}
			    \item Настройки краулера для запуска;
			    \item Аргументы для запуска;
			\end{enumerate}
		\end{enumerate}
		
		\item \textbf{Запуски краулеров\\}
		Должны быть реализованы запросы REST API для предоставления клиенту следующей функциональности
		\begin{enumerate}
		    \item Просмотр списка запусков в определенном статусе (\texttt{Pending}, \texttt{Running} или \texttt{Finished}) с пагинацией, совершенных в проектах, к которым у пользователя есть как минимум \texttt{ReadOnly} доступ;
		    \item Редактирование запуска - остановка запуска, перевод его в состояние \texttt{Finished}. Операция может быть применена только к запускам в состоянии \texttt{Running} или \texttt{Pending};
		    \item Удаление запуска - удаление всех данных о запуске из базы данных. Операция может быть применена только к запускам в состоянии \texttt{Finished};
		    \item Создание запуска со следующей информацией
		    \begin{enumerate}
		        \item Краулер, с которым происходит запуск;
		        \item Настройки запуска --  это могут быть как и предопределенные настроки на \texttt{scrapyd} \footnote{\url{http://doc.scrapy.org/en/latest/topics/settings.html}}, так и собственные настройки;
		        \item Аргументы запуска -- аргументы для запуска краулера, которые передаются через командную строку;
		        \item Описание запуска;
		    \end{enumerate}
		\end{enumerate}
		
		\item \textbf{Периодические запуски\\}
		Должны быть реализованы запросы REST API для предоставления клиенту следующей функциональности
		\begin{enumerate}
		    \item Просмотр списка периодических запусков с пагинацией;
		    \item Редактирование следующей информации о периодическом запуске
		    \begin{enumerate}
		        \item Настройки будущих запусков --  это могут быть как и предопределенные настроки на \texttt{scrapyd}, так и собственные настройки;
		        \item Аргументы будущих запусков -- аргументы для запуска краулера, которые передаются через командную строку;
		        \item Краулер, с помощью которого будет совершен запуск;
		        \item cron-expression расписания запуска;
		    \end{enumerate}
		    \item Удаление периодического запуска;
		    \item Отмена последующих запусков - перевод периодической задачи в состояние \texttt{Disabled};
		    \item Возобновление запусков - перевод периодической задачи в состояние \texttt{Enabled};
		    \item Создание периодического запуска со следующими данными
		    \begin{enumerate}
		        \item Название;
		        \item Описание -- опциональное;
		        \item Краулер;
		        \item Приоритетность, влияющая на очередь запусков (\texttt{Low}, \texttt{Normal} или \texttt{High});
		        \item Статус (\texttt{Enabled} или \texttt{Disabled});
		        \item Настройки будущих запусков --  это могут быть как и предопределенные настроки на \texttt{scrapyd}, так и собственные настройки;
		        \item Аргументы будущих запусков -- аргументы для запуска краулера, которые передаются через командную строку;
		    \end{enumerate}
		\end{enumerate}
	
	\end{enumerate}

		
	
	
	\newpage
	
	\section{Обращение к сервису}
    
    К данному сервису обращение осуществляется посредством REST запросов \cite{rest}. Далее в этом разделе приведены все возможные запросы к сервису с примерами аргументов к нему.
    
	\subsection{API - список}
	{
\setcounter{secnumdepth}{3}

\hypertarget{crawlers}{%
\paragraph{\texorpdfstring{\protect\hyperlink{Crawlers}{Crawlers}}{Crawlers}}\label{crawlers}}

\begin{itemize}
\tightlist
\item
  \protect\hyperlink{listSpiders}{\texttt{get\ /api/projects/\{projectId\}/crawlers}}
\item
  \protect\hyperlink{updateSpider}{\texttt{put\ /api/projects/\{projectId\}/crawlers/\{crawlerId\}}}
\end{itemize}

\hypertarget{login}{%
\paragraph{\texorpdfstring{\protect\hyperlink{Login}{Login}}{Login}}\label{login}}

\begin{itemize}
\tightlist
\item
  \protect\hyperlink{signIn}{\texttt{post\ /api/auth/signin}}
\end{itemize}

\hypertarget{logout}{%
\paragraph{\texorpdfstring{\protect\hyperlink{Logout}{Logout}}{Logout}}\label{logout}}

\begin{itemize}
\tightlist
\item
  \protect\hyperlink{logout}{\texttt{get\ /api/auth/logout}}
\end{itemize}

\hypertarget{membership}{%
\paragraph{\texorpdfstring{\protect\hyperlink{Membership}{Membership}}{Membership}}\label{membership}}

\begin{itemize}
\tightlist
\item
  \protect\hyperlink{addParticipants}{\texttt{put\ /api/projects/\{projectId\}/memberships/\{guestId\}/\{guestAccess\}}}
\item
  \protect\hyperlink{deleteParticipant}{\texttt{delete\ /api/projects/\{projectId\}/memberships/\{guestId\}}}
\item
  \protect\hyperlink{getParticipants}{\texttt{get\ /api/projects/\{projectId\}/memberships}}
\end{itemize}

\hypertarget{onetimejobs}{%
\paragraph{\texorpdfstring{\protect\hyperlink{OnetimeJobs}{OnetimeJobs}}{OnetimeJobs}}\label{onetimejobs}}

\begin{itemize}
\tightlist
\item
  \protect\hyperlink{cancel}{\texttt{put\ /api/projects/\{projectId\}/jobs/\{jobScrapydId\}/\{jobId\}}}
\item
  \protect\hyperlink{deleteJob}{\texttt{delete\ /api/projects/\{projectId\}/jobs/\{jobScrapydId\}/\{jobId\}}}
\item
  \protect\hyperlink{getJobsExecutions}{\texttt{get\ /api/projects/jobs/\{limit\}/\{status\}}}
\item
  \protect\hyperlink{schedule}{\texttt{post\ /api/projects/\{projectId\}/jobs}}
\end{itemize}

\hypertarget{periodicjobs}{%
\paragraph{\texorpdfstring{\protect\hyperlink{PeriodicJobs}{PeriodicJobs}}{PeriodicJobs}}\label{periodicjobs}}

\begin{itemize}
\tightlist
\item
  \protect\hyperlink{addPeriodicJob}{\texttt{post\ /api/projects/\{projectId\}/periodicJobs}}
\item
  \protect\hyperlink{changePeriodicJob}{\texttt{put\ /api/projects/\{projectId\}/periodicJobs/\{periodicJobId\}}}
\item
  \protect\hyperlink{deletePeriodicJob}{\texttt{delete\ /api/projects/\{projectId\}/periodicJobs/\{periodicJobId\}}}
\item
  \protect\hyperlink{disable}{\texttt{put\ /api/projects/\{projectId\}/periodicJobs/\{periodicJobId\}/disable}}
\item
  \protect\hyperlink{enable}{\texttt{put\ /api/projects/\{projectId\}/periodicJobs/\{periodicJobId\}/enable}}
\item
  \protect\hyperlink{getPeriodicJobs}{\texttt{get\ /api/projects/\{projectId\}/periodicJobs/\{limit\}}}
\end{itemize}

\hypertarget{projects}{%
\paragraph{\texorpdfstring{\protect\hyperlink{Projects}{Projects}}{Projects}}\label{projects}}

\begin{itemize}
\tightlist
\item
  \protect\hyperlink{createProject}{\texttt{post\ /api/projects}}
\item
  \protect\hyperlink{deleteProject}{\texttt{delete\ /api/projects/\{projectId\}}}
\item
  \protect\hyperlink{deployProject}{\texttt{put\ /api/projects/\{projectId\}/deploy}}
\item
  \protect\hyperlink{getProjects}{\texttt{get\ /api/projects/\{limit\}}}
\item
  \protect\hyperlink{updateProjectMetadata}{\texttt{put\ /api/projects/\{projectId\}}}
\end{itemize}

\hypertarget{registration}{%
\paragraph{\texorpdfstring{\protect\hyperlink{Registration}{Registration}}{Registration}}\label{registration}}

\begin{itemize}
\tightlist
\item
  \protect\hyperlink{signUp}{\texttt{post\ /api/auth/signup}}
\end{itemize}
}
	
	\subsubsection{Authorization}
	
\protect\hypertarget{signIn}{}{}



\begin{verbatim}
post /api/auth/signin
\end{verbatim}

Get authentication token ({signIn})

\hypertarget{request-body-1}{%
\subsubsection*{Request body}\label{request-body-1}}

body \protect\hyperlink{SignIn}{SignIn} (required)

{Body Parameter} --- Credentials

\hypertarget{return-type-2}{%
\subsubsection*{Return type}\label{return-type-2}}

\protect\hyperlink{Cookie}{Cookie}

\hypertarget{example-data-2}{%
\subsubsection*{Example data}\label{example-data-2}}

Content-Type: application/json

\begin{verbatim}
{
  "cookie" : "cookie",
  "login" : "login"
}
\end{verbatim}

\hypertarget{responses-2}{%
\subsubsection*{Responses}\label{responses-2}}

\hypertarget{section-9}{%
\paragraph{200}\label{section-9}}

successful operation \protect\hyperlink{Cookie}{Cookie}

\hypertarget{section-10}{%
\paragraph{400}\label{section-10}}

SignInBadRequest \protect\hyperlink{}{}

\hypertarget{section-11}{%
\paragraph{403}\label{section-11}}

InvalidCredentialsProvided \protect\hyperlink{}{}

\begin{center}\rule{0.5\linewidth}{0.5pt}\end{center}


\protect\hypertarget{logout}{}{}



\begin{verbatim}
get /api/auth/logout
\end{verbatim}

Logout ({logout})

\hypertarget{return-type-3}{%
\subsubsection*{Return type}\label{return-type-3}}

\protect\hyperlink{ActionAnyContent}{ActionAnyContent}

\hypertarget{example-data-3}{%
\subsubsection*{Example data}\label{example-data-3}}

Content-Type: application/json

\begin{verbatim}
{ }
\end{verbatim}

\hypertarget{responses-3}{%
\subsubsection*{Responses}\label{responses-3}}

\hypertarget{section-12}{%
\paragraph{200}\label{section-12}}

successful operation
\protect\hyperlink{ActionAnyContent}{ActionAnyContent}

\hypertarget{section-13}{%
\paragraph{401}\label{section-13}}

Unauthorized \protect\hyperlink{}{}

\begin{center}\rule{0.5\linewidth}{0.5pt}\end{center}


	
	\subsubsection{Registration}
	\protect\hypertarget{signUp}{}{}


\begin{verbatim}
post /api/auth/signup
\end{verbatim}

Get authentication token ({signUp})

\hypertarget{request-body-7}{%
\subsubsection*{Request body}\label{request-body-7}}

body \protect\hyperlink{SignUp}{SignUp} (required)

{Body Parameter} --- Credentials

\hypertarget{return-type-22}{%
\subsubsection*{Return type}\label{return-type-22}}

\protect\hyperlink{Cookie}{Cookie}

\hypertarget{example-data-22}{%
\subsubsection*{Example data}\label{example-data-22}}

Content-Type: application/json

\begin{verbatim}
{
  "cookie" : "cookie",
  "login" : "login"
}
\end{verbatim}

\hypertarget{responses-22}{%
\subsubsection*{Responses}\label{responses-22}}

\hypertarget{section-77}{%
\paragraph{200}\label{section-77}}

successful operation \protect\hyperlink{Cookie}{Cookie}

\hypertarget{section-78}{%
\paragraph{400}\label{section-78}}

SignUp body bad request \protect\hyperlink{}{}

\hypertarget{section-79}{%
\paragraph{403}\label{section-79}}

EmailWrongFormat \protect\hyperlink{}{}

\hypertarget{section-80}{%
\paragraph{409}\label{section-80}}

UserAlreadyExistsMessage \protect\hyperlink{}{}

\begin{center}\rule{0.5\linewidth}{0.5pt}\end{center}
	
	\subsubsection{Project}
	

\protect\hypertarget{createProject}{}{}



\begin{verbatim}
post /api/projects
\end{verbatim}

Create project ({createProject})

\hypertarget{request-body-5}{%
\subsubsection*{Request body}\label{request-body-5}}

body \protect\hyperlink{ProjectForm}{ProjectForm} (required)

{Body Parameter} --- Form with initial project data

\hypertarget{return-type-17}{%
\subsubsection*{Return type}\label{return-type-17}}

UUID

\hypertarget{example-data-17}{%
\subsubsection*{Example data}\label{example-data-17}}

Content-Type: application/json

\begin{verbatim}
"046b6c7f-0b8a-43b9-b35d-6489e6daee91"
\end{verbatim}

\hypertarget{responses-17}{%
\subsubsection*{Responses}\label{responses-17}}

\hypertarget{section-59}{%
\paragraph{200}\label{section-59}}

successful operation \protect\hyperlink{UUID}{UUID}

\hypertarget{section-60}{%
\paragraph{400}\label{section-60}}

ProjectFormBadRequest \protect\hyperlink{}{}

\hypertarget{section-61}{%
\paragraph{401}\label{section-61}}

Unauthorized \protect\hyperlink{}{}

\hypertarget{section-62}{%
\paragraph{500}\label{section-62}}

Couldn't create projects \protect\hyperlink{}{}

\begin{center}\rule{0.5\linewidth}{0.5pt}\end{center}

\protect\hypertarget{deleteProject}{}{}



\begin{verbatim}
delete /api/projects/{projectId}
\end{verbatim}

Delete project ({deleteProject})

We can't delete project from our DB in case we encounter error deleting
it from scrapyd

\hypertarget{path-parameters-15}{%
\subsubsection*{Path parameters}\label{path-parameters-15}}

projectId (required)

{Path Parameter} --- format: int64

\hypertarget{return-type-18}{%
\subsubsection*{Return type}\label{return-type-18}}

\protect\hyperlink{ActionAnyContent}{ActionAnyContent}

\hypertarget{example-data-18}{%
\subsubsection*{Example data}\label{example-data-18}}

Content-Type: application/json

\begin{verbatim}
{ }
\end{verbatim}

\hypertarget{responses-18}{%
\subsubsection*{Responses}\label{responses-18}}

\hypertarget{section-63}{%
\paragraph{200}\label{section-63}}

successful operation
\protect\hyperlink{ActionAnyContent}{ActionAnyContent}

\hypertarget{section-64}{%
\paragraph{401}\label{section-64}}

Unauthorized \protect\hyperlink{}{}

\hypertarget{section-65}{%
\paragraph{403}\label{section-65}}

Do not have permission to delete project \protect\hyperlink{}{}

\hypertarget{section-66}{%
\paragraph{422}\label{section-66}}

Error deleting project from scrapyd \protect\hyperlink{}{}

\begin{center}\rule{0.5\linewidth}{0.5pt}\end{center}

\protect\hypertarget{deployProject}{}{}



\begin{verbatim}
put /api/projects/{projectId}/deploy
\end{verbatim}

Deploy project's eggfile to scrapyd ({deployProject})

\hypertarget{path-parameters-16}{%
\subsubsection*{Path parameters}\label{path-parameters-16}}

projectId (required)

{Path Parameter} --- format: int64

\hypertarget{form-parameters}{%
\subsubsection*{Form parameters}\label{form-parameters}}

eggFile (required)

{Form Parameter} ---

\hypertarget{return-type-19}{%
\subsubsection*{Return type}\label{return-type-19}}

array{[}\protect\hyperlink{Crawler}{Crawler}{]}

\hypertarget{example-data-19}{%
\subsubsection*{Example data}\label{example-data-19}}

Content-Type: application/json

\begin{verbatim}
[ {
  "settings" : { },
  "name" : "name",
  "id" : 0,
  "projectId" : 6
}, {
  "settings" : { },
  "name" : "name",
  "id" : 0,
  "projectId" : 6
} ]
\end{verbatim}

\hypertarget{responses-19}{%
\subsubsection*{Responses}\label{responses-19}}

\hypertarget{section-67}{%
\paragraph{200}\label{section-67}}

successful operation

\hypertarget{section-68}{%
\paragraph{401}\label{section-68}}

Unauthorized \protect\hyperlink{}{}

\hypertarget{section-69}{%
\paragraph{403}\label{section-69}}

Do not have permission to deploy project \protect\hyperlink{}{}

\hypertarget{section-70}{%
\paragraph{422}\label{section-70}}

Error getting eggfile from multipart form/data or error deploy to
scrapyd \protect\hyperlink{}{}

\begin{center}\rule{0.5\linewidth}{0.5pt}\end{center}

\protect\hypertarget{getProjects}{}{}



\begin{verbatim}
get /api/projects/{limit}
\end{verbatim}

Get projects ({getProjects})

\hypertarget{path-parameters-17}{%
\subsubsection*{Path parameters}\label{path-parameters-17}}

limit (required)

{Path Parameter} --- Limit for request format: int32

\hypertarget{query-parameters-3}{%
\subsubsection*{Query parameters}\label{query-parameters-3}}

id (optional)

{Query Parameter} --- ID excludeFrom format: int64

\hypertarget{return-type-20}{%
\subsubsection*{Return type}\label{return-type-20}}

array{[}\protect\hyperlink{Project}{Project}{]}

\hypertarget{example-data-20}{%
\subsubsection*{Example data}\label{example-data-20}}

Content-Type: application/json

\begin{verbatim}
[ {
  "createdAt" : 6,
  "eggfile" : [ "eggfile", "eggfile" ],
  "spidersSettings" : { },
  "changedBy" : "046b6c7f-0b8a-43b9-b35d-6489e6daee91",
  "name" : "name",
  "description" : "description",
  "changedAt" : 1,
  "id" : 0,
  "ownerId" : "046b6c7f-0b8a-43b9-b35d-6489e6daee91"
}, {
  "createdAt" : 6,
  "eggfile" : [ "eggfile", "eggfile" ],
  "spidersSettings" : { },
  "changedBy" : "046b6c7f-0b8a-43b9-b35d-6489e6daee91",
  "name" : "name",
  "description" : "description",
  "changedAt" : 1,
  "id" : 0,
  "ownerId" : "046b6c7f-0b8a-43b9-b35d-6489e6daee91"
} ]
\end{verbatim}

\hypertarget{responses-20}{%
\subsubsection*{Responses}\label{responses-20}}

\hypertarget{section-71}{%
\paragraph{200}\label{section-71}}

successful operation

\hypertarget{section-72}{%
\paragraph{401}\label{section-72}}

Unauthorized \protect\hyperlink{}{}

\hypertarget{section-73}{%
\paragraph{500}\label{section-73}}

Couldn't get projects \protect\hyperlink{}{}

\begin{center}\rule{0.5\linewidth}{0.5pt}\end{center}

\protect\hypertarget{updateProjectMetadata}{}{}



\begin{verbatim}
put /api/projects/{projectId}
\end{verbatim}

Change project metadata ({updateProjectMetadata})

\hypertarget{path-parameters-18}{%
\subsubsection*{Path parameters}\label{path-parameters-18}}

projectId (required)

{Path Parameter} --- format: int64

\hypertarget{request-body-6}{%
\subsubsection*{Request body}\label{request-body-6}}

body \protect\hyperlink{ProjectChangeForm}{ProjectChangeForm} (required)

{Body Parameter} --- Form with metadata to be changed

\hypertarget{return-type-21}{%
\subsubsection*{Return type}\label{return-type-21}}

\protect\hyperlink{ActionProjectChangeForm}{ActionProjectChangeForm}

\hypertarget{example-data-21}{%
\subsubsection*{Example data}\label{example-data-21}}

Content-Type: application/json

\begin{verbatim}
{ }
\end{verbatim}

\hypertarget{responses-21}{%
\subsubsection*{Responses}\label{responses-21}}

\hypertarget{section-74}{%
\paragraph{200}\label{section-74}}

successful operation
\protect\hyperlink{ActionProjectChangeForm}{ActionProjectChangeForm}

\hypertarget{section-75}{%
\paragraph{401}\label{section-75}}

Unauthorized \protect\hyperlink{}{}

\hypertarget{section-76}{%
\paragraph{403}\label{section-76}}

Do not have permission to change project \protect\hyperlink{}{}

\begin{center}\rule{0.5\linewidth}{0.5pt}\end{center}

	
	\subsubsection{Membership}
	
\protect\hypertarget{addParticipants}{}{}

\begin{verbatim}
put /api/projects/{projectId}/memberships/{guestId}/{guestAccess}
\end{verbatim}

Add or change participant of project ({addParticipants})

Insert or update

\hypertarget{path-parameters-2}{%
\subsubsection*{Path parameters}\label{path-parameters-2}}

projectId (required)

{Path Parameter} --- format: int64

guestId (required)

{Path Parameter} --- format: uuid

guestAccess (required)

{Path Parameter} ---

\hypertarget{return-type-4}{%
\subsubsection*{Return type}\label{return-type-4}}

\protect\hyperlink{ActionAnyContent}{ActionAnyContent}

\hypertarget{example-data-4}{%
\subsubsection*{Example data}\label{example-data-4}}

Content-Type: application/json

\begin{verbatim}
{ }
\end{verbatim}

\hypertarget{responses-4}{%
\subsubsection*{Responses}\label{responses-4}}

\hypertarget{section-14}{%
\paragraph{200}\label{section-14}}

successful operation
\protect\hyperlink{ActionAnyContent}{ActionAnyContent}

\hypertarget{section-15}{%
\paragraph{401}\label{section-15}}

Unauthorized \protect\hyperlink{}{}

\hypertarget{section-16}{%
\paragraph{403}\label{section-16}}

Dont have permission to specified project \protect\hyperlink{}{}

\begin{center}\rule{0.5\linewidth}{0.5pt}\end{center}

\protect\hypertarget{deleteParticipant}{}{}



\begin{verbatim}
delete /api/projects/{projectId}/memberships/{guestId}
\end{verbatim}

Delete user from membership list ({deleteParticipant})

Only owner can delete from membership list

\hypertarget{path-parameters-3}{%
\subsubsection*{Path parameters}\label{path-parameters-3}}

projectId (required)

{Path Parameter} --- format: int64

guestId (required)

{Path Parameter} --- format: uuid

\hypertarget{return-type-5}{%
\subsubsection*{Return type}\label{return-type-5}}

\protect\hyperlink{ActionAnyContent}{ActionAnyContent}

\hypertarget{example-data-5}{%
\subsubsection*{Example data}\label{example-data-5}}

Content-Type: application/json

\begin{verbatim}
{ }
\end{verbatim}

\hypertarget{responses-5}{%
\subsubsection*{Responses}\label{responses-5}}

\hypertarget{section-17}{%
\paragraph{200}\label{section-17}}

successful operation
\protect\hyperlink{ActionAnyContent}{ActionAnyContent}

\hypertarget{section-18}{%
\paragraph{401}\label{section-18}}

Unauthorized \protect\hyperlink{}{}

\hypertarget{section-19}{%
\paragraph{403}\label{section-19}}

Do not have permission to delete member \protect\hyperlink{}{}

\begin{center}\rule{0.5\linewidth}{0.5pt}\end{center}

\protect\hypertarget{getParticipants}{}{}



\begin{verbatim}
get /api/projects/{projectId}/memberships
\end{verbatim}

Get list of members for project ({getParticipants})

Not paginated

\hypertarget{path-parameters-4}{%
\subsubsection*{Path parameters}\label{path-parameters-4}}

projectId (required)

{Path Parameter} --- format: int64

\hypertarget{return-type-6}{%
\subsubsection*{Return type}\label{return-type-6}}

array{[}\protect\hyperlink{Member}{Member}{]}

\hypertarget{example-data-6}{%
\subsubsection*{Example data}\label{example-data-6}}

Content-Type: application/json

\begin{verbatim}
[ {
  "accessRight" : { },
  "user" : {
    "name" : "name",
    "id" : "046b6c7f-0b8a-43b9-b35d-6489e6daee91",
    "login" : "login",
    "email" : "email"
  }
}, {
  "accessRight" : { },
  "user" : {
    "name" : "name",
    "id" : "046b6c7f-0b8a-43b9-b35d-6489e6daee91",
    "login" : "login",
    "email" : "email"
  }
} ]
\end{verbatim}

\hypertarget{responses-6}{%
\subsubsection*{Responses}\label{responses-6}}

\hypertarget{section-20}{%
\paragraph{200}\label{section-20}}

successful operation

\hypertarget{section-21}{%
\paragraph{401}\label{section-21}}

Unauthorized \protect\hyperlink{}{}

\hypertarget{section-22}{%
\paragraph{403}\label{section-22}}

Dont have permission to specified project \protect\hyperlink{}{}

\hypertarget{section-23}{%
\paragraph{500}\label{section-23}}

Couldn't get list of members \protect\hyperlink{}{}

\begin{center}\rule{0.5\linewidth}{0.5pt}\end{center}

	
	\subsubsection{Crawlers}
	

\protect\hypertarget{listSpiders}{}{}



\begin{verbatim}
get /api/projects/{projectId}/crawlers
\end{verbatim}

List spiders ({listSpiders})

List spiders of project without pagination

\hypertarget{path-parameters}{%
\subsubsection*{Path parameters}\label{path-parameters}}

projectId (required)

{Path Parameter} --- Project ID format: int64

\hypertarget{query-parameters}{%
\subsubsection*{Query parameters}\label{query-parameters}}

version (optional)

{Query Parameter} --- Version of the project

\hypertarget{return-type}{%
\subsubsection*{Return type}\label{return-type}}

array{[}\protect\hyperlink{Crawler}{Crawler}{]}

\hypertarget{example-data}{%
\subsubsection*{Example data}\label{example-data}}

Content-Type: application/json

\begin{verbatim}
[ {
  "settings" : { },
  "name" : "name",
  "id" : 0,
  "projectId" : 6
}, {
  "settings" : { },
  "name" : "name",
  "id" : 0,
  "projectId" : 6
} ]
\end{verbatim}

\hypertarget{responses}{%
\subsubsection*{Responses}\label{responses}}

\hypertarget{section-1}{%
\paragraph{200}\label{section-1}}

successful operation

\hypertarget{section-2}{%
\paragraph{401}\label{section-2}}

Unauthorized \protect\hyperlink{}{}

\hypertarget{section-3}{%
\paragraph{403}\label{section-3}}

Couldn't get project's spiders due to access rights permission
\protect\hyperlink{}{}

\hypertarget{section-4}{%
\paragraph{422}\label{section-4}}

Coulnd't get spiders from DB \protect\hyperlink{}{}

\begin{center}\rule{0.5\linewidth}{0.5pt}\end{center}

\protect\hypertarget{updateSpider}{}{}



\begin{verbatim}
put /api/projects/{projectId}/crawlers/{crawlerId}
\end{verbatim}

Update spider's settings ({updateSpider})

Updates only spider's settings. \texttt{projectId} should match
\texttt{spiderId}

\hypertarget{path-parameters-1}{%
\subsubsection*{Path parameters}\label{path-parameters-1}}

projectId (required)

{Path Parameter} --- format: int64

crawlerId (required)

{Path Parameter} --- format: int64

\hypertarget{request-body}{%
\subsubsection*{Request body}\label{request-body}}

body \protect\hyperlink{SpiderChangeForm}{SpiderChangeForm} (required)

{Body Parameter} --- Form with new settings

\hypertarget{return-type-1}{%
\subsubsection*{Return type}\label{return-type-1}}

\protect\hyperlink{ActionSpiderChangeForm}{ActionSpiderChangeForm}

\hypertarget{example-data-1}{%
\subsubsection*{Example data}\label{example-data-1}}

Content-Type: application/json

\begin{verbatim}
{ }
\end{verbatim}

\hypertarget{responses-1}{%
\subsubsection*{Responses}\label{responses-1}}

\hypertarget{section-5}{%
\paragraph{200}\label{section-5}}

successful operation
\protect\hyperlink{ActionSpiderChangeForm}{ActionSpiderChangeForm}

\hypertarget{section-6}{%
\paragraph{400}\label{section-6}}

Bad format SpiderChangeForm \protect\hyperlink{}{}

\hypertarget{section-7}{%
\paragraph{401}\label{section-7}}

Unauthorized \protect\hyperlink{}{}

\hypertarget{section-8}{%
\paragraph{403}\label{section-8}}

Can't change spider's data due to access right permission
\protect\hyperlink{}{}

\begin{center}\rule{0.5\linewidth}{0.5pt}\end{center}

	
	\subsubsection{Onetime jobs}
	\protect\hypertarget{cancel}{}{}



\begin{verbatim}
put /api/projects/{projectId}/jobs/{jobScrapydId}/{jobId}
\end{verbatim}

Cancel running and pending tasks ({cancel})

It moves both of the statuses to finished

\hypertarget{path-parameters-5}{%
\subsubsection*{Path parameters}\label{path-parameters-5}}

projectId (required)

{Path Parameter} --- format: int64

jobScrapydId (required)

{Path Parameter} --- format: uuid

jobId (required)

{Path Parameter} --- format: int64

\hypertarget{return-type-7}{%
\subsubsection*{Return type}\label{return-type-7}}

UUID

\hypertarget{example-data-7}{%
\subsubsection*{Example data}\label{example-data-7}}

Content-Type: application/json

\begin{verbatim}
"046b6c7f-0b8a-43b9-b35d-6489e6daee91"
\end{verbatim}

\hypertarget{responses-7}{%
\subsubsection*{Responses}\label{responses-7}}

\hypertarget{section-24}{%
\paragraph{200}\label{section-24}}

successful operation \protect\hyperlink{UUID}{UUID}

\hypertarget{section-25}{%
\paragraph{401}\label{section-25}}

Unauthorized \protect\hyperlink{}{}

\hypertarget{section-26}{%
\paragraph{403}\label{section-26}}

NoAccess \protect\hyperlink{}{}

\hypertarget{section-27}{%
\paragraph{422}\label{section-27}}

WrongStatus \protect\hyperlink{}{}

\begin{center}\rule{0.5\linewidth}{0.5pt}\end{center}

\protect\hypertarget{deleteJob}{}{}



\begin{verbatim}
delete /api/projects/{projectId}/jobs/{jobScrapydId}/{jobId}
\end{verbatim}

Deletes finished job execution instance ({deleteJob})

It removes all the information from DB

\hypertarget{path-parameters-6}{%
\subsubsection*{Path parameters}\label{path-parameters-6}}

projectId (required)

{Path Parameter} --- format: int64

jobScrapydId (required)

{Path Parameter} --- format: uuid

jobId (required)

{Path Parameter} --- format: int64

\hypertarget{return-type-8}{%
\subsubsection*{Return type}\label{return-type-8}}

UUID

\hypertarget{example-data-8}{%
\subsubsection*{Example data}\label{example-data-8}}

Content-Type: application/json

\begin{verbatim}
"046b6c7f-0b8a-43b9-b35d-6489e6daee91"
\end{verbatim}

\hypertarget{responses-8}{%
\subsubsection*{Responses}\label{responses-8}}

\hypertarget{section-28}{%
\paragraph{200}\label{section-28}}

successful operation \protect\hyperlink{UUID}{UUID}

\hypertarget{section-29}{%
\paragraph{401}\label{section-29}}

Unauthorized \protect\hyperlink{}{}

\hypertarget{section-30}{%
\paragraph{403}\label{section-30}}

User doesn't have at least ReadAndWrite access \protect\hyperlink{}{}

\hypertarget{section-31}{%
\paragraph{422}\label{section-31}}

Couldn't delete job execution \protect\hyperlink{}{}

\begin{center}\rule{0.5\linewidth}{0.5pt}\end{center}

\protect\hypertarget{getJobsExecutions}{}{}



\begin{verbatim}
get /api/projects/jobs/{limit}/{status}
\end{verbatim}

Get list of job executions with pagination ({getJobsExecutions})

\begin{verbatim}
    With pagination. Get all of the current jobs for all user's project.
\end{verbatim}

\hypertarget{path-parameters-7}{%
\subsubsection*{Path parameters}\label{path-parameters-7}}

limit (required)

{Path Parameter} --- Limit for request format: int32

status (required)

{Path Parameter} --- Status of job

\hypertarget{query-parameters-1}{%
\subsubsection*{Query parameters}\label{query-parameters-1}}

fromId (optional)

{Query Parameter} --- ID excludeFrom format: int64

\hypertarget{return-type-9}{%
\subsubsection*{Return type}\label{return-type-9}}

array{[}\protect\hyperlink{Job}{Job}{]}

\hypertarget{example-data-9}{%
\subsubsection*{Example data}\label{example-data-9}}

Content-Type: application/json

\begin{verbatim}
[ {
  "jobInstanceId" : 6,
  "scrapydId" : "046b6c7f-0b8a-43b9-b35d-6489e6daee91",
  "project" : {
    "name" : "name",
    "id" : 2
  },
  "startTime" : 1,
  "id" : 0,
  "endTime" : 5,
  "spider" : {
    "name" : "name",
    "id" : 5
  },
  "status" : { }
}, {
  "jobInstanceId" : 6,
  "scrapydId" : "046b6c7f-0b8a-43b9-b35d-6489e6daee91",
  "project" : {
    "name" : "name",
    "id" : 2
  },
  "startTime" : 1,
  "id" : 0,
  "endTime" : 5,
  "spider" : {
    "name" : "name",
    "id" : 5
  },
  "status" : { }
} ]
\end{verbatim}

\hypertarget{responses-9}{%
\subsubsection*{Responses}\label{responses-9}}

\hypertarget{section-32}{%
\paragraph{200}\label{section-32}}

successful operation

\hypertarget{section-33}{%
\paragraph{401}\label{section-33}}

Unauthorized \protect\hyperlink{}{}

\hypertarget{section-34}{%
\paragraph{500}\label{section-34}}

Couldn't get jobs \protect\hyperlink{}{}

\begin{center}\rule{0.5\linewidth}{0.5pt}\end{center}

\protect\hypertarget{schedule}{}{}



\begin{verbatim}
post /api/projects/{projectId}/jobs
\end{verbatim}

Schedule onetime job ({schedule})

User has to have \texttt{ReadAndWrite} access to project. Initial status
of the job = \texttt{pending}.Creates and starts new job with chosen
crawler

\hypertarget{path-parameters-8}{%
\subsubsection*{Path parameters}\label{path-parameters-8}}

projectId (required)

{Path Parameter} --- format: int64

\hypertarget{request-body-2}{%
\subsubsection*{Request body}\label{request-body-2}}

body \protect\hyperlink{OnetimeJobForm}{OnetimeJobForm} (required)

{Body Parameter} --- Form with settings of onetime job for scheduling

\hypertarget{return-type-10}{%
\subsubsection*{Return type}\label{return-type-10}}

\protect\hyperlink{ActionOnetimeJobForm}{ActionOnetimeJobForm}

\hypertarget{example-data-10}{%
\subsubsection*{Example data}\label{example-data-10}}

Content-Type: application/json

\begin{verbatim}
{ }
\end{verbatim}

\hypertarget{responses-10}{%
\subsubsection*{Responses}\label{responses-10}}

\hypertarget{section-35}{%
\paragraph{200}\label{section-35}}

successful operation
\protect\hyperlink{ActionOnetimeJobForm}{ActionOnetimeJobForm}

\hypertarget{section-36}{%
\paragraph{400}\label{section-36}}

Bad format OnetimeJobForm \protect\hyperlink{}{}

\hypertarget{section-37}{%
\paragraph{401}\label{section-37}}

Unauthorized \protect\hyperlink{}{}

\hypertarget{section-38}{%
\paragraph{403}\label{section-38}}

Can't schedule job due to access right permission \protect\hyperlink{}{}

\hypertarget{section-39}{%
\paragraph{409}\label{section-39}}

Couldn't schedule job \protect\hyperlink{}{}

\begin{center}\rule{0.5\linewidth}{0.5pt}\end{center}

	
	\subsubsection{Periodic Jobs}
	\protect\hypertarget{addPeriodicJob}{}{}



\begin{verbatim}
post /api/projects/{projectId}/periodicJobs
\end{verbatim}

Creates periodic job (jobInstance) ({addPeriodicJob})

Creates JonInstance in DB. Schedules jobs according to specified
\texttt{cron\ expression}. Checks for user access rights.

\hypertarget{path-parameters-9}{%
\subsubsection*{Path parameters}\label{path-parameters-9}}

projectId (required)

{Path Parameter} --- format: int64

\hypertarget{request-body-3}{%
\subsubsection*{Request body}\label{request-body-3}}

body \protect\hyperlink{PeriodicJobCreateForm}{PeriodicJobCreateForm}
(required)

{Body Parameter} --- Form to create periodic job

\hypertarget{return-type-11}{%
\subsubsection*{Return type}\label{return-type-11}}

UUID

\hypertarget{example-data-11}{%
\subsubsection*{Example data}\label{example-data-11}}

Content-Type: application/json

\begin{verbatim}
"046b6c7f-0b8a-43b9-b35d-6489e6daee91"
\end{verbatim}

\hypertarget{responses-11}{%
\subsubsection*{Responses}\label{responses-11}}

\hypertarget{section-40}{%
\paragraph{200}\label{section-40}}

successful operation \protect\hyperlink{UUID}{UUID}

\hypertarget{section-41}{%
\paragraph{401}\label{section-41}}

Unauthorized \protect\hyperlink{}{}

\hypertarget{section-42}{%
\paragraph{403}\label{section-42}}

Dont have at least write access to specified project
\protect\hyperlink{}{}

\hypertarget{section-43}{%
\paragraph{422}\label{section-43}}

Couldn't create job instance \protect\hyperlink{}{}

\begin{center}\rule{0.5\linewidth}{0.5pt}\end{center}

\protect\hypertarget{changePeriodicJob}{}{}



\begin{verbatim}
put /api/projects/{projectId}/periodicJobs/{periodicJobId}
\end{verbatim}

Changes the periodic Job data ({changePeriodicJob})

Checks for user access rights and job-project connection.

\hypertarget{path-parameters-10}{%
\subsubsection*{Path parameters}\label{path-parameters-10}}

projectId (required)

{Path Parameter} --- format: int64

periodicJobId (required)

{Path Parameter} --- format: int64

\hypertarget{request-body-4}{%
\subsubsection*{Request body}\label{request-body-4}}

body \protect\hyperlink{PeriodicJobChangeForm}{PeriodicJobChangeForm}
(required)

{Body Parameter} --- Form to change periodic job data

\hypertarget{return-type-12}{%
\subsubsection*{Return type}\label{return-type-12}}

\protect\hyperlink{ActionPeriodicJobChangeForm}{ActionPeriodicJobChangeForm}

\hypertarget{example-data-12}{%
\subsubsection*{Example data}\label{example-data-12}}

Content-Type: application/json

\begin{verbatim}
{ }
\end{verbatim}

\hypertarget{responses-12}{%
\subsubsection*{Responses}\label{responses-12}}

\hypertarget{section-44}{%
\paragraph{200}\label{section-44}}

successful operation
\protect\hyperlink{ActionPeriodicJobChangeForm}{ActionPeriodicJobChangeForm}

\hypertarget{section-45}{%
\paragraph{401}\label{section-45}}

Unauthorized \protect\hyperlink{}{}

\hypertarget{section-46}{%
\paragraph{403}\label{section-46}}

Dont have at least write access to specified project or job-project
don't correspond \protect\hyperlink{}{}

\hypertarget{section-47}{%
\paragraph{422}\label{section-47}}

Couldn't change job instance \protect\hyperlink{}{}

\hypertarget{section-48}{%
\paragraph{500}\label{section-48}}

Error performing the update in DB \protect\hyperlink{}{}

\begin{center}\rule{0.5\linewidth}{0.5pt}\end{center}

\protect\hypertarget{deletePeriodicJob}{}{}



\begin{verbatim}
delete /api/projects/{projectId}/periodicJobs/{periodicJobId}
\end{verbatim}

Delete periodic job ({deletePeriodicJob})

Deletes periodic job instance (changed type to Onetime) and cancels all
of the future job scheduled.

\hypertarget{path-parameters-11}{%
\subsubsection*{Path parameters}\label{path-parameters-11}}

projectId (required)

{Path Parameter} --- format: int64

periodicJobId (required)

{Path Parameter} --- format: int64

\hypertarget{return-type-13}{%
\subsubsection*{Return type}\label{return-type-13}}

\protect\hyperlink{ActionAnyContent}{ActionAnyContent}

\hypertarget{example-data-13}{%
\subsubsection*{Example data}\label{example-data-13}}

Content-Type: application/json

\begin{verbatim}
{ }
\end{verbatim}

\hypertarget{responses-13}{%
\subsubsection*{Responses}\label{responses-13}}

\hypertarget{section-49}{%
\paragraph{200}\label{section-49}}

successful operation
\protect\hyperlink{ActionAnyContent}{ActionAnyContent}

\hypertarget{section-50}{%
\paragraph{401}\label{section-50}}

Unauthorized \protect\hyperlink{}{}

\hypertarget{section-51}{%
\paragraph{403}\label{section-51}}

NoPermission \protect\hyperlink{}{}

\hypertarget{section-52}{%
\paragraph{422}\label{section-52}}

JobCouldn'tBeDeleted \protect\hyperlink{}{}

\begin{center}\rule{0.5\linewidth}{0.5pt}\end{center}

\protect\hypertarget{disable}{}{}



\begin{verbatim}
put /api/projects/{projectId}/periodicJobs/{periodicJobId}/disable
\end{verbatim}

Sets status of periodicJob to disabled. ({disable})

Cancels all of the future job scheduled. Does not modify running type,
only running status.

\hypertarget{path-parameters-12}{%
\subsubsection*{Path parameters}\label{path-parameters-12}}

projectId (required)

{Path Parameter} --- format: int64

periodicJobId (required)

{Path Parameter} --- format: int64

\hypertarget{return-type-14}{%
\subsubsection*{Return type}\label{return-type-14}}

\protect\hyperlink{ActionAnyContent}{ActionAnyContent}

\hypertarget{example-data-14}{%
\subsubsection*{Example data}\label{example-data-14}}

Content-Type: application/json

\begin{verbatim}
{ }
\end{verbatim}

\hypertarget{responses-14}{%
\subsubsection*{Responses}\label{responses-14}}

\hypertarget{section-53}{%
\paragraph{200}\label{section-53}}

successful operation
\protect\hyperlink{ActionAnyContent}{ActionAnyContent}

\begin{center}\rule{0.5\linewidth}{0.5pt}\end{center}

\protect\hypertarget{enable}{}{}



\begin{verbatim}
put /api/projects/{projectId}/periodicJobs/{periodicJobId}/enable
\end{verbatim}

Enable scheduling jobs. ({enable})

Continues to schedule job executions. Does not modify running type, only
running status.

\hypertarget{path-parameters-13}{%
\subsubsection*{Path parameters}\label{path-parameters-13}}

projectId (required)

{Path Parameter} --- format: int64

periodicJobId (required)

{Path Parameter} --- format: int64

\hypertarget{return-type-15}{%
\subsubsection*{Return type}\label{return-type-15}}

\protect\hyperlink{ActionAnyContent}{ActionAnyContent}

\hypertarget{example-data-15}{%
\subsubsection*{Example data}\label{example-data-15}}

Content-Type: application/json

\begin{verbatim}
{ }
\end{verbatim}

\hypertarget{responses-15}{%
\subsubsection*{Responses}\label{responses-15}}

\hypertarget{section-54}{%
\paragraph{200}\label{section-54}}

successful operation
\protect\hyperlink{ActionAnyContent}{ActionAnyContent}

\begin{center}\rule{0.5\linewidth}{0.5pt}\end{center}

\protect\hypertarget{getPeriodicJobs}{}{}



\begin{verbatim}
get /api/projects/{projectId}/periodicJobs/{limit}
\end{verbatim}

Get list of periodic jobs with pagination ({getPeriodicJobs})

Gets data from DB. No requests to scrapyd needed. User has to have
access (at least \texttt{readonly}) to requested project

\hypertarget{path-parameters-14}{%
\subsubsection*{Path parameters}\label{path-parameters-14}}

projectId (required)

{Path Parameter} --- format: int64

limit (required)

{Path Parameter} --- format: int32

\hypertarget{query-parameters-2}{%
\subsubsection*{Query parameters}\label{query-parameters-2}}

exclusiveFrom (optional)

{Query Parameter} --- format: int64

\hypertarget{return-type-16}{%
\subsubsection*{Return type}\label{return-type-16}}

array{[}\protect\hyperlink{JobInstance}{JobInstance}{]}

\hypertarget{example-data-16}{%
\subsubsection*{Example data}\label{example-data-16}}

Content-Type: application/json

\begin{verbatim}
[ {
  "cron" : "cron",
  "settings" : { },
  "description" : "description",
  "id" : 0,
  "title" : "title",
  "priority" : { },
  "projectId" : 6,
  "spider" : 1
}, {
  "cron" : "cron",
  "settings" : { },
  "description" : "description",
  "id" : 0,
  "title" : "title",
  "priority" : { },
  "projectId" : 6,
  "spider" : 1
} ]
\end{verbatim}

\hypertarget{responses-16}{%
\subsubsection*{Responses}\label{responses-16}}

\hypertarget{section-55}{%
\paragraph{200}\label{section-55}}

successful operation

\hypertarget{section-56}{%
\paragraph{401}\label{section-56}}

Unauthorized \protect\hyperlink{}{}

\hypertarget{section-57}{%
\paragraph{403}\label{section-57}}

Dont have at least read access to specified project
\protect\hyperlink{}{}

\hypertarget{section-58}{%
\paragraph{500}\label{section-58}}

Couldn't get list of periodic jobs \protect\hyperlink{}{}

\begin{center}\rule{0.5\linewidth}{0.5pt}\end{center}

	
	
	
	\newpage
	
	\addition{Используемые понятия и определения}
	\begin{description}
		\item[\textbf{Web scraping}] -- это сбор данных с различных интернет-ресурсов. Общий принцип его работы можно объяснить следующим образом: некий автоматизированный код выполняет GET-запросы на целевой сайт и получая ответ, парсит HTML-документ, ищет данные и преобразует их в заданный формат. \label{terms:webscraping}
		\item[\textbf{Проект}] -- сущность для объединения и предоставления доступа к запускам/краулерам/периодическим задачам. \label{terms:project}
		
		\item[\textbf{Веб краулер}] --  программа, являющаяся составной частью поисковой системы и предназначенная для перебора страниц Интернета с целью занесения информации о них в базу данных поисковика. Неотъемлемая часть проекта. Именно с помощью пауков пользователь может “краулить” сайты для сбора необходимой информации. \label{terms:spider}
		\item[\textbf{Запуск}] -- единоразовый запуск краулера с настройками и аргументами, указанными для этого запуска. \label{terms:job}
		\item[\textbf{Периодический запуск}] -- запуск с множеством настроек, повторяющийся в определенные периоды времени (запуски по cron-expression).
		\label{terms:pjob}
	\end{description}
	
	\newpage
	\newpage
	%\section{Источники, использованные при разработке}
	\renewcommand{\refname}{Список источников}
	\addcontentsline{toc}{section}{\refname}
	\begin{thebibliography}{7}
		\bibitem{scrapyd} Github scrapyd/scrapyd [Электронный ресурс] URL: \url{https://github.com/scrapy/scrapyd} (Дата обращения: 16.04.2020, режим доступа: свободный)
		\bibitem{gost}Единая система программной документации – М.: ИПК, Издательство стандартов, 2000, 125 стр.
		\bibitem{postgresql} Postgresql [Электронный ресурс] URL:\url{https://www.postgresql.org} (Дата обращения: 16.04.2020, режим доступа: свободный)
		
		\bibitem{play} Play-framework [Электронный ресурс] URL:\url{https://www.playframework.com} (Дата обращения: 16.04.2020, режим доступа: свободный)
		
		\bibitem{slick} Slick [Электронный ресурс] URL: \url{https://scala-slick.org}(Дата обращения: 16.04.2020, режим доступа: свободный)
		
	\end{thebibliography}
						\newpage
	\listRegistration
	\addcontentsline{toc}{section}{Лист регистрации изменений}
\end{document}