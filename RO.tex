\documentclass[a4paper,12pt]{article}
\usepackage{styledoc19}


\begin{document} % конец преамбулы, начало документа
	
	\year{2020}
	\docNumber{RU.17701729.04.13-01 34 01-1}
	\docFormat{Руководство оператора}
	\student{БПИ 174}{Д. Ю. Редникина}
	\supervisor{Профессор департамента \vfill программной инженерии  факультета компьютерных наук, к.т.н}
	{Е. М. Гринкруг}
	\project{СИСТЕМА УПРАВЛЕНИЯ ЗАДАНИЯМИ ПО АВТОМАТИЧЕСКОМУ СБОРУ ДАННЫХ ИЗ СЕТИ ИНТЕРНЕТ}
	
	\firstPage
	\newpage
	\secondPage
	\newpage
	\thirdPage
	\newpage
	\section{Назначение программы}
	\subsection{Функциональное назначение}
	Система будет применяться как средство управления проектами по созданию, редактированию и запуску веб краулеров для сбора данных в сети интернет. Продукт позволит следить за запусками в режиме реального времени, а также создавать периодические запуски по расписанию.

	\subsection{Эксплуатационное назначение}
	Web-приложение является компонентом CRM-системы для благотворительного фонда <<AIAIN>>, позволяющей облегчить бизнес процессы работы фонда с благополучателями и донорами. Web-приложение призвано обеспечить необходимую сотрудникам фонда функциональность для работы с системой. Им будут пользоваться как менеджеры фонда, так и администраторы, члены комиссий, операторы фонда, контент-менеджеры фонда. Каждый из пользователей будет иметь доступ к необходимой ему функциональности по обработке заявок, управлению фондом, администрированию и т.д.
	\subsection{Состав выполняемых функций}
	Следующие требования зафиксированы в документе <<Cистема управления заданиями по автоматическому сбору данных из сети Интернет. Техническое задание>> к составу выполняемых функций:
	

\begin{enumerate}
		
		\item \textbf{Авторизация\\}
		Чтобы использовать сервис, клиентская программа должна иметь возможность авторизоваться в системе с помощью REST API
		\begin{enumerate}
			\item Для регистрации пользователю нужно указать следующие данные 
			\begin{enumerate}
			    \item Почта - уникальна для каждого зарегистрированного пользователя; 
			    \item Имя - длина больше 1 символ;
			    \item Логин - длина больше 2 символов;
			    \item Пароль - длина больше 2 символов;
			\end{enumerate}
			\item Для авторизации пользователя в системе должны быть указаны следующие данные
			\begin{enumerate}
			    \item Почта;
			    \item Пароль;
			\end{enumerate}
		\end{enumerate}
		
		\item \textbf{Проекты\\}
		Должны быть реализованы запросы REST API для предоставления клиенту следующей функциональности
		\begin{enumerate}
	    	\item Создание проекта со следующей информацией
	    	\begin{enumerate}
	    	    \item Имя проекта;
	    	    \item Описание проекта - опциональное поле;
	    	\end{enumerate}
			\item Обновление метаданных о проекте (редактирование) могут быть обновлены только участником с минимальным уровнем дотупа \texttt{ReadAndWrite}. Следующие данные могут быть обновлены:
			\begin{enumerate}
			    \item Имя проекта;
			    \item Описание проекта;
			    \item Настройки проекта для запуска краулеров;
			    \item Аргументы для запуска краулеров проекта;
			\end{enumerate}
			\item Обновление \texttt{egg} файла проекта (редактирование) -- минимальный уровень доступа участника, обновляющий данные о проекте \texttt{ReadAndWrite}.  
			\item Удаление данных о проекте. Удалить проект может только владелец \texttt{Owner}.
			\item Просмотр списка проектов (с пагинацией), к которым у пользователя есть как минимум \texttt{ReadOnly} доступ.
		\end{enumerate}
		
		\item \textbf{Участники проектов\\}
		Должны быть реализованы запросы REST API для предоставления клиенту следующей функциональности
		\begin{enumerate}
	    	\item Просмотр информации об участниках проекта;
	    	\begin{enumerate}
	    	    \item Имя, почта, логин участника;
	    	    \item Статус участника в проекте (\texttt{ReadOnly}, \texttt{ReadAndWrite} или \texttt{Owner});
	    	\end{enumerate}
			\item Обновление статуса участника проекта. Это действие совершать может только владелец проекта; 
			\item Удаление участника из проекта. Данное действие может совершать только владелец проекта;
			\item Добавление нового участника с указанными правами на редактирование. Данное действие может совершать только владелец проекта;
		\end{enumerate}
		
		\item \textbf{Краулеры\\}
		Должны быть реализованы запросы REST API для предоставления клиенту следующей функциональности
		\begin{enumerate}
			\item Просмотр списка краулеров проекта;
			\item Редактирование информации о краулере для последующих запусков. Следующая информация может быть изменена
			\begin{enumerate}
			    \item Настройки краулера для запуска;
			    \item Аргументы для запуска;
			\end{enumerate}
		\end{enumerate}
		
		\item \textbf{Запуски краулеров\\}
		Должны быть реализованы запросы REST API для предоставления клиенту следующей функциональности
		\begin{enumerate}
		    \item Просмотр списка запусков в определенном статусе (\texttt{Pending}, \texttt{Running} или \texttt{Finished}) с пагинацией, совершенных в проектах, к которым у пользователя есть как минимум \texttt{ReadOnly} доступ;
		    \item Редактирование запуска - остановка запуска, перевод его в состояние \texttt{Finished}. Операция может быть применена только к запускам в состоянии \texttt{Running} или \texttt{Pending};
		    \item Удаление запуска - удаление всех данных о запуске из базы данных. Операция может быть применена только к запускам в состоянии \texttt{Finished};
		    \item Создание запуска со следующей информацией
		    \begin{enumerate}
		        \item Краулер, с которым происходит запуск;
		        \item Настройки запуска --  это могут быть как и предопределенные настроки на \texttt{scrapyd} \footnote{\url{http://doc.scrapy.org/en/latest/topics/settings.html}}, так и собственные настройки;
		        \item Аргументы запуска -- аргументы для запуска краулера, которые передаются через командную строку;
		        \item Описание запуска;
		    \end{enumerate}
		\end{enumerate}
		
		\item \textbf{Периодические запуски\\}
		Должны быть реализованы запросы REST API для предоставления клиенту следующей функциональности
		\begin{enumerate}
		    \item Просмотр списка периодических запусков с пагинацией;
		    \item Редактирование следующей информации о периодическом запуске
		    \begin{enumerate}
		        \item Настройки будущих запусков --  это могут быть как и предопределенные настроки на \texttt{scrapyd}, так и собственные настройки;
		        \item Аргументы будущих запусков -- аргументы для запуска краулера, которые передаются через командную строку;
		        \item Краулер, с помощью которого будет совершен запуск;
		        \item cron-expression расписания запуска;
		    \end{enumerate}
		    \item Удаление периодического запуска;
		    \item Отмена последующих запусков - перевод периодической задачи в состояние \texttt{Disabled};
		    \item Возобновление запусков - перевод периодической задачи в состояние \texttt{Enabled};
		    \item Создание периодического запуска со следующими данными
		    \begin{enumerate}
		        \item Название;
		        \item Описание -- опциональное;
		        \item Краулер;
		        \item Приоритетность, влияющая на очередь запусков (\texttt{Low}, \texttt{Normal} или \texttt{High});
		        \item Статус (\texttt{Enabled} или \texttt{Disabled});
		        \item Настройки будущих запусков --  это могут быть как и предопределенные настроки на \texttt{scrapyd}, так и собственные настройки;
		        \item Аргументы будущих запусков -- аргументы для запуска краулера, которые передаются через командную строку;
		    \end{enumerate}
		\end{enumerate}
	
	\end{enumerate}

		
	
	
	
	\newpage
	\section{Условия выполнения программы}
	\subsection{Минимальный состав аппаратных средств}
	Минимальный состав технических конмпонент, необходимых для нормального функционирования программы:
	\begin{enumerate}
        \item Компьютер оснащенный процессором Intel Core i5 с тактовой частотой 2,3 ГГц;
        \item 16 Гб ОЗУ;
        \item Жесткий диск с объемом свободной памяти более чем 50 ГБ;
        \item Клавиатура и мышь;
        \item Доступ в интернет.
\end{enumerate}
	\subsection{Минимальный состав программных средств}
	Для нормального функционирования программы требуется компьютер, оснащенный следующими программными компонентами:
	\begin{enumerate}
        \item macOS 10.15.2;
        \item scrapyd~\cite{scrapyd};
        \item Scala 2.12.6;
        \item Play-framework 2.6.13;
        \item PostgreSQL 11~\cite{postgresql};
    \end{enumerate}
	\subsection{Требования к персоналу}
	Минимальное количество персонала, требуемого для работы программы, должно составлять не менее 1 штатной единицы со следующими навыками:    
    
    \begin{enumerate}
        \item Базовые навыки администрирования Unix~\cite{unix} систем;
        \item Базовые навыки администрирования базы данных PostgreSQL~\cite{postgresql};
        \item Базовые навыки работы с sbt~\cite{sbt}.
    \end{enumerate}
					\newpage 
	\section{Выполнение программы}
	
	\subsection{Подготовка проекта}
	В комплект поставки программы входит техническая документация, приложение (исполняемые файлы, примеры запросов и прочие необходимые для работы программы файлы) и презентацию проекта.
	\subsubsection{Запуск scrapyd}
	
	В директории проекта (или в любой другой директории на компьютере) надо создать пустую папку с произвольным названием.
	
	В пустой папке надо запустить предустановленный сервер командой \textt{scrapyd} (см. листинг ~\ref{scrapdyDir}). 
	
	\begin{lstlisting}[frame=single, basicstyle=\footnotesize\ttfamily,caption=Запуск scrapyd,captionpos=b,label=scrapdyDir]
> mkdir scrapyd_server
> cd scrapyd_server
> ls

> scrapyd
\end{lstlisting}
	
	\subsubsection{Создание таблиц}
	
	Для корректной работы сервера необходимо запустить и проинициализировать базу данных PostgreSQL~\cite{postgresql}.
	
	Для этого в директории проекта необходимо запустить следующую комманду из листинга ниже \ref{sbtMain}:
	
	\begin{lstlisting}[frame=single,basicstyle=\footnotesize\ttfamily,caption=Создание и инициализация таблиц,captionpos=b,label=sbtMain]
> sbt "runMain models.common.DBCreator"
\end{lstlisting}
	
	\subsection{Запуск программы}

	Для запуска сервера необходимо выполнить команду, находясь в директории проекта, из листинга \ref{sbtrun}.

\begin{lstlisting}[frame=single,basicstyle=\footnotesize\ttfamily,caption=Запуск сервера,captionpos=b,label=sbtrun]
> sbt run
\end{lstlisting}

    \addition{Используемые понятия и определения}
	\begin{description}
		\item[\textbf{Web scraping}] -- это сбор данных с различных интернет-ресурсов. Общий принцип его работы можно объяснить следующим образом: некий автоматизированный код выполняет GET-запросы на целевой сайт и получая ответ, парсит HTML-документ, ищет данные и преобразует их в заданный формат. \label{terms:webscraping}
		\item[\textbf{Проект}] -- сущность для объединения и предоставления доступа к запускам/краулерам/периодическим задачам. \label{terms:project}
		
		\item[\textbf{Веб краулер}] --  программа, являющаяся составной частью поисковой системы и предназначенная для перебора страниц Интернета с целью занесения информации о них в базу данных поисковика. Неотъемлемая часть проекта. Именно с помощью пауков пользователь может “краулить” сайты для сбора необходимой информации. \label{terms:spider}
		\item[\textbf{Запуск}] -- единоразовый запуск краулера с настройками и аргументами, указанными для этого запуска. \label{terms:job}
		\item[\textbf{Периодический запуск}] -- запуск с множеством настроек, повторяющийся в определенные периоды времени (запуски по cron-expression).
		\label{terms:pjob}
	\end{description}
	                            \newpage
	%\section{Источники, использованные при разработке}
	\renewcommand{\refname}{Список источников}
	\addcontentsline{toc}{section}{\refname}
	\begin{thebibliography}{7}
		\bibitem{scrapyd} Github scrapyd/scrapyd [Электронный ресурс] URL: \url{https://github.com/scrapy/scrapyd} (Дата обращения: 16.04.2020, режим доступа: свободный)
		\bibitem{gost}Единая система программной документации – М.: ИПК, Издательство стандартов, 2000, 125 стр.
		\bibitem{scalatestplus} ScalaTest+Play [Электронный ресурс] URL:\url{http://www.scalatest.org/plus}(Дата обращения: 16.04.2020, режим доступа: свободный)
		\bibitem{playsilhouettetestkit} Testing - silhouette [Электронный ресурс] URL:\url{https://www.silhouette.rocks/docs/testing} (Дата обращения: 16.04.2020, режим доступа: свободный)
		\bibitem{postgresql} Postgresql [Электронный ресурс] URL:\url{https://www.postgresql.org} (Дата обращения: 16.04.2020, режим доступа: свободный)
		\bibitem{unix} Unix [Электронный ресурс] URL: \url{https://en.wikipedia.org/wiki/Unix} (дата обращения: 10.10.2018).
		\bibitem{sbt} SBT [Электронный ресурс] URL: \uel{https://www.scala-sbt.org} (Дата обращения: 16.04.2020, режим доступа: свободный)
	\end{thebibliography}
						
						\newpage
	\listRegistration
	\addcontentsline{toc}{section}{Лист регистрации изменений}
\end{document} % конец документа