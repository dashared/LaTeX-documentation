\documentclass[a4paper,12pt]{article}
\usepackage{styledoc19}


\begin{document} % конец преамбулы, начало документа
	
	\year{2020}
	\docNumber{RU.17701729.04.13-01 34 01-1}
	\docFormat{Руководство оператора}
	\student{БПИ 174}{Д. Ю. Редникина}
	\supervisor{Профессор департамента \vfill программной инженерии  факультета компьютерных наук, к.т.н}
	{Е. М. Гринкруг}
	\project{СИСТЕМА УПРАВЛЕНИЯ ЗАДАНИЯМИ ПО АВТОМАТИЧЕСКОМУ СБОРУ ДАННЫХ ИЗ СЕТИ ИНТЕРНЕТ}
	
	\firstPage
	\newpage
	\secondPage
	\newpage
	\thirdPage
	\newpage
	\section{Назначение программы}
	\subsection{Функциональное назначение}
	Система будет применяться как средство управления проектами по созданию, редактированию и запуску веб краулеров для сбора данных в сети интернет. Продукт позволит следить за запусками в режиме реального времени, а также создавать периодические запуски по расписанию.

	\subsection{Эксплуатационное назначение}
	Web-приложение является компонентом CRM-системы для благотворительного фонда <<AIAIN>>, позволяющей облегчить бизнес процессы работы фонда с благополучателями и донорами. Web-приложение призвано обеспечить необходимую сотрудникам фонда функциональность для работы с системой. Им будут пользоваться как менеджеры фонда, так и администраторы, члены комиссий, операторы фонда, контент-менеджеры фонда. Каждый из пользователей будет иметь доступ к необходимой ему функциональности по обработке заявок, управлению фондом, администрированию и т.д.
	\subsection{Состав выполняемых функций}
	Следующие требования зафиксированы в документе <<Cистема управления заданиями по автоматическому сбору данных из сети Интернет. Техническое задание>> к составу выполняемых функций:
	\newlist{subreg}{enumerate}{10}
 
\setlist[subreg, 1]{label=\textbf{FR-\arabic*.}}
\setlist[subreg, 2]{label*=\textbf{\arabic*.}}
\setlist[subreg, 3]{label=\arabic*.}

\begin{subreg}
    \item \label{FR-1} \textbf{Аутентификация\\} 
	Сотрудник фонда должен иметь возможность авторизоваться в системе с логином/паролем, предварительно полученным после регистрации по почте. Процесс регистрации и получения логина/пароля описан в пункте \ref{enum:reg}. Если пользователь не зарегистрирован в системе или авторизуется с неверными данными, то система должна отобразить соответствующее сообщение. 
	
	При первоначальном авторизации в системе на новом устройстве пользователь должен увидеть диалоговое окно с возможностью принять или отклонить получение пуш-уведомлений (подробнее о пуш-уведомлениях в разделе \ref{push}).
	
	\item \textbf{Настройки\\}
    У пользователя должна быть возможность управлять личными данными в настройках системы.
    \begin{subreg} \label{settings}
        \item Должна быть возможность просмотра информации о своем профиле в настройках системы:
        \begin{subreg}
        \item ФИО, почта;
        \item Дата рождения;
        \item Город, страна, телефон;
        \item Фотография;
        \item Роль в системе (см. Приложение \ref{stuff});
        \item Назначенные категории (только для пользователей с ролью <<Член комиссии>>);
        \end{subreg}
        \item Должна быть возможность изменения основной информации:
    \begin{subreg}
        \item ФИО;
        \item Город, страна, телефон;
        \item Фотография;
        \item Дата рождения;
    \end{subreg}
        \item Должна быть возможность выбрать язык системы: русский или английский;
        \item \label{enum:push} Должна быть возможность включить/отключить получение пуш-уведомлений; 
    \end{subreg}
    \item \textbf{Пуш-уведомления\\} \label{push}
    Сотрудники фонда должны иметь возможность получать пуш-уведомления, если такая настройка включена (см. пункт \ref{enum:push}).
    
    \begin{subreg}
        \item Должна быть возможность получения пуш-уведомлений на входящие сообщения в чатах поддержки (для пользователя с ролью <<Оператор>>, подробнее о чатах в пункте \ref{req:chats});
        
        \item Должна быть возможность получения пуш-уведомлений при изменений статусов заявок, которые назначены на конкретного менеджера (для роли <<Менеджер>> и <<Член комиссии>>, подробнее в пункте \ref{req:status});
        
        \item Должна быть возможность получения пуш-уведомлений при изменении статуса операции в системе блокчейн (подробнее в пункте \ref{req:blockchain});
        
        \item Должна быть возможность просмотра списка уведомлений и информации о них:
        \begin{subreg}
        \item Дата и время уведомления;
        \item Тип уведомления;
        \item Инициатор уведомления;
        \end{subreg}
    \end{subreg}
    
    \item \textbf{Статусы операций в системе блокчейн\\} \label{req:blockchain}
    Сотрудники фонда должны иметь возможность просматривать изменения статусов операций в системе блокчейн, а именно: дата, время совершенной операции, тип операции, статус операций, дату, время обновления статуса.
    
    \item \textbf{Управление пользователями\\}
        Этот функционал должен быть доступен только для пользователя с ролью <<Администратор>> (кроме пункта \ref{enum:managers}).
        \begin{subreg}
        \label{admin}
        \item \label{enum:admin_1} Должна быть возможность просмотра информации о пользователе: его ФИО, роль в системе, город, страна, фотография, статус в системе (заблокирован или нет), назначенные категории (только для пользователей с ролью <<Член комиссии>>);
        \item Должна быть возможность изменить все поля из пункта \ref{enum:admin_1}, кроме почты;
        
        \item \label{enum:reg} Должна быть возможность зарегистрировать пользователя в системе, указав:
        \begin{subreg}
            \item ФИО;
            \item Почту;
            \item Роль пользователя в системе (см. Приложение \ref{stuff});
            \item Назначенные категории (только для пользователей с выбранной ролью <<Член комиссии>>);
        \end{subreg}
        После совершения процесса регистрации на указанную почту пользователю приходит логин/пароль для дальнейшей аутентификации в системе;
        \item \label{enum:managers} Должна быть возможность просмотра информации о сотрудниках фонда для пользователя с ролью <<Член комиссии>>: ФИО, роль, город, страна, день рождения, телефон, назначенные категории (если есть), заявки назначенные на пользователя (если есит);
        \end{subreg}
        
    \item \textbf{Логи системы\\}
    У пользователя с ролью <<Администратор>> должна быть возможность просматривать логи системы в формате JSON с обязательными полями: дата регистрации события, тип события и также описание события.
    
    \item \textbf{Транзакции\\}
    Этот функционал должен быть доступен только для пользователя с ролью <<Член комиссии>>.
    \begin{subreg}
        \item Должна быть возможность просматривать транзакции, совершенные внутри системы, информацию о них:
    \begin{subreg}
        \item Дата, время совершения транзакции;
        \item Кто совершил транзакцию (ФИО донора);
        \item Сумма транзакции;
        \item На какую заявку транзакция была совершена - основная информация о заявке: название, автор, тип заявки;
    \end{subreg}
        \item Должна быть возможность провести ручную транзакцию -- вести данные о платеже, поступившем напрямую в фонд. При вводе транзакции должна быть возможность указать цель платежа: на одну из заявок фонда или на нужды фонда, а также ФИО пользователя от которого поступило пожертвование, сумму пожертвования;
    \end{subreg}
    
    \item \textbf{Категории\\}
    Этот функционал должен быть доступен только для пользователя с ролью <<Член комиссии>>.
    \begin{subreg}
        \item Должна быть возможность просматривать категории фонда, доступные для назначения на заявки, а именно:
        \begin{subreg}
            \item ID - уникальный идентификатор категории;
            \item Название категории на английском;
            \item Название категории на русском языке;
            \item Название категории на арабском;
            \item Видимость категории - некоторые категории должны быть скрыты от пользователей при назначении категории на заявку;
        \end{subreg}
        \item Должна быть возможность изменить все данные о категориях;
        \item Должна быть возможность удалить категорию. Для удаления доступны только те категории, которые не используются в системе (т.е не назначены на пользователей или на заявку);
    \end{subreg}
    
    \item \textbf{Заявки\\}
    Эта функциональность должна быть доступна для пользователей с ролью <<Член комисии>> и <<Менеджер>>;
    \begin{subreg}
    \item \label{req:status} Должна быть возможность изменять статусы заявок в зависимости от роли пользователя и предыдущего статуса заявки в соответствии с диаграммой жизненного цикла заявки (см. Приложение \ref{status});
    \item Должна быть возможность отредактировать данные о заявке (одобренная сумма, срок сбора средств), когда она находится в статусе <<В обработке>>; 
    \item Должна быть возможность создать заявку в системе от лица незарегистрированного пользователя. При создании заявки нужно указать: название,  описание, сумма сбора, категория заявки, документы и дата сбора. Заявка сразу создается в статусе <<Активная>> от имени фонда. Эта функциональность должна быть доступна только для пользователей с ролью <<Член комиссии>>;
    \item Должна быть возможность оставлять комментирии к заявке;
    \item Должна быть возможность просматривать комментарии к заявке: текст сообщения, дату и автора;
    \item Должна быть возможность менять менеджера, который назначен на обработку заявки;
    \item Должна быть возможность закрыть сбор средств на заявку;
    \item Должна быть возможность просмотреть информацию о голосовании по заявке в статусе <<Ждет подтверждения члена комиссии>>, а именно: кто имеет право проголосовать и их решение, а также статус голосования (в процессе, принято, отклонено);
    \item У пользователей с ролью <<Член комиссии>> должна быть возможность проголосовать по заявке (за принятие или против), если категория заявки совпадает с назначенной на члена комиссии категорией;
    \end{subreg}
    
    \item \textbf{Чаты\\} \label{req:chats}
    Данная функциональность должна быть доступна только пользователям с ролью <<Оператор>>;

    \begin{subreg}
    \item Должна быть возможность просматривать список чатов: ФИО собеседника, текст сообщения, количество непрочитанных сообщений;
    \item Должна быть возможность написать сообщение;
    \end{subreg}
    
    \item \textbf{Управление контентом фонда\\}
    Данная функциональность должна быть доступна только для пользователей с ролью <<Контент-менеджер>>;
    
    \begin{subreg}
    \item Должна быть возможность просмотривать м редактировать часто задаваемые вопросы в формате \texttt{markdown} \cite{md};
    \item Должна быть возможность просмотривать, редактировать, удалять и создавать новости фонда. Для создания нужно указать название, описание новости и фотографию;
    \item Должна быть возможность просмотра и редактирования информации о фонде: описания и загруженных документов;
    \end{subreg}
    
\end{subreg}


\renewcommand{\labelenumi}{\arabic{enumi}.}

\renewcommand{\labelenumii}{\arabic{enumii}.}

\renewcommand{\labelenumiii}{\arabic{enumiii}.}





	
	\newpage
	\section{Условия выполнения программы}
	\subsection{Минимальный состав аппаратных средств}
	Минимальный состав технических конмпонент, необходимых для нормального функционирования программы:
	\begin{enumerate}
        \item Компьютер оснащенный процессором Intel Core i5 с тактовой частотой 2,3 ГГц;
        \item 16 Гб ОЗУ;
        \item Жесткий диск с объемом свободной памяти более чем 50 ГБ;
        \item Клавиатура и мышь;
        \item Доступ в интернет.
\end{enumerate}
	\subsection{Минимальный состав программных средств}
	Для нормального функционирования программы требуется компьютер, оснащенный следующими программными компонентами:
	\begin{enumerate}
        \item macOS 10.15.2;
        \item scrapyd~\cite{scrapyd};
        \item Scala 2.12.6;
        \item Play-framework 2.6.13;
        \item PostgreSQL 11~\cite{postgresql};
    \end{enumerate}
	\subsection{Требования к персоналу}
	Минимальное количество персонала, требуемого для работы программы, должно составлять не менее 1 штатной единицы со следующими навыками:    
    
    \begin{enumerate}
        \item Базовые навыки администрирования Unix~\cite{unix} систем;
        \item Базовые навыки администрирования базы данных PostgreSQL~\cite{postgresql};
        \item Базовые навыки работы с sbt~\cite{sbt}.
    \end{enumerate}
					\newpage 
	\section{Выполнение программы}
	
	\subsection{Подготовка проекта}
	В комплект поставки программы входит техническая документация, приложение (исполняемые файлы, примеры запросов и прочие необходимые для работы программы файлы) и презентацию проекта.
	\subsubsection{Запуск scrapyd}
	
	В директории проекта (или в любой другой директории на компьютере) надо создать пустую папку с произвольным названием.
	
	В пустой папке надо запустить предустановленный сервер командой \textt{scrapyd} (см. листинг ~\ref{scrapdyDir}). 
	
	\begin{lstlisting}[frame=single, basicstyle=\footnotesize\ttfamily,caption=Запуск scrapyd,captionpos=b,label=scrapdyDir]
> mkdir scrapyd_server
> cd scrapyd_server
> ls

> scrapyd
\end{lstlisting}
	
	\subsubsection{Создание таблиц}
	
	Для корректной работы сервера необходимо запустить и проинициализировать базу данных PostgreSQL~\cite{postgresql}.
	
	Для этого в директории проекта необходимо запустить следующую комманду из листинга ниже \ref{sbtMain}:
	
	\begin{lstlisting}[frame=single,basicstyle=\footnotesize\ttfamily,caption=Создание и инициализация таблиц,captionpos=b,label=sbtMain]
> sbt "runMain models.common.DBCreator"
\end{lstlisting}
	
	\subsection{Запуск программы}

	Для запуска сервера необходимо выполнить команду, находясь в директории проекта, из листинга \ref{sbtrun}.

\begin{lstlisting}[frame=single,basicstyle=\footnotesize\ttfamily,caption=Запуск сервера,captionpos=b,label=sbtrun]
> sbt run
\end{lstlisting}

    \addition{Используемые понятия и определения}
	\begin{description}
		\item[\textbf{Web scraping}] -- это сбор данных с различных интернет-ресурсов. Общий принцип его работы можно объяснить следующим образом: некий автоматизированный код выполняет GET-запросы на целевой сайт и получая ответ, парсит HTML-документ, ищет данные и преобразует их в заданный формат. \label{terms:webscraping}
		\item[\textbf{Проект}] -- сущность для объединения и предоставления доступа к запускам/краулерам/периодическим задачам. \label{terms:project}
		
		\item[\textbf{Веб краулер}] --  программа, являющаяся составной частью поисковой системы и предназначенная для перебора страниц Интернета с целью занесения информации о них в базу данных поисковика. Неотъемлемая часть проекта. Именно с помощью пауков пользователь может “краулить” сайты для сбора необходимой информации. \label{terms:spider}
		\item[\textbf{Запуск}] -- единоразовый запуск краулера с настройками и аргументами, указанными для этого запуска. \label{terms:job}
		\item[\textbf{Периодический запуск}] -- запуск с множеством настроек, повторяющийся в определенные периоды времени (запуски по cron-expression).
		\label{terms:pjob}
	\end{description}
	                            \newpage
	%\section{Источники, использованные при разработке}
	\renewcommand{\refname}{Список источников}
	\addcontentsline{toc}{section}{\refname}
	\begin{thebibliography}{7}
		\bibitem{scrapyd} Github scrapyd/scrapyd [Электронный ресурс] URL: \url{https://github.com/scrapy/scrapyd} (Дата обращения: 16.04.2020, режим доступа: свободный)
		\bibitem{gost}Единая система программной документации – М.: ИПК, Издательство стандартов, 2000, 125 стр.
		\bibitem{scalatestplus} ScalaTest+Play [Электронный ресурс] URL:\url{http://www.scalatest.org/plus}(Дата обращения: 16.04.2020, режим доступа: свободный)
		\bibitem{playsilhouettetestkit} Testing - silhouette [Электронный ресурс] URL:\url{https://www.silhouette.rocks/docs/testing} (Дата обращения: 16.04.2020, режим доступа: свободный)
		\bibitem{postgresql} Postgresql [Электронный ресурс] URL:\url{https://www.postgresql.org} (Дата обращения: 16.04.2020, режим доступа: свободный)
		\bibitem{unix} Unix [Электронный ресурс] URL: \url{https://en.wikipedia.org/wiki/Unix} (дата обращения: 10.10.2018).
		\bibitem{sbt} SBT [Электронный ресурс] URL: \uel{https://www.scala-sbt.org} (Дата обращения: 16.04.2020, режим доступа: свободный)
	\end{thebibliography}
						
						\newpage
	\listRegistration
	\addcontentsline{toc}{section}{Лист регистрации изменений}
\end{document} % конец документа