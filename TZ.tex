\documentclass[a4paper,12pt,reqno]{article}

\usepackage{styledoc19}

\begin{document} % конец преамбулы, начало документа
	
	
	\year{2020}
    \docNumber{RU.17701729.04.13-01 ТЗ 01-1-ЛУ}
	\docFormat{Техническое задание}
	\student{БПИ 174}{Д. Ю. Редникина}
	
	\project{СИСТЕМА УПРАВЛЕНИЯ ЗАДАНИЯМИ ПО АВТОМАТИЧЕСКОМУ СБОРУ ДАННЫХ ИЗ СЕТИ ИНТЕРНЕТ}
	
	\supervisor{Профессор департамента \vfill программной инженерии  факультета компьютерных наук, к.т.н}
	{Е. М. Гринкруг}
	
	\firstPage
						\newpage
	\secondPage
						\newpage
	\thirdPage
						\newpage
	\section{Введение}
	\subsection{Наименование программы}
	\subsubsection{Наименование программы на русском языке}
	CRM-система для благотворительного фонда <<AIAIN>>. Web-приложение для сотрудников фонда
	\subsubsection{Наименование программы на английском языке}
	System for managing tasks of collecting data from the Internet

	\subsection{Краткая характеристика области применения}
	Ежегодно сотни тысяч людей жертвуют свои деньги некоммерческим организациям. Еще больше людей обращаются за помощью в общественные благотворительные фонды. Управлять благотворительным фондом становится все сложнее. Нужно не только принимать пожертвования и регистрировать заявки на сборы средств, но и контролировать работников фонда, волонтеров, подготавливать документацию для вышестоящих органов и многое другое. 

Следовательно, некоммерческие организации нуждаются в платформе с функционалом, ориентированным на бизнес процессы фонда, чтобы обрабатывать всю необходимую информацию в одном месте. К сожалению, большинство фондов пользуются электронными таблицами или, что еще хуже, ведут записи в бумажной форме. Как результат, на каждое действие тратятся большое количество времени и ресурсов. Арабский фонд <<AIAIN>> также столкнулся с проблемой автоматизации бизнес процессов, специфичных для предметной области благотворительности. 


Разрабатываемое Web-приложение ориентировано на специфические нужды некоммерческой организации <<AIAIN>>. Web-приложение для сотрудников фонда может предоставить все преимущества, которыми бизнес-компании пользовались в течение многих лет и чего так не хватало этому фонду. Это также позволит наладить бизнес-процессы внутри фонда <<AIAIN>>, настроить тайм-менеджмент и автоматизировать составление отчетности.
	\newpage
	\section{Основания для разработки}
	\subsection{Документы, на основании которых ведется разработка}
	
	
	Приказ декана факультета компьютерных наук И.В. Аржанцева "Об утверждении тем, руководителей курсовых работ студентов образовательной программы «Программная инженерия» факультета компьютерных наук" № 2.3-02/1112-04 от 11.12.2019.
	
	\subsection{Наименование темы разработки}
	
	Наименование темы разработки – <<CRM-система для благотворительного фонда <<AIAIN>>. Web-приложение для сотрудников фонда>> (<<System for managing tasks of collecting data from the Internet
>>)


 Программа выполняется в рамках темы курсовой работы в соответствии с учебным планом подготовки бакалавров по направлению 09.03.04 «Программная инженерия» Национального исследовательского университета «Высшая школа экономики», факультет компьютерных наук.
	
	\newpage 
	\section{Назначение разработки}
	 
	\subsection{Функциональное назначение}
	Система будет применяться как средство управления проектами по созданию, редактированию и запуску веб краулеров для сбора данных в сети интернет. Продукт позволит следить за запусками в режиме реального времени, а также создавать периодические запуски по расписанию.

	\subsection{Эксплуатационное назначение}
	Web-приложение является компонентом CRM-системы для благотворительного фонда <<AIAIN>>, позволяющей облегчить бизнес процессы работы фонда с благополучателями и донорами. Web-приложение призвано обеспечить необходимую сотрудникам фонда функциональность для работы с системой. Им будут пользоваться как менеджеры фонда, так и администраторы, члены комиссий, операторы фонда, контент-менеджеры фонда. Каждый из пользователей будет иметь доступ к необходимой ему функциональности по обработке заявок, управлению фондом, администрированию и т.д. 
	
						\newpage
	\section{Требования к программе}
	
	\subsection{Функциональные требования}

    \newlist{subreg}{enumerate}{10}
 
\setlist[subreg, 1]{label=\textbf{FR-\arabic*.}}
\setlist[subreg, 2]{label*=\textbf{\arabic*.}}
\setlist[subreg, 3]{label=\arabic*.}

\begin{subreg}
    \item \label{FR-1} \textbf{Аутентификация\\} 
	Сотрудник фонда должен иметь возможность авторизоваться в системе с логином/паролем, предварительно полученным после регистрации по почте. Процесс регистрации и получения логина/пароля описан в пункте \ref{enum:reg}. Если пользователь не зарегистрирован в системе или авторизуется с неверными данными, то система должна отобразить соответствующее сообщение. 
	
	При первоначальном авторизации в системе на новом устройстве пользователь должен увидеть диалоговое окно с возможностью принять или отклонить получение пуш-уведомлений (подробнее о пуш-уведомлениях в разделе \ref{push}).
	
	\item \textbf{Настройки\\}
    У пользователя должна быть возможность управлять личными данными в настройках системы.
    \begin{subreg} \label{settings}
        \item Должна быть возможность просмотра информации о своем профиле в настройках системы:
        \begin{subreg}
        \item ФИО, почта;
        \item Дата рождения;
        \item Город, страна, телефон;
        \item Фотография;
        \item Роль в системе (см. Приложение \ref{stuff});
        \item Назначенные категории (только для пользователей с ролью <<Член комиссии>>);
        \end{subreg}
        \item Должна быть возможность изменения основной информации:
    \begin{subreg}
        \item ФИО;
        \item Город, страна, телефон;
        \item Фотография;
        \item Дата рождения;
    \end{subreg}
        \item Должна быть возможность выбрать язык системы: русский или английский;
        \item \label{enum:push} Должна быть возможность включить/отключить получение пуш-уведомлений; 
    \end{subreg}
    \item \textbf{Пуш-уведомления\\} \label{push}
    Сотрудники фонда должны иметь возможность получать пуш-уведомления, если такая настройка включена (см. пункт \ref{enum:push}).
    
    \begin{subreg}
        \item Должна быть возможность получения пуш-уведомлений на входящие сообщения в чатах поддержки (для пользователя с ролью <<Оператор>>, подробнее о чатах в пункте \ref{req:chats});
        
        \item Должна быть возможность получения пуш-уведомлений при изменений статусов заявок, которые назначены на конкретного менеджера (для роли <<Менеджер>> и <<Член комиссии>>, подробнее в пункте \ref{req:status});
        
        \item Должна быть возможность получения пуш-уведомлений при изменении статуса операции в системе блокчейн (подробнее в пункте \ref{req:blockchain});
        
        \item Должна быть возможность просмотра списка уведомлений и информации о них:
        \begin{subreg}
        \item Дата и время уведомления;
        \item Тип уведомления;
        \item Инициатор уведомления;
        \end{subreg}
    \end{subreg}
    
    \item \textbf{Статусы операций в системе блокчейн\\} \label{req:blockchain}
    Сотрудники фонда должны иметь возможность просматривать изменения статусов операций в системе блокчейн, а именно: дата, время совершенной операции, тип операции, статус операций, дату, время обновления статуса.
    
    \item \textbf{Управление пользователями\\}
        Этот функционал должен быть доступен только для пользователя с ролью <<Администратор>> (кроме пункта \ref{enum:managers}).
        \begin{subreg}
        \label{admin}
        \item \label{enum:admin_1} Должна быть возможность просмотра информации о пользователе: его ФИО, роль в системе, город, страна, фотография, статус в системе (заблокирован или нет), назначенные категории (только для пользователей с ролью <<Член комиссии>>);
        \item Должна быть возможность изменить все поля из пункта \ref{enum:admin_1}, кроме почты;
        
        \item \label{enum:reg} Должна быть возможность зарегистрировать пользователя в системе, указав:
        \begin{subreg}
            \item ФИО;
            \item Почту;
            \item Роль пользователя в системе (см. Приложение \ref{stuff});
            \item Назначенные категории (только для пользователей с выбранной ролью <<Член комиссии>>);
        \end{subreg}
        После совершения процесса регистрации на указанную почту пользователю приходит логин/пароль для дальнейшей аутентификации в системе;
        \item \label{enum:managers} Должна быть возможность просмотра информации о сотрудниках фонда для пользователя с ролью <<Член комиссии>>: ФИО, роль, город, страна, день рождения, телефон, назначенные категории (если есть), заявки назначенные на пользователя (если есит);
        \end{subreg}
        
    \item \textbf{Логи системы\\}
    У пользователя с ролью <<Администратор>> должна быть возможность просматривать логи системы в формате JSON с обязательными полями: дата регистрации события, тип события и также описание события.
    
    \item \textbf{Транзакции\\}
    Этот функционал должен быть доступен только для пользователя с ролью <<Член комиссии>>.
    \begin{subreg}
        \item Должна быть возможность просматривать транзакции, совершенные внутри системы, информацию о них:
    \begin{subreg}
        \item Дата, время совершения транзакции;
        \item Кто совершил транзакцию (ФИО донора);
        \item Сумма транзакции;
        \item На какую заявку транзакция была совершена - основная информация о заявке: название, автор, тип заявки;
    \end{subreg}
        \item Должна быть возможность провести ручную транзакцию -- вести данные о платеже, поступившем напрямую в фонд. При вводе транзакции должна быть возможность указать цель платежа: на одну из заявок фонда или на нужды фонда, а также ФИО пользователя от которого поступило пожертвование, сумму пожертвования;
    \end{subreg}
    
    \item \textbf{Категории\\}
    Этот функционал должен быть доступен только для пользователя с ролью <<Член комиссии>>.
    \begin{subreg}
        \item Должна быть возможность просматривать категории фонда, доступные для назначения на заявки, а именно:
        \begin{subreg}
            \item ID - уникальный идентификатор категории;
            \item Название категории на английском;
            \item Название категории на русском языке;
            \item Название категории на арабском;
            \item Видимость категории - некоторые категории должны быть скрыты от пользователей при назначении категории на заявку;
        \end{subreg}
        \item Должна быть возможность изменить все данные о категориях;
        \item Должна быть возможность удалить категорию. Для удаления доступны только те категории, которые не используются в системе (т.е не назначены на пользователей или на заявку);
    \end{subreg}
    
    \item \textbf{Заявки\\}
    Эта функциональность должна быть доступна для пользователей с ролью <<Член комисии>> и <<Менеджер>>;
    \begin{subreg}
    \item \label{req:status} Должна быть возможность изменять статусы заявок в зависимости от роли пользователя и предыдущего статуса заявки в соответствии с диаграммой жизненного цикла заявки (см. Приложение \ref{status});
    \item Должна быть возможность отредактировать данные о заявке (одобренная сумма, срок сбора средств), когда она находится в статусе <<В обработке>>; 
    \item Должна быть возможность создать заявку в системе от лица незарегистрированного пользователя. При создании заявки нужно указать: название,  описание, сумма сбора, категория заявки, документы и дата сбора. Заявка сразу создается в статусе <<Активная>> от имени фонда. Эта функциональность должна быть доступна только для пользователей с ролью <<Член комиссии>>;
    \item Должна быть возможность оставлять комментирии к заявке;
    \item Должна быть возможность просматривать комментарии к заявке: текст сообщения, дату и автора;
    \item Должна быть возможность менять менеджера, который назначен на обработку заявки;
    \item Должна быть возможность закрыть сбор средств на заявку;
    \item Должна быть возможность просмотреть информацию о голосовании по заявке в статусе <<Ждет подтверждения члена комиссии>>, а именно: кто имеет право проголосовать и их решение, а также статус голосования (в процессе, принято, отклонено);
    \item У пользователей с ролью <<Член комиссии>> должна быть возможность проголосовать по заявке (за принятие или против), если категория заявки совпадает с назначенной на члена комиссии категорией;
    \end{subreg}
    
    \item \textbf{Чаты\\} \label{req:chats}
    Данная функциональность должна быть доступна только пользователям с ролью <<Оператор>>;

    \begin{subreg}
    \item Должна быть возможность просматривать список чатов: ФИО собеседника, текст сообщения, количество непрочитанных сообщений;
    \item Должна быть возможность написать сообщение;
    \end{subreg}
    
    \item \textbf{Управление контентом фонда\\}
    Данная функциональность должна быть доступна только для пользователей с ролью <<Контент-менеджер>>;
    
    \begin{subreg}
    \item Должна быть возможность просмотривать м редактировать часто задаваемые вопросы в формате \texttt{markdown} \cite{md};
    \item Должна быть возможность просмотривать, редактировать, удалять и создавать новости фонда. Для создания нужно указать название, описание новости и фотографию;
    \item Должна быть возможность просмотра и редактирования информации о фонде: описания и загруженных документов;
    \end{subreg}
    
\end{subreg}


\renewcommand{\labelenumi}{\arabic{enumi}.}

\renewcommand{\labelenumii}{\arabic{enumii}.}

\renewcommand{\labelenumiii}{\arabic{enumiii}.}





    
    \subsection{Требования к формату входных и выходных данных}

	\begin{enumerate}
		\setcounter{enumii}{2}
		\item В качестве входных данных сервер принимает REST~\cite{rest} запросы от клиентских приложений, в теле которых передаются сериализованные в формате JSON~\cite{json} данные.
		\item Сервер обрабатывает JSON~\cite{json} ответы от сервера \texttt{scrapyd} \cite{scrapyd}.
		\item Сервер принимает информацию от базы данных PostgreSQL~\cite{postgresql}.
	\end{enumerate}
	
	
	\subsection{Условия эксплуатации}
	\subsubsection{Климатические условия}
	Климатические условия должны сопадать с климатическими условиями эксплуатации устройства. 
	\subsubsection{Требования к пользователю}
	
	Пользователь должен быть ознакомлен с документами <<Руководство программиста  <<CRM-система для благотворительного фонда <<AIAIN>>. Web-приложение для сотрудников фонда>> и <<Руководство пользователя <<CRM-система для благотворительного фонда <<AIAIN>>. Web-приложение для сотрудников фонда>>, а также разбираться в терминологии \ref{terms}.
	\subsection{Требования к составу и параметру технических средств}
	
	Минимальный состав технических компонент, необходимый для нормального функционирования программы:
    
    \begin{enumerate}
        \item компьютер оснащенный процессором не ниже Intel Pentium/Celeron, или совместимый с ними с тактовой частотой не ниже 1,3 ГГц;
        \item 1 Гб ОЗУ или более;
        \item жесткий диск с объемом свободной памяти не менее 4 ГБ;
        \item клавиатура;
        \item доступ в интернет.
    \end{enumerate}
	
	
	\subsection{Требования к информационной и программной совместимости}
	
	Для нормального функционирования программы требуется компьютер, оснащенный следующими программными компонентами:
    
    \begin{enumerate}
        \item Ubuntu Server 18.04.2 LTS~\cite{ubuntu};
        \item PostgreSQL 11~\cite{postgresql};
        \item scrapyd~\cite{scrapyd};
        \item Scala 2.12.6~\cite{scala};
    \end{enumerate}
	
	
	\subsection{Требования к маркировке и упаковке}
	Приложение должно быть доступно для установки из архива проекта, при скачивании из системы LMS НИУ ВШЭ~\cite{lms}.
	
						\newpage
	\section{Требования к программной документации}
    Состав программной документации должен включать в себя следующие компоненты:
\begin{enumerate}
	\item Техническое задание <<CRM-система для благотворительного фонда <<AIAIN>>. Web-приложение для сотрудников фонда>> (ГОСТ 19.201-78) \label{tz}
	\item Программа и методика испытаний <<CRM-система для благотворительного фонда <<AIAIN>>. Web-приложение для сотрудников фонда>> (ГОСТ 19.301-78) \label{pmi}
	\item Руководство оператора <<CRM-система для благотворительного фонда <<AIAIN>>. Web-приложение для сотрудников фонда>> (ГОСТ 19.505-79) \label{ro}
	\item Текст программы <<CRM-система для благотворительного фонда <<AIAIN>>. Web-приложение для сотрудников фонда>> (ГОСТ 19.401-78) \label{tp}
\end{enumerate}

\indent
Вся документация должна быть составлена согласно ЕСПД (ГОСТ 19.101-77, 19.104-78, 19.105-78, 19.106-78 и ГОСТ к соответствующим документам (см. выше)) \cite{gost}. Все документы сдаются в электронном виде в составе выпускной квалификационной работы LMS НИУ ВШЭ.

% Пояснительная записка <<CRM-система для благотворительного фонда <<AIAIN>>. Web-приложение для сотрудников фонда>> должна быть проверена на плагиат ($< 40\% $ заимствований). Документ, подтвержадющий проверку Пояснительной записки сдается в печатном виде вместе с подписанным отзывом от научного руководителя.

	
						\newpage
	\section{Технико-экономические показатели}
	\subsection{Предполагаемая потребность}
	Программа будет использоваться программистами, которые используют web-scraping (\ref{terms}) для отслеживания изменений, скачивания данных из сети интернет. 
	
	\subsection{Ориентировочная экономическая эффективность} 
	Полная функциональность главного аналога \cite{scrapinghub} не доступна для бесплатного использования. 
	
	Разрабатываемая система будет бесплатной и будет иметь англоязычный интерфейс.
	
						\newpage
	\section{Стадии и этапы разработки}
	
	\subsection{Необходимые стадии разработки, этапы и содержание работ}
	\begin{enumerate}
		\item \textit{Техническое задание:}
		\begin{enumerate}
			\item Этапы разработки:
			\begin{enumerate}
				\item Обоснование необходимости разработки программы; 
				\item Постановка задачи; 
				\item Сбор исходных материалов; 
				\item Выбор и обоснование критериев эффективности и качества разрабатываемой программы; 
				\item Обоснование необходимости проведения научно-исследовательских работ; 
			\end{enumerate}
			\item Разработка и утверждение технического задания:
			\begin{enumerate}
				\item Определение требований к программе; 
				\item Определение стадий, этапов и сроков разработки программы и документации на неё; 
				\item Согласование и утверждение технического задания; 
			\end{enumerate}
		\end{enumerate}
		\item \textit{Технический проект:}
		\begin{enumerate}
			\item Разработка технического проекта:
			\begin{enumerate}
				\item Уточнение структуры входных и выходных данных; 
				\item Разработка алгоритма решения задачи; 
				\item Определение формы представления входных и выходных данных; 
				\item Разработка структуры программы; 
				\item Окончательное определение конфигурации технических средств. 
			\end{enumerate}
			\item Утверждение технического проекта:
			\begin{enumerate}
				\item Разработка пояснительной записки; 
				\item Согласование и утверждение технического проекта. 
			\end{enumerate}
		\end{enumerate}
		\item \textit{Рабочий проект:}
		\begin{enumerate}
			\item Разработка программы:
			\begin{enumerate}
				\item Программирование и отладка программы. 
			\end{enumerate}
			\item Разработка программной документации:
			\begin{enumerate}
				\item Разработка программных документов в соответствии с требованиями ГОСТ 19.101-77 \cite{gost}. 
			\end{enumerate}
			\item Испытания программы:
			\begin{enumerate}
				\item Разработка, согласование и утверждение порядка и методики испытаний; 
				\item Корректировка программы и программной документации по результатам испытаний.
			\end{enumerate}
		\end{enumerate}
	\end{enumerate}
	
	% приложения нумеруются отдельно и надо выровнять по правому краю

						\newpage
	\addition{Используемые понятия и определения} \label{terms}
	\begin{description}
		\item[\textbf{Web scraping}] -- это сбор данных с различных интернет-ресурсов. Общий принцип его работы можно объяснить следующим образом: некий автоматизированный код выполняет GET-запросы на целевой сайт и получая ответ, парсит HTML-документ, ищет данные и преобразует их в заданный формат. \label{terms:webscraping}
		\item[\textbf{Проект}] -- сущность для объединения и предоставления доступа к запускам/краулерам/периодическим задачам. \label{terms:project}
		
		\item[\textbf{Веб краулер}] --  программа, являющаяся составной частью поисковой системы и предназначенная для перебора страниц Интернета с целью занесения информации о них в базу данных поисковика. Неотъемлемая часть проекта. Именно с помощью пауков пользователь может “краулить” сайты для сбора необходимой информации. \label{terms:spider}
		\item[\textbf{Запуск}] -- единоразовый запуск краулера с настройками и аргументами, указанными для этого запуска. \label{terms:job}
		\item[\textbf{Периодический запуск}] -- запуск с множеством настроек, повторяющийся в определенные периоды времени (запуски по cron-expression).
		\label{terms:pjob}
	\end{description}


						\newpage
	%\section{Источники, использованные при разработке}
	%\renewcommand{\refname}{Список источников}
	% \addcontentsline{toc}{subsection}{\refname}
	\patchcmd{\thebibliography}{\section*{\refname}}{}{}{}
	\addition{Список источников}
	\begin{thebibliography}{3}
		\bibitem{gost}Единая система программной документации – М.: ИПК, Издательство стандартов, 2000, 125 стр.
		\bibitem{lms} 
		LMS [Электронный ресурс] URL: 
		\url{https://lms.hse.ru} (Дата обращения: 16.05.2020, режим доступа: свободный)
		\bibitem{json} JSON [Электронный ресурс] URL: \url{https://www.json.org} (Дата обращения: 16.05.2020, режим доступа: свободный)
		\bibitem{rest} Representational state transfer URL: \url{https://en.wikipedia.org/wiki/Representational\_state\\\_transfer} (дата обращения: 2020.12.14).
		\bibitem{scrapinghub} Scrapinghub URL: https://scrapinghub.com (дата обращения: 2019.12.14).
		\bibitem{postgresql} Postgresql [Электронный ресурс] URL:\url{https://www.postgresql.org} (Дата обращения: 16.04.2020, режим доступа: свободный)
		\bibitem{scrapyd} Github scrapyd/scrapyd [Электронный ресурс] URL: \url{https://github.com/scrapy/scrapyd} (Дата обращения: 16.04.2020, режим доступа: свободный)
		\bibitem{ubuntu} Ubuntu [Электронный ресурс] URL: \url{https://www.ubuntu.com} (Дата обращения: 16.04.2020).
		\bibitem{scala} Scala [Электронный ресурс] URL: \url{https://www.scala-lang.org} (Дата обращения: 16.04.2020).
		
	\end{thebibliography}

						\newpage
	\listRegistration

\end{document} % конец документа