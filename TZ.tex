\documentclass[a4paper,12pt,reqno]{article}

\usepackage{styledoc19}

\begin{document} % конец преамбулы, начало документа
	
	
	\year{2020}
    \docNumber{RU.17701729.04.13-01 ТЗ 01-1-ЛУ}
	\docFormat{Техническое задание}
	\student{БПИ 174}{Д. Ю. Редникина}
	
	\project{СИСТЕМА УПРАВЛЕНИЯ ЗАДАНИЯМИ ПО АВТОМАТИЧЕСКОМУ СБОРУ ДАННЫХ ИЗ СЕТИ ИНТЕРНЕТ}
	
	\supervisor{Профессор департамента \vfill программной инженерии  факультета компьютерных наук, к.т.н}
	{Е. М. Гринкруг}
	
	\firstPage
						\newpage
	\secondPage
						\newpage
	\thirdPage
						\newpage
	\section{Введение}
	\subsection{Наименование программы}
	\subsubsection{Наименование программы на русском языке}
	Cистема управления заданиями по автоматическому сбору данных из сети Интернет

	\subsubsection{Наименование программы на английском языке}
	System for managing tasks of collecting data from the Internet

	\subsection{Краткая характеристика области применения}
	Технологии web-scraping используются как в науке, так и в бизнесе - многие люди чувствуют потребность в извлечении данных из HTML разметки интернет страниц. Существующие аналоги реализуют базовый функционал(сбор данных), но не предоставляют такие дополнительные возможности как периодический запуск или совместное редактирование. Многие аналоги (прим. scrapyd~\cite{scrapyd}) имеют ограниченный функционал.
Главные возможности, которыми продукт обеспечит предполагаемых пользователей:
\begin{itemize}
    \item Совместное управление запусками клаулеров
    \item Периодический запуск задач
    \item Сбор логов, ошибок
    \item Группировка краулеров,а также их запусков в проект 
    \item Бесплатная функциональность
\end{itemize}

	\newpage
	\section{Основания для разработки}
	\subsection{Документы, на основании которых ведется разработка}
	
	
	Приказ декана факультета компьютерных наук И.В. Аржанцева <<Об утверждении тем, руководителей выпускных квалификационных работ студентов образовательной программы «Программная инженерия» факультета компьютерных наук>> № 2.3-02/280421-1 от 28.04.2021.
	
	\subsection{Наименование темы разработки}
	
	Наименование темы разработки – <<Cистема управления заданиями по автоматическому сбору данных из сети Интернет
>> (<<System for managing tasks of collecting data from the Internet
>>)


 Программа выполняется в рамках темы курсовой работы в соответствии с учебным планом подготовки бакалавров по направлению 09.03.04 «Программная инженерия» Национального исследовательского университета «Высшая школа экономики», факультет компьютерных наук.
	
	\newpage 
	\section{Назначение разработки}
	 
	\subsection{Функциональное назначение}
	Система будет применяться как средство управления проектами по созданию, редактированию и запуску веб краулеров для сбора данных в сети интернет. Продукт позволит следить за запусками в режиме реального времени, а также создавать периодические запуски по расписанию.

	\subsection{Эксплуатационное назначение}
	Web-приложение является компонентом CRM-системы для благотворительного фонда <<AIAIN>>, позволяющей облегчить бизнес процессы работы фонда с благополучателями и донорами. Web-приложение призвано обеспечить необходимую сотрудникам фонда функциональность для работы с системой. Им будут пользоваться как менеджеры фонда, так и администраторы, члены комиссий, операторы фонда, контент-менеджеры фонда. Каждый из пользователей будет иметь доступ к необходимой ему функциональности по обработке заявок, управлению фондом, администрированию и т.д. 
	
						\newpage
	\section{Требования к программе}
	
	\subsection{Функциональные требования}

    

\begin{enumerate}
		
		\item \textbf{Авторизация\\}
		Чтобы использовать сервис, клиентская программа должна иметь возможность авторизоваться в системе с помощью REST API
		\begin{enumerate}
			\item Для регистрации пользователю нужно указать следующие данные 
			\begin{enumerate}
			    \item Почта - уникальна для каждого зарегистрированного пользователя; 
			    \item Имя - длина больше 1 символ;
			    \item Логин - длина больше 2 символов;
			    \item Пароль - длина больше 2 символов;
			\end{enumerate}
			\item Для авторизации пользователя в системе должны быть указаны следующие данные
			\begin{enumerate}
			    \item Почта;
			    \item Пароль;
			\end{enumerate}
		\end{enumerate}
		
		\item \textbf{Проекты\\}
		Должны быть реализованы запросы REST API для предоставления клиенту следующей функциональности
		\begin{enumerate}
	    	\item Создание проекта со следующей информацией
	    	\begin{enumerate}
	    	    \item Имя проекта;
	    	    \item Описание проекта - опциональное поле;
	    	\end{enumerate}
			\item Обновление метаданных о проекте (редактирование) могут быть обновлены только участником с минимальным уровнем дотупа \texttt{ReadAndWrite}. Следующие данные могут быть обновлены:
			\begin{enumerate}
			    \item Имя проекта;
			    \item Описание проекта;
			    \item Настройки проекта для запуска краулеров;
			    \item Аргументы для запуска краулеров проекта;
			\end{enumerate}
			\item Обновление \texttt{egg} файла проекта (редактирование) -- минимальный уровень доступа участника, обновляющий данные о проекте \texttt{ReadAndWrite}.  
			\item Удаление данных о проекте. Удалить проект может только владелец \texttt{Owner}.
			\item Просмотр списка проектов (с пагинацией), к которым у пользователя есть как минимум \texttt{ReadOnly} доступ.
		\end{enumerate}
		
		\item \textbf{Участники проектов\\}
		Должны быть реализованы запросы REST API для предоставления клиенту следующей функциональности
		\begin{enumerate}
	    	\item Просмотр информации об участниках проекта;
	    	\begin{enumerate}
	    	    \item Имя, почта, логин участника;
	    	    \item Статус участника в проекте (\texttt{ReadOnly}, \texttt{ReadAndWrite} или \texttt{Owner});
	    	\end{enumerate}
			\item Обновление статуса участника проекта. Это действие совершать может только владелец проекта; 
			\item Удаление участника из проекта. Данное действие может совершать только владелец проекта;
			\item Добавление нового участника с указанными правами на редактирование. Данное действие может совершать только владелец проекта;
		\end{enumerate}
		
		\item \textbf{Краулеры\\}
		Должны быть реализованы запросы REST API для предоставления клиенту следующей функциональности
		\begin{enumerate}
			\item Просмотр списка краулеров проекта;
			\item Редактирование информации о краулере для последующих запусков. Следующая информация может быть изменена
			\begin{enumerate}
			    \item Настройки краулера для запуска;
			    \item Аргументы для запуска;
			\end{enumerate}
		\end{enumerate}
		
		\item \textbf{Запуски краулеров\\}
		Должны быть реализованы запросы REST API для предоставления клиенту следующей функциональности
		\begin{enumerate}
		    \item Просмотр списка запусков в определенном статусе (\texttt{Pending}, \texttt{Running} или \texttt{Finished}) с пагинацией, совершенных в проектах, к которым у пользователя есть как минимум \texttt{ReadOnly} доступ;
		    \item Редактирование запуска - остановка запуска, перевод его в состояние \texttt{Finished}. Операция может быть применена только к запускам в состоянии \texttt{Running} или \texttt{Pending};
		    \item Удаление запуска - удаление всех данных о запуске из базы данных. Операция может быть применена только к запускам в состоянии \texttt{Finished};
		    \item Создание запуска со следующей информацией
		    \begin{enumerate}
		        \item Краулер, с которым происходит запуск;
		        \item Настройки запуска --  это могут быть как и предопределенные настроки на \texttt{scrapyd} \footnote{\url{http://doc.scrapy.org/en/latest/topics/settings.html}}, так и собственные настройки;
		        \item Аргументы запуска -- аргументы для запуска краулера, которые передаются через командную строку;
		        \item Описание запуска;
		    \end{enumerate}
		\end{enumerate}
		
		\item \textbf{Периодические запуски\\}
		Должны быть реализованы запросы REST API для предоставления клиенту следующей функциональности
		\begin{enumerate}
		    \item Просмотр списка периодических запусков с пагинацией;
		    \item Редактирование следующей информации о периодическом запуске
		    \begin{enumerate}
		        \item Настройки будущих запусков --  это могут быть как и предопределенные настроки на \texttt{scrapyd}, так и собственные настройки;
		        \item Аргументы будущих запусков -- аргументы для запуска краулера, которые передаются через командную строку;
		        \item Краулер, с помощью которого будет совершен запуск;
		        \item cron-expression расписания запуска;
		    \end{enumerate}
		    \item Удаление периодического запуска;
		    \item Отмена последующих запусков - перевод периодической задачи в состояние \texttt{Disabled};
		    \item Возобновление запусков - перевод периодической задачи в состояние \texttt{Enabled};
		    \item Создание периодического запуска со следующими данными
		    \begin{enumerate}
		        \item Название;
		        \item Описание -- опциональное;
		        \item Краулер;
		        \item Приоритетность, влияющая на очередь запусков (\texttt{Low}, \texttt{Normal} или \texttt{High});
		        \item Статус (\texttt{Enabled} или \texttt{Disabled});
		        \item Настройки будущих запусков --  это могут быть как и предопределенные настроки на \texttt{scrapyd}, так и собственные настройки;
		        \item Аргументы будущих запусков -- аргументы для запуска краулера, которые передаются через командную строку;
		    \end{enumerate}
		\end{enumerate}
	
	\end{enumerate}

		
	
	
    
    \subsection{Требования к формату входных и выходных данных}

	\begin{enumerate}
		\setcounter{enumii}{2}
		\item В качестве входных данных сервер принимает REST~\cite{rest} запросы от клиентских приложений, в теле которых передаются сериализованные в формате JSON~\cite{json} данные.
		\item Сервер обрабатывает JSON~\cite{json} ответы от сервера \texttt{scrapyd} \cite{scrapyd}.
		\item Сервер принимает информацию от базы данных PostgreSQL~\cite{postgresql}.
	\end{enumerate}
	
	
	\subsection{Условия эксплуатации}
	\subsubsection{Климатические условия}
	Климатические условия должны сопадать с климатическими условиями эксплуатации устройства. 
	\subsubsection{Требования к пользователю}
	
	Пользователь должен быть ознакомлен с документами <<Руководство программиста  <<Cистема управления заданиями по автоматическому сбору данных из сети Интернет
>> и <<Руководство пользователя <<Cистема управления заданиями по автоматическому сбору данных из сети Интернет
>>, а также разбираться в терминологии \ref{terms}.
	\subsection{Требования к составу и параметру технических средств}
	
	Минимальный состав технических компонент, необходимый для нормального функционирования программы:
    
    \begin{enumerate}
        \item компьютер оснащенный процессором не ниже Intel Pentium/Celeron, или совместимый с ними с тактовой частотой не ниже 1,3 ГГц;
        \item 1 Гб ОЗУ или более;
        \item жесткий диск с объемом свободной памяти не менее 4 ГБ;
        \item клавиатура;
        \item доступ в интернет.
    \end{enumerate}
	
	
	\subsection{Требования к информационной и программной совместимости}
	
	Для нормального функционирования программы требуется компьютер, оснащенный следующими программными компонентами:
    
    \begin{enumerate}
        \item Ubuntu Server 18.04.2 LTS~\cite{ubuntu};
        \item PostgreSQL 11~\cite{postgresql};
        \item scrapyd~\cite{scrapyd};
        \item Scala 2.12.6~\cite{scala};
    \end{enumerate}
	
	
	\subsection{Требования к маркировке и упаковке}
	Приложение должно быть доступно для установки из архива проекта, при скачивании из системы LMS НИУ ВШЭ~\cite{lms}.
	
						\newpage
	\section{Требования к программной документации}
    Состав программной документации должен включать в себя следующие компоненты:
\begin{enumerate}
	\item Техническое задание <<Cистема управления заданиями по автоматическому сбору данных из сети Интернет
>> (ГОСТ 19.201-78) \label{tz}
	\item Программа и методика испытаний <<Cистема управления заданиями по автоматическому сбору данных из сети Интернет
>> (ГОСТ 19.301-78) \label{pmi}
	\item Пояснительная записка <<Cистема управления заданиями по автоматическому сбору данных из сети Интернет
>> (ГОСТ 19.404-79) \label{pz}
	\item Руководство оператора <<Cистема управления заданиями по автоматическому сбору данных из сети Интернет
>> (ГОСТ 19.505-79) \label{ro}
	\item Текст программы <<Cистема управления заданиями по автоматическому сбору данных из сети Интернет
>> (ГОСТ 19.401-78) \label{tp}
\end{enumerate}

\indent
Вся документация должна быть составлена согласно ЕСПД (ГОСТ 19.101-77, 19.104-78, 19.105-78, 19.106-78 и ГОСТ к соответствующим документам (см. выше)) \cite{gost}. Все документы сдаются в электронном виде в составе курсовой работы LMS НИУ ВШЭ.

Пояснительная записка <<Cистема управления заданиями по автоматическому сбору данных из сети Интернет
>> должна быть проверена на плагиат ($< 40\% $ заимствований). Документ, подтвержадющий проверку Пояснительной записки сдается в печатном виде вместе с подписанным отзывом от научного руководителя.

	
						\newpage
	\section{Технико-экономические показатели}
	\subsection{Предполагаемая потребность}
	Программа будет использоваться программистами, которые используют web-scraping (\ref{terms}) для отслеживания изменений, скачивания данных из сети интернет. 
	
	\subsection{Ориентировочная экономическая эффективность} 
	Полная функциональность главного аналога \cite{scrapinghub} не доступна для бесплатного использования. 
	
	Разрабатываемая система будет бесплатной и будет иметь англоязычный интерфейс.
	
						\newpage
	\section{Стадии и этапы разработки}
	
	\subsection{Необходимые стадии разработки, этапы и содержание работ}
	\begin{enumerate}
		\item \textit{Техническое задание:}
		\begin{enumerate}
			\item Этапы разработки:
			\begin{enumerate}
				\item Обоснование необходимости разработки программы; 
				\item Постановка задачи; 
				\item Сбор исходных материалов; 
				\item Выбор и обоснование критериев эффективности и качества разрабатываемой программы; 
				\item Обоснование необходимости проведения научно-исследовательских работ; 
			\end{enumerate}
			\item Разработка и утверждение технического задания:
			\begin{enumerate}
				\item Определение требований к программе; 
				\item Определение стадий, этапов и сроков разработки программы и документации на неё; 
				\item Согласование и утверждение технического задания; 
			\end{enumerate}
		\end{enumerate}
		\item \textit{Технический проект:}
		\begin{enumerate}
			\item Разработка технического проекта:
			\begin{enumerate}
				\item Уточнение структуры входных и выходных данных; 
				\item Разработка алгоритма решения задачи; 
				\item Определение формы представления входных и выходных данных; 
				\item Разработка структуры программы; 
				\item Окончательное определение конфигурации технических средств. 
			\end{enumerate}
			\item Утверждение технического проекта:
			\begin{enumerate}
				\item Разработка пояснительной записки; 
				\item Согласование и утверждение технического проекта. 
			\end{enumerate}
		\end{enumerate}
		\item \textit{Рабочий проект:}
		\begin{enumerate}
			\item Разработка программы:
			\begin{enumerate}
				\item Программирование и отладка программы. 
			\end{enumerate}
			\item Разработка программной документации:
			\begin{enumerate}
				\item Разработка программных документов в соответствии с требованиями ГОСТ 19.101-77 \cite{gost}. 
			\end{enumerate}
			\item Испытания программы:
			\begin{enumerate}
				\item Разработка, согласование и утверждение порядка и методики испытаний; 
				\item Корректировка программы и программной документации по результатам испытаний.
			\end{enumerate}
		\end{enumerate}
	\end{enumerate}
	
	% приложения нумеруются отдельно и надо выровнять по правому краю

						\newpage
	\addition{Используемые понятия и определения} \label{terms}
	\begin{description}
		\item[\textbf{Web scraping}] -- это сбор данных с различных интернет-ресурсов. Общий принцип его работы можно объяснить следующим образом: некий автоматизированный код выполняет GET-запросы на целевой сайт и получая ответ, парсит HTML-документ, ищет данные и преобразует их в заданный формат. \label{terms:webscraping}
		\item[\textbf{Проект}] -- сущность для объединения и предоставления доступа к запускам/краулерам/периодическим задачам. \label{terms:project}
		
		\item[\textbf{Веб краулер}] --  программа, являющаяся составной частью поисковой системы и предназначенная для перебора страниц Интернета с целью занесения информации о них в базу данных поисковика. Неотъемлемая часть проекта. Именно с помощью пауков пользователь может “краулить” сайты для сбора необходимой информации. \label{terms:spider}
		\item[\textbf{Запуск}] -- единоразовый запуск краулера с настройками и аргументами, указанными для этого запуска. \label{terms:job}
		\item[\textbf{Периодический запуск}] -- запуск с множеством настроек, повторяющийся в определенные периоды времени (запуски по cron-expression).
		\label{terms:pjob}
	\end{description}


						\newpage
	%\section{Источники, использованные при разработке}
	%\renewcommand{\refname}{Список источников}
	% \addcontentsline{toc}{subsection}{\refname}
	\patchcmd{\thebibliography}{\section*{\refname}}{}{}{}
	\addition{Список источников}
	\begin{thebibliography}{3}
		\bibitem{gost}Единая система программной документации – М.: ИПК, Издательство стандартов, 2000, 125 стр.
		\bibitem{lms} 
		LMS [Электронный ресурс] URL: 
		\url{https://lms.hse.ru} (Дата обращения: 16.05.2020, режим доступа: свободный)
		\bibitem{json} JSON [Электронный ресурс] URL: \url{https://www.json.org} (Дата обращения: 16.05.2020, режим доступа: свободный)
		\bibitem{rest} Representational state transfer URL: \url{https://en.wikipedia.org/wiki/Representational\_state\\\_transfer} (дата обращения: 2020.12.14).
		\bibitem{scrapinghub} Scrapinghub URL: https://scrapinghub.com (дата обращения: 2019.12.14).
		\bibitem{postgresql} Postgresql [Электронный ресурс] URL:\url{https://www.postgresql.org} (Дата обращения: 16.04.2020, режим доступа: свободный)
		\bibitem{scrapyd} Github scrapyd/scrapyd [Электронный ресурс] URL: \url{https://github.com/scrapy/scrapyd} (Дата обращения: 16.04.2020, режим доступа: свободный)
		\bibitem{ubuntu} Ubuntu [Электронный ресурс] URL: \url{https://www.ubuntu.com} (Дата обращения: 16.04.2020).
		\bibitem{scala} Scala [Электронный ресурс] URL: \url{https://www.scala-lang.org} (Дата обращения: 16.04.2020).
		
	\end{thebibliography}

						\newpage
	\listRegistration

\end{document} % конец документа